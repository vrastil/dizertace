%%% Title page of~the~thesis and~other mandatory pages

%%% Title page of~the~thesis

\pagestyle{empty}
% \hypersetup{pageanchor=false}
% \begin{center}

% \centerline{\mbox{\includegraphics[width=166mm]{logo-en-converted-to.pdf}}}

% \vspace{-8mm}
% \vfill

% {\bf\Large DOCTORAL THESIS}

% \vfill

% {\LARGE\ThesisAuthor}

% \vspace{15mm}

% {\LARGE\bfseries\ThesisTitle}

% \vfill

% \Department

% \vfill

% \begin{tabular}{rl}

% Supervisor of~the~doctoral thesis: & \Supervisor \\
% \noalign{\vspace{2mm}}
% Study programme: & \StudyProgramme \\
% \end{tabular}

% \vfill

% % Zde doplňte rok
% Prague \YearSubmitted

% \end{center}

% \newpage

% %%% Here should be a~bound sheet included -- a~signed copy of~the~"doctoral
% %%% thesis assignment". This assignment is NOT a~part of~the~electronic
% %%% version of~the~thesis. DO NOT SCAN.

% %%% A~page with~a~solemn declaration to~the~doctoral thesis

% \openright
% \hypersetup{pageanchor=true}
% \pagestyle{plain}
% \pagenumbering{roman}
% \vglue 0pt plus 1fill

% \noindent
% I declare that I carried out this doctoral thesis independently, and only with the cited
% sources, literature and~other professional sources.

% \medskip\noindent
% I understand that my work relates to~the~rights and~obligations under the~Act No.~121/2000 Sb.,
% the~Copyright Act, as amended, in~particular the~fact that the~Charles
% University has the~right to~conclude a~license agreement on~the~use of~this
% work as a~school work pursuant to~Section 60 subsection 1 of~the~Copyright Act.

% \vspace{10mm}

% \hbox{\hbox to 0.5\hsize{%
% In~Prague 17.8.2020
% \hss}\hbox to 0.5\hsize{%
% .......................
% \hss}}

% \vspace{20mm}
% \newpage

% %%% Dedication

% \openright

% \noindent

% \Dedication

% \newpage

% %%% Mandatory information page of~the~thesis

% \openright

\vbox to 0.5\vsize{
\setlength\parindent{0mm}
\setlength\parskip{5mm}

{\bf Název:}
Studium temné energie a modifikované gravitace a jejich vliv na kosmologické parametry vesmíru

{\bf Autor:}
\ThesisAuthor

{\bf Pracoviště:}
Fyzikální ústav Akademie Věd

{\bf Vedoucí práce:}
\Supervisor, Fyzikální ústav Akademie Věd

{\bf Abstrakt:}
Zrychlená expanze vesmíru představuje jednu z hlavních záhad teoretické fyziky. Ačkoli předpoklad nenulové kosmologické konstanty poskytuje minimální rozšíření obecné relativity, které je v souladu s pozorováními, bylo navrženo mnoho teorií modifikované gravitace jako možných alternativ. Předpovědi tvoření kosmických struktur pro tyto modely v nelineárním režimu jsou velmi drahé a je obtížné, ne-li nemožné, prozkoumat celý obrovský prostor modelů a parametrů pomocí \textit{N}-částicových simulací s vysokým rozlišením. Dokonce i v mírně nelineárním režimu mohou být perturbační metody nesmírně složité. V práci zkoumáme, zda zjednodušené dynamické aproximace, použitelné pro určitou sadu kosmologických pozorovatelných, mohou být použity pro zkoumání modelů modifikované gravitace s přijatelnou přesností. V případě chameleoní gravitace jsme zjistili, že její efekty jsou skryty na škálách menších než kupy galaxií. Na velkých kosmologických škálách jsme zjistili, že přibližné metody lze použít ke zkoumání baryonových akustických oscilací, $k\sim 0.1~h\text{Mpc}^{-1}$, ale ne moc dále.

{\bf Klíčová slova:}
temná energie, modifikovaná gravitace, \textit{N}-částicové simulace, přibližné metody

\vss}
