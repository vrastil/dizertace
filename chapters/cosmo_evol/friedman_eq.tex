\subsection{Friedmann equations}
The homogeneous and isotropic space-time is described by Friedmann-Lema\^{i}tre-Robertson-Walker (FLRW) metric given by
\eq{
    \label{eq:flrw}
    \dd s^2=g_\uv\dd x^\mu \dd x^\nu=-\dd t^2+a^2(t)\left[\frac{\dd r^2}{1-Kr^2} + r^2\left(\dd\theta^2+\sin^2\theta\dd\phi^2 \right) \right]\,,
}
where $g_{\mu\nu}$ is a metric tensor, $a(t)$ is a scale factor with cosmic time $t$ and $K=+1,-1,0$ is a curvature that corresponds to closed, open, and flat geometries. The scale factor can be arbitrary rescaled and we adopt the common normalization $a_0=1$. We can also transform metric \eqref{eq:flrw} to a more convenient form by setting $r=\sin\chi\ (K=+1),r=\chi\ (K=0)$, and $r=\sinh\chi\ (K=-1)$. The FLRW metric is then given by
\eq{
    \label{eq:flrw_chi}
    \dd s^2=-\dd t^2+a^2(t)\left[\dd \chi^2 + f_K^2(\chi)\left(\dd\theta^2+\sin^2\theta\dd\phi^2 \right)\right]\,,
}
where
\eq{
    f_K(\chi)=
    \begin{cases}
        \sin\chi & (K=+1)\,, \\
        \chi & (K=0)\,, \\
        \sinh\chi & (K=-1)\,.
    \end{cases}
}
By allowing taking the limit $K\to0$ we can rewrite $f_K(\chi)$ in a unified way
\eq{
    \label{eq:f_K}
    f_K(\chi)=\frac{1}{\sqrt{-K}}\sinh{(\sqrt{-K}\chi)}\,.
}
In addition to the cosmic time $t$, we also introduce the conformal time $\eta$ defined by
\eq{
    \eta\equiv\int{a^{-1}\dd t}\,.
}
The dynamical equations of motion in the expanding Universe can be derived from the Einstein equations resulting in the Friedmann equations
\eq{
    \label{eq:Friedmann}
    H^2 &= \frac{8\pi G}{3}\rho - \frac{K}{a^2}+\frac{\Lambda}{3}\,, \\
    \label{eq:Friedmann_2}
    \frac{\ddot a}{a} &= -\frac{4\pi G}{3}(\rho + 3p)+\frac{\Lambda}{3}\,,\\
    \label{eq:Friedmann-continuity}
    \dot\rho &= -3H(\rho+p)\,,
}
where the overdot denotes a time derivative with respect to $t$, $H\equiv\dot a/a$ is the Hubble parameter, $\Lambda$ the cosmological constant (more details in \autoref{ssec:lambda}), $\rho(t)$ is energy-density and $p$ pressure. We will be also using the conformal Hubble quantity
\eq{
    \mathcal{H}\equiv\frac{1}{a}\frac{\dd a}{\dd\eta}=Ha\,.
}
These three equations are not independent and we need to also specify the equation of state, $\rho=\rho(p,t)$. The first equation is usually also written in the form
\eq{
    \Omega_m+\Omega_\gamma+\Omega_K+\Omega_\Lambda=1\,,
}
where
\eq{
    \label{eq:omega}
    \Omega_m\equiv\frac{8\pi G\rho_m}{3H^2}\,,\hspace{3pt}
    \Omega_\gamma\equiv\frac{8\pi G\rho_\gamma}{3H^2}\,,\hspace{3pt}
    \Omega_K\equiv-\frac{K}{a^2H^2}\,,\hspace{3pt}
    \Omega_\Lambda\equiv\frac{\Lambda}{3H^2}
}

Most matter species in the Universe can be described by a simple equation of state
\eq{
    p=w\rho\,,
}
where $w$ is constant. This lead to the following solutions in flat universe
\eq{
    \rho\propto a^{-3(1+w)}\,,\hspace{6pt}a\propto (t-t_i)^{2/(3(1+w)))}\,,
}
where $t_i$ is an integration constant. Relativistic matter has the equation of state $w=1/3$ and the cosmic evolution during the radiation-dominated epoch is given by $\rho\propto a^{-4}$ and $a\propto(t-t_i)^{1/2}$. Non-relativistic matter has negligible pressure $(w=0)$ and the evolution during the matter- dominated era is given by $\rho\propto a^{-3}$ and $a\propto(t-t_i)^{2/3}$.

The accelerated expansion of the Universe $(\ddot a>0)$ without the cosmological constant is possible only for $w<-1/3$ (see Friedman equation \eqref{eq:Friedmann_2}), i.e. negative pressure. Note that in Newtonian gravity there is no such equivalent. The pressure is related to a force associated with a local potential that depends on the position in space but in homogeneous and isotropic universe there is no such potential.

When $w=-1$, the energy-density $\rho$ is constant and corresponds to the cosmological constant. This energy-density connected to the cosmological constant is
\eq{
    \rho_\Lambda=\frac{\Lambda}{8\pi G}\,.
}
With constant energy-density the Friedmann equations lead to an exponential expansion $a\propto\exp{Ht}$.
