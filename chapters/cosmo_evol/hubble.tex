\subsection{Hubble`s law}
During the 1920s astronomers Slipher and Hubble found that the observed wavelength $\lambda_o$ of absorption lines of distant galaxies is larger than the wavelength $\lambda$ in the rest frame \parencite{1929PNAS...15..168H}. This is due to the fact that the wavelength is stretched in proportion to the scale factor in an expanding Universe. The redshift $z$ is defined as
\eq{
    z\equiv\frac{\lambda_0}{\lambda} - 1=a^{-1}-1\,.
}
In an expanding Universe a physical distance $r$ from an observer (at the origin) to an object is given by $r=a(t)x$, where $x$ denotes the comoving distance. For objects which are motionless with respect to the Hubble flow the comoving distance remains constant. Taking the time derivative of $r$ we obtain
\eq{
    v_H\equiv\dot r=Hr+a\dot x
}
Because of the cosmic expansion, more distant objects are moving faster from us with velocity $v_H$. On the other hand, the peculiar velocity $v_p\equiv a\dot x$ describes the movement of an object with respect to the local Hubble flow. For small peculiar velocities and near objects $(z\ll1)$ we obtain
\eq{
    v\sim H_0r\,,
}
which is the law Hubble reported in 1929 by plotting the recessional velocity $v$ versus the distance $r$. His measurements were noisy but the current measurements give the Hubble constant the value \parencite{planck_cosm}
\eq{
    H_0=(67.4\pm0.5)\rm{\ km\ s}^{-1}\rm{Mpc}^{-1}
}