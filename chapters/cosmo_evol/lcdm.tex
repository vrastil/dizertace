\section{Standard \LCDM model -- successes and issues}
\todo{Move elsewhere?}

Standard cosmological model, the \LCDM\ model or the concordance model, assumes that the Universe was created in the Big Bang from infinitely hot and dense energy and now is the Universe composed of about 5\% ordinary matter, 27\% dark matter, and 68\% dark energy \parencite{redefineLCDM}. The \LCDM\ model is based upon two theoretical models -- the Standard Model of Particle Physics and  the General Theory of Relativity. The model also assumes that the universe is homogeneous and isotropic on sufficiently large (cosmic) scales. The model is mathematically described above.

The \LCDM\ model represents general relativity with a cosmological constant \(\Lambda\) which is associated with dark energy and a universe containing sufficiently massive dark matter particles, i.e., cold dark matter. However, the nature of both dark energy and dark matter is unknown.

In 1948, \textcite{PhysRev.74.505.2} suggested that the elements could have been made during the early hot matter-energy phase associated with the Big Bang and predicted their representation -- hydrogen about 75\%, helium about 25\% and small amounts of deuterium, lithium, and other light elements.

The other great success of the Big bang model is prediction of the Cosmic Microwave Background (CMB) radiation and its temperature. In 1948, \textcite{1948Natur.162..774A} calculated the present temperature of the CMB to be about 5 K, remarkably close to the modern value of about 2.73 K, determined by the COBE satellite. In addition, the COBE results showed an extremely isotropic and homogeneous CMB. This led to the need for an inflationary phase \parencite{1981PhRvD..23..347G} of strongly accelerated expansion prior to the decoupling of photons from ordinary matter.

The \LCDM\ model has additional major assumption about existence of dark matter. The notion of dark matter arose from observations of large astronomical objects such as galaxies and clusters of galaxies, which displayed gravitational effects that could not be accounted for by the visible matter. In particular, the observations of \textcite{1980ApJ...238..471R}, who measured the rotation curves for the luminous matter of many spiral galaxies together with the observations of \textcite{1978PhDT.......195B}, who compiled 21 cm rotation curves for neutral hydrogen gas that extended far beyond the luminous matter of each galaxy, showed that the composite rotation curves were essentially flat out to the edge of the 21 cm data. This implied that considerably more mass was required to be present in each galaxy. This invisible matter was called dark matter and since to date no dark matter has been definitely detected and the nature of dark matter remains unknown.

The notion of dark energy arose from two sets of observations of supernovae of Type Ia by \textcite{riess} and \textcite{1999ApJ...517..565P} that suggested that the expansion of the universe is accelerating. The conclusion from these observations was that the universe had to contain enough energy to overcome gravity. This energy was named “dark energy.”

\subsection{Cosmological constant}

\todo{Heyrovsky D.:

\textit{
Specific comments on the scientific content: the cosmological-constant term is a natural part of the general form of Einstein’s equations, not a random ad-hoc term; most “problems” with Lambda are connected with its interpretation as vacuum energy density, they do not occur in its interpretation simply as a fundamental constant; the work could have included the observationally most relevant result: the cosmological evolution of the equation of state w(a) predicted by the different modified theories; weak lensing can be analyzed for rich galaxy clusters rather than for single galaxies.}
}

\todo{\parencite[e.g.][]{2015arXiv150205296P, 2012CRPhy..13..566M}}

\todo{Additional arguments against $\Lambda$ -- quantum corrections, etc.}
Standard cosmological \LCDM\ model is in a good agreement with all measurements of CMB \parencite{planck_cosm}, type Ia supernovae \parencite{Abbott_2019}, or BAO \parencite{BAO_results}. However, this concordance model has some significant fundamental problems and even though it may serve as an effective description of our Universe we should be looking for a deeper explanation of the accelerated expansion of the Universe.
\subsubsection{Fine-Tuning Problem}
The first problem could be phrased as the question: \textit{Why is the observed value of $\Lambda$ so small in Planck units?} In order to realize the cosmic acceleration today, we require that the cosmological constant $\Lambda$ is of the order of the square of the present Hubble parameter $H_0$. If we interpret this as an energy density, it is equivalent to
\eq{
\rho_\Lambda=\Lambda \Mpl^2\approx10^{-120} \Mpl^4\approx10^{-47}\mbox{ GeV}^4.
}
Suppose that this energy density comes from the vacuum energy. The zero-point energy of some field of mass $m$ with momentum $k$ is
\eq{
E_0=\left\langle0\left|H\right|0\right\rangle=V\int{\frac{\dd^3k}{(2\pi)^3}\frac12\sqrt{k^2+m^2}},
}
where $V=(2\pi)^3\delta^3(0)$ is the volume of space. If we trust our theory up to some cut-off scale $\Lambda_{UV}\gg m$, we obtain the vacuum energy density
\eq{
\rho_{vac}=\int_0^{\Lambda_{UV}}{\frac{4\pi k^2 \dd k}{(2\pi)^3}}\frac12\sqrt{k^2+m^2}\approx\frac{\Lambda_{UV}^4}{16\pi^2}.
}
If we take the cut-off scale to be order of the Planck scale $\Mpl\approx10^{18}$ GeV, we get a value of about 120 orders of magnitude larger than the observed value. This situation is not better for different scales in particle physics. For the SUSY\footnote{In some supersymmetric theories, the number of fermionic and bosonic degrees of freedom are equal. The energy of the vacuum fluctuations per degree of freedom is the same in magnitude but opposite in sign for fermions and bosons of the same mass. Therefore the fermion and boson contributions cancel each other and the total vacuum energy density (and consequently $\Lambda$) vanishes. As the supersymmetry has to be broken today (we do not observe supersymetry in nature) supersymmetric partners can have different masses of order $\Lambda_{SUSY}\approx10^{3}$ GeV.} scale $\Lambda_{SUSY}\approx10^{3}$ GeV, even for the QCD scale $\Lambda_{QCD}\approx0.1$ GeV, is $\rho_{vac}$ still much larger than $\rho_\Lambda$. Even if the value of $\Lambda$ does not originate from the vacuum energy there is no such a small scale in known physics -- not in GR nor in quantum physics. This huge discrepancy between theoretical predictions and observed value of $\Lambda$ is a major issue in physics and cosmology.
\subsubsection{Coincidence Problem}
The second problem  could be phrased as the question: \textit{Why is the energy density $\rho_\Lambda$ so close to the present matter density?} While the energy density of the cosmological constant (vacuum energy) remains constant in time and was completely negligible in most of the past, the energy density of matter evolves like $\rho_m\propto(1+z)^3$ and will be entirely negligible in the future. According to the Copernican principle \parencite{ellis_maartens_maccallum_2012} we do not live in a special place \textbf{nor time}. Thus it is very unlikely that these two components will have densities of the same order of magnitude in the present. If $\rho^{(0)}_\Lambda/\rho^{(0)}_m$ was just 10 or 100 times smaller, we would not see any accelerated expansion. If it were a few orders of magnitude larger than one, the transition to the accelerated universe would occur at a large redshift.
\\

The so-called anthropic principle can seemingly give us the explanation for both of the two cosmological constant problems, why it is small and why the acceleration starts now. Because if the vacuum energy has been big and dominant from the earlier epoch, there would be no chance to form structures in the Universe, like galaxies, stars, planets and us, intelligent lives. But this anthropic explanation of the value $\rho_\Lambda$ makes sense only if there is a multiverse with a lot different realizations of $\rho_\Lambda$. But we live only in our realization of the Universe and therefore we cannot verify whether the anthropic principle solves the cosmological constant problem or not.

Beside these two problems concerning the cosmological constant, the observations of coherent acoustic oscillations in the CMB \parencite{planck_cosm} has turned the notion of accelerated expansion in the very early universe (inflation) into an integral part of the cosmic standard model. This early accelerated expansion was not due to the cosmological constant, because in that case the inflation would not stop and today`s universe would not be possible. Therefore, we have to postulate some new scalar field (inflaton) \parencite{2015IJMPD..2430010C} that we know so little about. If we do not properly understand the past dynamics of the universe how can we accept the cosmological constant as an explanation of the present acceleration without doubt?