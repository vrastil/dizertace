\section{Cosmological Surveys}
There are many projects and~missions which study properties of~the~dark energy, either as a~main scientific goal or as a~complementary program. Here we mention some of the biggest surveys while directing the reader for further references.
\subsection{Sloan Digital Sky Surveys}
Sloan Digital Sky Surveys (SDSS, \cite{SDSS}) aims to~create the~most detailed three-dimensional maps of~the~Universe. From the~beginning of~regular surveys in~2000 till 2014 there were seven finished surveys in~total (SDSS-I/II results \cite{SDSS_I_II}, SDSS-III results \cite{BOSS_results}), while there are three ongoing surveys from 2014 (SDSS-IV, \cite{2017AJ....154...28B}) and~a~planned panoptic spectroscopic survey (SDSS-V, \cite{2017arXiv171103234K}) which will start collecting data in~summer 2020. The~Baryon Oscillation Spectroscopic Survey (BOSS, \cite{BOSS}) was a~six-year program (Fall 2009 -- Spring 2014) designed to~measure the~scale of~baryonic acoustic oscillations (BAO, see \autoref{sec:bao}) in~the~clustering of~matter. The~Extended Baryon Oscillation Spectroscopic Survey (eBOSS, \cite{2016AJ....151...44D}) is the~new cosmological survey within a~SDSS-IV six-year program. eBOSS conducts observations of~galaxies and quasars and will expand the~selection of~luminous red galaxies beyond that probed by BOSS.
\subsection{Dark Energy Survey}
The~Dark Energy Survey (DES) is a program designed to~uncover the~nature of~dark energy by measuring the history of~cosmic expansion with~high precision. DES is an~optical near-infrared survey of~5000 deg\sq\ of~the~South Galactic Cap. Starting in~August of~2013 and~continuing till January 2019, DES begun to~survey a~large swath of~the~southern sky out to~vast distances in~order to~provide new clues to~these most fundamental questions \parencite{DES}.
\subsection{Euclid}
Euclid is an~ESA (European Space Agency) high-precision space mission designed to study dark matter and dark energy through mapping the~geometry and~evolution of~the~Universe \parencite{euclid,2010arXiv1001.0061R}. For this purpose, the~Euclid will mainly use two independent cosmological probes -- weak gravitational lensing and~baryonic acoustic oscillation -- out to~redshift $z\sim2$. As complementary probes, Euclid will use galaxy clusters and~the~Integrated Sachs--Wolfe effect. The Euclid mission will start in~mid-2022. The~overview of~the~Euclid system design and~scientific requirements can be found in~\cite{2011arXiv1110.3193L}.
\subsection{Vera C. Rubin Observatory}
The~Vera C. Rubin Observatory project, previously known as the~Large Synoptic Survey Telescope, will conduct the~10-year Legacy Survey of~Space and~Time (LSST, \cite{lsst}). LSST is a~ground-based telescope in Chile and will produce a~6-band (300 -- 1100 nm) wide-field deep astronomical survey over 20,000 deg\sq\ of~the~southern sky. The LSST will scan the sky very rapidly (more than 800 images each night) and each patch of~the~sky will be visited about 1000 times during the whole survey. LSST`s data will be used for studying the dark matter and~the~dark energy. LSST will also detect and~track potentially hazardous asteroids. The~project is in~the~construction phase and~will begin its full science operations in~2022 \parencite{lsst_web}.
\subsection{Planck}
The~Planck mission (\cite{planck}) was a~European Space Agency mission with~significant participation from NASA. It was launched into space on~May 14, 2009, and~was orbiting the~second Lagrange point of~our Earth-sun system, about 1.5 million km (930,000 miles) away. Planck was measuring the~Cosmic Microwave Background (CMB) over a~broad range of~far-infrared wavelengths. The mission`s goal was to study the~geometry and~contents of~the~Universe \parencite[for results, see][]{planck_cosm}. Planck`s mission ended on~23 October 2013, after nearly 4.5 years of operations \parencite{planck_web}.
\subsection{Nancy Grace Roman Space Telescope}
The~Nancy Grace Roman Space Telescope (formerly known as WFIRST, the~Wide Field Infrared Survey Telescope) is a~NASA space mission designed to~study dark energy, exoplanets, and~infrared astrophysics \parencite{wmap_web}. It will perform wide-field imaging and spectroscopic survey of~the~near-infrared sky. These data will be used to determine the~expansion history of~the~Universe and~the~growth history of~larges-scale structures. WFIRST`s mission should start in~the~mid-2020s \parencite{WFIRST_report}.
