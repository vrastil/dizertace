\subsection{Correlation Function}
\label{sec:corr}
The~two-point correlation function is defined by the~equation \eqref{eq:corr}. We computed the~correlation function as an~inverse Fourier transformation of~\eqref{eq:pk_xi} as
\eq{
\label{eq:bao}
\xi(r)=\left\langle \delta(\mb x)\delta(\mb{x+r})\right\rangle=\int\frac{\dd^3\mb k}{(2\pi)^3}P(k)e^{i\mb{k\cdot r}}\,.
}
We computed the~correlation function $\xi(r)$ for~different approximation schemes at~different redshifts directly from the~measured matter power spectrum $P(k)$ using equation \eqref{eq:bao}. In~\autoref{fig:corr_func}, we display an~example of~the~correlation function at~redshift $z=0.5$. All approximations agree reasonably with~the~full non-linear predictions of~the~emulator for~the~location of~the~BAO peak and~its width.
\begin{figure}[bt]
\centering
	\begin{subfigure}{0.9\textwidth}
        \includegraphicscustomlegend{simulations_approx/corr_func/corr_func_r2_z_z_eff}
	\end{subfigure}
	\begin{subfigure}{0.9\textwidth}
		\centering
		\includegraphicscustom{simulations_approx/corr_func//corr_func_r2_z_z_eff}
	\end{subfigure}
	\caption{Two-point correlation function for~different approximation schemes at~$z=0.5$}.
	\label{fig:corr_func}
\end{figure}

Individual features of~the~BAO peak -- location $r_0$, amplitude $A$ and~width $\sigma$ -- are obtained by fitting a~Gaussian to~$r^2\xi(r)$ around the~BAO peak
\eq{
\xi_G(r)=A\cdot e^{-(r-r_0)^2/2\sigma^2}
}
In~\autoref{fig:corr_peak}, we show these features of~the~BAO peak relative to~the~non-linear predictions of~the~emulator as a~function of~time. All approximations can predict the~location of~the~peak with~1\% accuracy. In~predicting the~shape of~the~peak, however, they do worse -- all approximations deviate in~predictions for~the~peak width and~amplitude. Note that which approximation does best (compared to~the~others) for~a~given quantity is a~function of~redshift.

%From comparison with~the~linear theory we see that FFA and~FPA are predicting something between the~linear and~non-linear prediction but ZA and~TZA predict values quite off the~linear prediction as one would expect from these approximations.

\begin{figure}[tbp]
\newcommand{\corrwidth}{0.70}
\centering
	\begin{subfigure}{\corrwidth\textwidth}
        \includegraphicscustomlegend{simulations_approx/corr_func/corr_peak_loc_z_eff}
	\end{subfigure}
	\begin{subfigure}{\corrwidth\textwidth}
		\centering
		\includegraphicscustom{simulations_approx/corr_func//corr_peak_loc_z_eff}
		\caption{Location}
	\end{subfigure}
	\begin{subfigure}{\corrwidth\textwidth}
		\centering
		\includegraphicscustom{simulations_approx/corr_func//corr_peak_amp_z_eff}
		\caption{Amplitude}
	\end{subfigure}
	\begin{subfigure}{\corrwidth\textwidth}
		\centering
		\includegraphicscustom{simulations_approx/corr_func//corr_peak_width_z_eff}
		\caption{Width}
	\end{subfigure}
	\caption{Location, amplitude and~width of~the~BAO peak (relative to~the~non-linear prediction) as a~function of~the~redshift.}
	\label{fig:corr_peak}
\end{figure}

%%%%%%%%%%%%%%%%%%%%%%%
% Halo mass function
%%%%%%%%%%%%%%%%%%%%%%%
\subsection{Halo mass function}
We computed the~halo mass function \eqref{eq:hmf} from the~amplitude of~density fluctuations for~all approximation schemes. Four our simple analysis we just use the~fitting formula \eqref{eq:hmf_jenkins} instead of~implementing the~full friend-of-friend algorithm.

The~comparison of~approximations in~mass range $(10^{11}-10^{15}M_\odot)$, is shown in~\autoref{fig:hmf_diff_z}. We see that TZA for~higher redshifts is completely wrong as its missing a~lot of~power on~these scales. At~higher redshifts, this gets better as the~TZA gets more power on~these scales without so much shell-crossing as ZA. We see that for~$z=0$ TZA gives results closer to~the~linear prediction than ZA.
\begin{figure*}[p]
	\thisfloatpagestyle{empty}
	\centering
	\begin{subfigure}{1.0\textwidth}
		\includegraphicscustomlegend{simulations_approx/hmf/z0.0_b100_M1024_hmf_z_z_eff}
	\end{subfigure}
	\begin{subfigure}{0.5\textwidth}
		\includegraphicscustom{simulations_approx/hmf/z3.9_b100_M1024_hmf_z_z_eff}
		\caption{$z=\zone$}
	\end{subfigure}%
	\begin{subfigure}{0.5\textwidth}
		\includegraphicscustom{simulations_approx/hmf/z1.8_b100_M1024_hmf_z_z_eff}
		\caption{$z=\ztwo$}
	\end{subfigure}
	\begin{subfigure}{0.5\textwidth}
		\includegraphicscustom{simulations_approx/hmf/z0.5_b100_M1024_hmf_z_z_eff}
		\caption{$z=\zthree$}
	\end{subfigure}%
	\begin{subfigure}{0.5\textwidth}
		\includegraphicscustom{simulations_approx/hmf/z0.0_b100_M1024_hmf_z_z_eff}
		\caption{$z=\zfour$}
	\end{subfigure}
	%
	% relative
	%
	\begin{subfigure}{0.5\textwidth}
		\includegraphicscustom{simulations_approx/hmf/diff_z3.9_b100_M1024_hmf_z_z_eff}
		\caption{$z=\zone$}
	\end{subfigure}%
	\begin{subfigure}{0.5\textwidth}
		\includegraphicscustom{simulations_approx/hmf/diff_z1.8_b100_M1024_hmf_z_z_eff}
		\caption{$z=\ztwo$}
	\end{subfigure}
	\begin{subfigure}{0.5\textwidth}
		\includegraphicscustom{simulations_approx/hmf/diff_z0.5_b100_M1024_hmf_z_z_eff}
		\caption{$z=\zthree$}
	\end{subfigure}%
	\begin{subfigure}{0.5\textwidth}
		\includegraphicscustom{simulations_approx/hmf/diff_z0.0_b100_M1024_hmf_z_z_eff}
		\caption{$z=\zfour$}
	\end{subfigure}
	\caption{Halo mass function for~different approximation schemes at~different redshifts. Top four (a -- d) are absolute functions while four bottom ones (e -- h) represents HMFs relative to~non-linear prediction of~\LCDM.}
	\label{fig:hmf_diff_z}
\end{figure*}
\floatpagestyle{plain}

Also in~\autoref{fig:hmf_diff_z} we show the~halo mass function in~more detail, mainly as a~relative difference between prediction of~approximation schemes and~prediction of~the~non-linear theory. We see that approximation schemes generally predicts less massive halos $(M\gtrsim10^{12}M_\odot)$ and~more light halos $(M\lesssim10^{12}M_\odot)$. We see that simple PM simulations get results close to~the~non-linear prediction, except at~higher redshift.