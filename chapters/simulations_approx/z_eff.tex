\subsection{Effective Redshift}
\label{sec:z_eff}
%%%%%%%%%%%%%%%%%%%%%%%%%%%%%%%%%%%%%%%%
%%%%%  -> suppression
%%%%%%%%%%%%%%%%%%%%%%%%%%%%%%%%%%%%%%%%
The power spectrum growth is slower than the linear prediction in the case of FFA or FPA. For FFA, this can be understood from the equation of motion \eqref{eq:FFA}. As the particles approach the minimum of the gravitational well, their velocities decrease as the gradient of the potential drops. In these approximations, as the velocity potential is constant in time, there is no change in slope as more and more particles arrive into the gravitational well, which is certainly not realistic. (If possible, the resulting artifacts should be corrected when using these approximations.)

For FPA, the reason for the suppression is very similar, although not as significant. Particles are not moving without any memory of their previous positions as in the case of FFA. They do not stop at the bottom of the well but rather oscillate around it. However, equation \eqref{eq:FPA} still drives them toward the velocities of FFA and they slow down inside gravitational wells.

Note that this effect of slower growth due to a decreasing gradient of the velocity potential is not bound to some particular scale -- it occurs both in shallow large wells as well as in deep and more concentrated wells. Consequently, it can be compensated via a correction factor. A similar situation occurs in particle-mesh codes when the number of time-steps is restricted because the assumption of constancy in velocity and force (the ``drift'' and ``kick'' terms in a standard integrator) fails to hold. In approximate particle-mesh codes, this error may be compensated by employing a ZA-motivated correction~(\cite{Ref:Feng}) in the time-stepping. The same technique could be applied here by combining the approximations in a suitable way or, as we do, by simply calibrating against linear theory, which yields the same result.

Additionally, the dynamical approximations in FFA and FPA can lead to artificial non-linear enhancement of the growth at later times on small scales within deep wells. It takes more time to get into a local gravitational well than in standard \nbody\ but once the particles are there, they may form stable cores as they simply move towards the local potential minimum. On these small scales, the approximations are not valid in any case, so this point is mostly academic.

As discussed above, the suppression of linear growth in FFA and FPA can be viewed as a modified growth function and we can introduce a simple rescaling via an effective redshift $z\eff$ so that the power spectrum on large linear scales matches the linear prediction
\eq{
	\langle P(k, z\eff) \rangle = \langle P\lin(k, z)\rangle\,,
}
where we are averaging over ``large scales''. In our simulations this mean we are fitting the linear power spectrum in the range from the minimum available wave-number $k_\text{\tiny min}=2\pi/L$ to half a decade $k=\sqrt{10}k_\text{\tiny min}$.

In the rest of the paper when we are comparing our results with a prediction of the linear or non-linear theory, we use this effective time instead of the simulation time unless stated otherwise.

In \autoref{fig:D_eff_Pk}, we compare the effective growth function with the linear growth function of \LCDM. For ZA and TZA there is almost no suppression as we are comparing large scales. However, at later times (or if we had used smaller scales) we would see an exponential suppression as described in \cite{Bharadwaj_1996}. For FFA and FPA we have an almost linear dependence of $D\eff$ on $a$, for FFA with a slope of approximately $8\%$ and for FPA with the slope being approximately $6\%$.
\begin{figure}
  \centering
    
    \begin{subfigure}{\textwidth}
        \includegraphicscustomlegend{simulations_approx/z_eff/D_eff_Pk}
	\end{subfigure}
	\begin{subfigure}{\textwidth}
        \includegraphicscustom{simulations_approx/z_eff/D_eff_Pk}
	\end{subfigure}
  \caption{Effective growth factor $D\eff$ for different approximation schemes based on ratio of power spectrum on large scales compared to the linear prediction.}
  \label{fig:D_eff_Pk}
\end{figure}

In \autoref{fig:timestep} we study the dependence of $D\eff$ on the number of time-steps. A larger number of time-steps improves $D\eff$ for FPA but, as expected, cannot eliminate the effect. In the case of FFA, $D\eff$ actually decreases with increase in number of time-steps. In this case, for smaller number of time-steps, particles do not evolve exactly along characteristic curves of the (initial) potential and the deceleration is suppressed.
\begin{figure}
  \centering
    
    \begin{subfigure}{\textwidth}
        \includegraphicscustomlegend{simulations_approx/z_eff/timesteps}
	\end{subfigure}
	\begin{subfigure}{\textwidth}
        \includegraphicscustom{simulations_approx/z_eff/timesteps}
	\end{subfigure}
  \caption{Effective growth factor $D\eff$ at $z=0$ for FFA and FPA as a function of number of time-steps.}
  \label{fig:timestep}
\end{figure}

After the linear theory compensation via the effective time, all approximations match the linear prediction on large scales (by definition), however, on smaller scales there are considerable differences. In the case of ZA (and TZA) at early times there is enhancement of power above linear theory, but at later times, the lack of the ability to follow small-scale structure (``diffusion'' at caustics) causes a suppression of power that continues to leak to larger scales. For FFA and FPA, however, the behavior is quite different. At early times particles on the smallest scales are almost in the minimum of the local gravitational potential (and consequently at the minimum of the velocity potential) and do not evolve much. This effect is weaker for FPA where particles have inertia and are not overdamped. At later times more and more particles end up in these (constant-in-time) gravitational wells and we observe (partial) non-linear gravitational clustering.
