\subsection{Matter Power Spectrum}
\label{sec:pwr_spec}
The power spectrum is defined by equation~\eqref{eq:pk}. In \autoref{fig:pwr_spec_all} we show the power spectra $P(k)$ at redshift $z=0$ for the different approximation schemes. The gray areas represent variation across different realizations (different simulation runs). We see that on large (linear) scales all approximations reproduce the linear theory prediction (note that this is because the FFA and FFP results have been compensated for a slower growth, as described below). On smaller scales, differences between the approximations become apparent. The ZA loses power compared to the linear power spectrum on these scales due to shell-crossing. In contrast, FFA and FPA do much better and can even partially simulate non--linear clustering.
% In \autoref{fig:pwr_spec_all}, we used the so--called \textit{effective time}, described in the following subsection, to study the differences in the \textit{shape} of power spectra between the individual approximations. In the following we will show that FFA and FPA actually have slower growth than linear theory.

\begin{figure}[hbt]
\centering
	\begin{subfigure}{0.9\textwidth}
        \includegraphicscustomlegend{simulations_approx/pwr_spec/pwr_spec}
	\end{subfigure}
	\begin{subfigure}{0.9\textwidth}
		\includegraphicscustom{simulations_approx/pwr_spec/pwr_spec}
	\end{subfigure}%
    \caption{Matter power spectrum $P(k)$ at redshift $z=0$ for different approximation schemes. Grey areas represent variations across different runs. Due to shell-crossing, ZA filters the power at higher $k$, falling below the linear power spectrum, whereas FFA and FPA retain some features of the full non--linear clustering.}
    \label{fig:pwr_spec_all}
\end{figure}

%%%%%%%%%%%%%%%%%%%%%%%%%%%%%%%%%%%%%%%%
%%%%%  -> power spectrum difference
%%%%%%%%%%%%%%%%%%%%%%%%%%%%%%%%%%%%%%%%
\subsubsection{Comparison with linear theory}
The relative differences between power spectra and the linear predictions at different redshifts are plotted in \autoref{fig:pwr_spec_diff}. By the linear prediction in this case is meant the initial power spectrum of the realization linearly evolved to the given epoch. The error bars once again represent variation across different realizations but individual differences are computed from the same realization of the power spectrum.
\begin{figure*}[!hbt]
\centering
	\begin{subfigure}{1.0\textwidth}
        \includegraphicscustomlegend{simulations_approx/pwr_spec/pwr_spec_diff_init_ZA}
	\end{subfigure}
	\begin{subfigure}{0.5\textwidth}
		\includegraphicscustom{simulations_approx/pwr_spec/pwr_spec_diff_emu}	
		\caption{\LCDM\ (nl)}	
	\end{subfigure}
	\begin{subfigure}{0.5\textwidth}
		\includegraphicscustom{simulations_approx/pwr_spec/pwr_spec_diff_init_ZA}
		\caption{Zel'dovich approximation}
	\end{subfigure}%
	\begin{subfigure}{0.5\textwidth}
		\includegraphicscustom{simulations_approx/pwr_spec/pwr_spec_diff_init_TZA}
		\caption{Truncated Zel'dovich approximation}
	\end{subfigure}
	\begin{subfigure}{0.5\textwidth}
		\includegraphicscustom{simulations_approx/pwr_spec/pwr_spec_diff_init_FF}
		\caption{Frozen-flow approximation}
	\end{subfigure}%
	\begin{subfigure}{0.5\textwidth}
		\includegraphicscustom{simulations_approx/pwr_spec/pwr_spec_diff_init_FP}
		\caption{Frozen-potential approximation}
	\end{subfigure}
	\caption{Relative differences between power spectra of approximations and the linear prediction at different redshifts. ZA predicts power spectra at large scales very well but fails on small scales at later times. FFA and FPA do not have this problem at small scales but the power spectrum grows more slowly across all scales. The (CosmicEmu) non--linear power spectrum is shown in the upper panel for comparison.}
	\label{fig:pwr_spec_diff}
\end{figure*}

The upper panel shows the non--linear power spectrum $P(k)$ generated by the CosmicEmu emulator \parencite{Heitmann:2015xma} for comparison purposes. The emulator predictions involve interpolations across results from a finite number of high-resolution simulations run with different cosmological parameters; the results are accurate at the level of a few percent.

As expected, the ZA predicts power spectra at large scales very well but fails on small scales at later times. In the case of FFA and FPA, the power spectrum growth is slower than linear theory predicts but, unlike ZA, this behavior is across all scales and there is significantly less suppression of power on smaller scales.