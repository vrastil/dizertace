\chapter{Cosmological Surveys}
There are many other projects and mission which study properties of the dark energy, either as a main scientific goal, or as a complementary program. Among the big surveys are e.g. Sloan Digital Sky Surveys (SDSS, official site \cite{SDSS}) -- from the beginning of regular surveys in 2000 till 2014 there were seven finished surveys in total (SDSS-I/II results \cite{SDSS_I_II}, SDSS-III results \cite{BOSS_results}), while there are three ongoing surveys from 2014 (SDSS-IV). Other surveys are e.g. Wilkinson Microwave Anisotropy Probe (WMAP, official site \cite{WMAP}, results \cite{WMAP_results}); Planck (official site \cite{planck}, results \cite{planck_cosm}); Hobby-Eberly Telescope Dark Energy Experiment (HETDEX, official site \cite{hetdex}); or Wide field Multi-Object Spectrograph (WFMOS, Gemini study report \cite{WFMOS}). In this section we just briefly introduce three further examples -- finished BOSS, ongoing DES, and future WFIRST.


\section{BOSS}
The Sloan Digital Sky Survey`s (SDSS-III) Baryon Oscillation Spectroscopic Survey (BOSS) is a six-year program (Fall 2009 -- Spring 2014)  that uses the wide-field 2.5-m telescope at Apache Point Observatory. The BOSS is designed to measure the scale of BAO in the clustering of matter over a larger volume than the combined efforts of all previous spectroscopic surveys of large-scale structures. BOSS uses 1.5 million luminous galaxies to measure BAO to redshifts $z<0.7$. Observations of neutral hydrogen in the Ly$\alpha$ forest in more than 150,000 quasar spectra constrain BAO over the redshift range $2.15 < z < 3.5$ \cite{BOSS}.

There are two double spectrographs, each covering the wavelength range 361 nm -- 1014 nm with resolution $R=\lambda/\Delta\lambda$ ranging from 1300 at the blue end to 2600 at the red end.  Both spectrographs have a red channel with a 4k $\times$ 4k, 15\um  pixel CCD from Lawrence Berkeley National Laboratory (LBNL). Both spectrographs have a blue channel with a 4k $\times$ 4k, 15\um  pixel CCD from  e2v. The instrument is fed by 1000 optical fibers (500 per spectrograph), each subtending 2'' on the sky.

Using the acoustic scale as a physically calibrated ruler, BOSS determines the angular diameter distance with a precision of 1\% at redshifts $z = 0.3$ and $z = 0.57$ using the distribution of galaxies and measurements of $H(z)$ to 1.8\% and 1.7\%  at the same redshifts. At redshifts $z\sim2.5$ the  angular diameter distance and $H\mins(z)$ is measured to an accuracy of 1.9\% using Ly$\alpha$ forest.

BAO measurements with the CMB-calibrated physical scale of the sound horizon and SN data yields of $H_0=(67.3\pm1.1)$ \unith\ with 1.7\% precision \cite{BOSS_results}. This measurement assumes standard pre-recombination physics but is insensitive to assumptions about dark energy or space curvature. When we allow more general forms of evolving dark energy, the BAO+SN+CMB parameter constraints are always consistent with flat \LCDM\ values at $1\sigma$. While the overal $\chi^2$ of model fits is satisfactory, the Ly$\alpha$ forest BAO measurements are in moderate $(2-2.5\sigma)$ tension with model predictions. Models with early dark energy that tracks the dominant energy component at high redshift remain consistent with expansion history constraints, and they yield a higher $H_0$ and lower matter clustering amplitude, improving agreement with some low-redshift observations.

\section{DES}
\label{DES}
The Dark Energy Survey (DES) is designed to probe the origin of the accelerating universe and to help uncover the nature of dark energy by measuring the \mbox{14-billion-year} history of cosmic expansion with high precision. DES is an optical near infrared survey of 5000 deg\sq\ of the South Galactic Cap to $r\sim24$ in $grizy$ spectrum. DES`s instrument consists primarily of a new camera, Dark Energy Camera (DECam), specifically designed to be sensitive to the highly redshifted light from distant galaxies. DECam is mounted on a classic telescope, the Blanco 4-m telescope at the Cerro Tololo Inter-American Observatory (CTIO) in La Serena, Chile. The imaging system is supported by a combination of microwave and optical data links that will provide the recorded data to the survey members. Starting in August of 2013 and continuing for five years, DES has begun to survey a large swath of the southern sky out to vast distances in order to provide new clues to this most fundamental questions \cite{DES}.

The survey data allow to measure the dark energy and dark matter densities and the dark energy equation of state through four independent methods: galaxy clusters (counts and spatial distributions at $0.1<z<1.3$), weak gravitational lensing tomography (on several redshift shells to $z\sim1$), galaxy angular clustering, and supernova distances (at $0.3<z<0.8$).

The main tool is the DECam, 74 2k $\times$ 4k 570 Mpx digital camera built at Fermilab in Batavia. It provides a 2.2\textdegree field of view image at 0.27''/pixel. It covers wavelength range 400--1100 nm with five filters $(grizy)$. The electronics will allow an entire digital image to be read out and recorded in 17 seconds, time that it takes the telescope to move to its next viewing position.

From the first two years of observation a mass map from weak gravitational lensing shear measurements over 139 deg\sq\ has been reconstructed \cite{DES_mass}. There is a good agreement between the mass map and the distribution of massive galaxy clusters identified using a red-sequence cluster finder. These measurements are consistent with simulated galaxy catalogs based on \LCDM\ \Nbody s, suggesting low systematics uncertainties in the map.

\section{Euclid}

\section{Large Synoptic Survey Telescope}

\subsection{Dark Energy Science Collaboration}

\section{Planck}

\section{W-FIRST}
The Wide-Field Infrared Survey Telescope (WFIRST) is a NASA large space mission designed to settle essential questions in dark energy, exoplanets, and infrared astrophysics. It is designed to perform wide-field imaging and slitless spectroscopic surveys of the near infrared sky. The current Astrophysics Focused Telescope Assets (AFTA) design of the mission makes use of an existing 2.4-m telescope to enhance sensitivity and imaging performance. It is the top-ranked large space mission in the New Worlds, New Horizon (NWNH) Decadal Survey of Astronomy and Astrophysics. The main instrument is a wide-field multi-filter NIR imager and spectrometer. With the 2.4-m telescope, a coronagraph instrument has been added to the payload for direct imaging of exoplanets and debris disks. If authorized for a mission start in 2017, WFIRST-AFTA would launch in the early 2020s \cite{WFIRST}.

The mission will feature strategic key science programs plus a large program of guest observations (see Appendix A of \cite{WFIRST_report}). The main focus is on the dark energy and fundamental cosmology (determine the expansion history of the Universe and the growth history of its largest structures). The next scientific goal is discovering of exoplanets -- by microlensing photometric survey of the Galactic bulge and by a direct high-contrast imaging and spectroscopic survey of the nearest stars. Data for general astrophysics science will be gathered by surveys at high Galactic latitudes and Galactic bulge. Relatively huge priority is assigned to the guest observer science program.

The payload features a 2.4-m telescope, which feeds the wide-field instrument (wide-field channel and an integral field unit spectrograph channel) and the coronagraph instrument. The wide-field channel covers a wavelength range 0.76--2.0 \um and a spectroscopy mode covering 1.35--1.89 \um. The wide-field focal plane uses 18 4k $\times$ 4k HgCdTe detector arrays. The integral field unit channel uses an image slicer and spectrograph to provide individual spectra of each slice covering the 0.6--2.0 \um spectral range. The coronagraph instrument provides high-contrast imaging and spectroscopy. Direct imaging is provided over a bandpass of 430--970 nm, while spectroscopy is provided by the spectrograph over the spectral range of 0.6--0.97 \um with a spectral resolution of $R\sim70$ \cite{WFIRST_report}.

As WFIRST will be a NIR mission it will require visible band photometry for photo-z determination. LSST will be the premier ground-based facility to provide those data. As for Euclid, the baryon acoustic oscillation spectroscopic survey will be helpful for calibrating photo-z determinations for LSST. The comparison of shear determinations between WFIRST and LSST measurements will be useful for understanding shape measurement systematics with both facilities.

\section{Cosmology with large-scale structure surveys}