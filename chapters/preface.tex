\chapter*{Introduction}
\addcontentsline{toc}{chapter}{Introduction}
\todo{Accelerated expansion, dark energy, dark matter, unknown nature

Cosmo surveys to measure the dark energy

Methods to measure the dark energy

Alternatives to cosmological constant

Cosmological simulations as a tool to probe modified gravity

Chameleon gravity / Hu-Sawicki as ``our'' model

Approximate methods -- which, why and what we expect

Note about using \cite{mastersthesis_vrastil}}

\section*{Units and conventions}
\todo{Fix table of abbreviations

Fill abbreviations}

Throughout this work we use units such that $c=\hbar=k_B=1$, where $c$ is the speed of light, $\hbar$ is reduced Planck`s constant, and $k_B$ is Boltzmann`s constant.The gravitational constant $G$ is related to the Planck mass $\mpl=1.2211\times10^{19}$ GeV via $G=1/\mpl^2$ and the reduced Planck mass $\Mpl = 2.4357\times10^{18}$ GeV via $\kappa^2\equiv8\pi G=1/\Mpl^2$ respectively. We list frequently used symbols after the Table of Contents. We adopt the metric signature $(-, +, +, +)$. The Greek indices such as $\mu$ and $\nu$ run from 0 to 3 whereas the Latin indices such as $i$ and $j$ run from 1 to 3. When referring to present values of quantities we use subscript $0$, e.g. $H_0$ to denote the present value of the Hubble parameter.