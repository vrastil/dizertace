\chapter*{Introduction}
\addcontentsline{toc}{chapter}{Introduction}
Since the twentieth century, astronomers have accumulated conclusive evidence that the content of the Universe is mostly of unknown origin and that the ordinary matter, baryonic matter, constitutes tiny fraction of the energy-density of the Universe -- only 5\%. The remaining bulk of the Universe is composed from 70\% of a dark energy causing the accelerated expansion of the Universe, whereas 25\% is in a form of a dark matter causing the formation of the structures in the Universe. Dark energy ranks as one of the most important discoveries in cosmology, with profound implications for astronomy and fundamental physics. Many experiments were, and are trying to discover the nature of these dark sectors of the Universe while other projects are planned for the future.

The Vera C. Rubin Observatory, previously referred to as the Large Synoptic Survey Telescope (LSST), is a ground-based telescope being built in northern Chile. Thanks to a large aperture, wide field survey telescope and 3200 Megapixel camera it will image faint astronomical objects across the sky. The LSST will rapidly scan the sky, charting objects that change or move: from exploding supernovae to potentially hazardous near-Earth asteroids. The LSST will produce very deep survey and its images will trace billions of remote galaxies, providing multiple probes of the mysterious dark matter and dark energy.

\todo{
Euclid, BOSS, DES, Planck, W-FIRST

Methods to measure the dark energy

Alternatives to cosmological constant

Cosmological simulations as a tool to probe modified gravity

Chameleon gravity / Hu-Sawicki as ``our'' model

Approximate methods -- which, why and what we expect

Note about using \cite{mastersthesis_vrastil}}

\section*{Units and conventions}
\todo{Fix table of abbreviations

Fill abbreviations}

Throughout this work we use units such that $c=\hbar=k_B=1$, where $c$ is the speed of light, $\hbar$ is reduced Planck`s constant, and $k_B$ is Boltzmann`s constant.The gravitational constant $G$ is related to the Planck mass $\mpl=1.2211\times10^{19}$ GeV via $G=1/\mpl^2$ and the reduced Planck mass $\Mpl = 2.4357\times10^{18}$ GeV via $\kappa^2\equiv8\pi G=1/\Mpl^2$ respectively. We list frequently used symbols after the Table of Contents. We adopt the metric signature $(-, +, +, +)$. The Greek indices such as $\mu$ and $\nu$ run from 0 to 3 whereas the Latin indices such as $i$ and $j$ run from 1 to 3. When referring to present values of quantities we use subscript $0$, e.g. $H_0$ to denote the present value of the Hubble parameter.