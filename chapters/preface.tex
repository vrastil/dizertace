\chapter{Introduction}
With~the~measurement of~an~accelerated rate of~expansion of~the~Universe, various alternatives to~standard Einstein's theory of~gravity have been suggested to~explain the~acceleration, attempting to~bypass problems connected to~the~cosmological constant.

Predictions of~structure formation and~other observables cannot be obtained analytically even for~the~standard theory of~gravity, let alone for~highly non-linear equations of~modified gravity. One must employ numerical methods such as \nbodysim s. What is possible to~study numerically in~the~fully non-linear regime for~standard gravity can be very expensive for~modified gravity, and~it is difficult, if not impossible, to~explore such a~huge space of~models and~parameters using high-resolution \nbodysim s.

Due to~the~numerical difficulties connected with~highly non-linear equations of~modified gravities, one must often look for~new ways how to~address these problems. We explore whether simplified dynamical approximations, applicable for~a~certain set of~cosmological probes, can be used to~investigate models of~modified gravity with~acceptable accuracy in~the~latter instance.

For~our purposes we chose to~explore and~test these simplified methods on~the~Hu-Sawicki \fR\ model. \fR\ gravity can be considered as the~simplest example of~an~extended theory of~gravity. \fR\ gravity can be also studied as a~new scalar field, chameleon field, with~strong non-standard gravitation coupling to~other fields. The~chameleon field has a~mass-dependent on~surrounding density and~can, therefore, escape standard Solar system tests through the so-called chameleon mechanism. We will study this mechanism on~galaxy scales, cluster scales, and~mainly on~large cosmological scales.

The~thesis is organized as follows: in~\autoref{chpt:cosmo_evol} we review basics of~the~evolution of~the~Universe; in~\autoref{chpt:de_mg} we describe the~chameleon theory and we study the~non-linear behavior of~the~chameleon field numerically in~systems exhibiting spherical symmetry, which is something not studied previously. In~\autoref{chpt:cosmo_sim} we describe different techniques when dealing with~large cosmological simulations, how we adapted them in~our own code for~\nbodysim s, and~our contributions to~a~publicly available software \code{CCL}. In~\autoref{chpt:app_schemes} we introduce different approximations we use in~our simulations. Unlike previously studied cases of~Einstein--de Sitter universes we adapted these approximations to~general \LCDM\ cosmologies. In~\autoref{chpt:app_sims} we describe original results of~our cosmological \nbodysim s using approximate schemes. In~this chapter, we also study the~chameleon gravity behavior on~cosmological scales. %In~\autoref{chpt:cosmo_surveys} we review present-day and~future experiments designed to~study our Universe. 
In~the~end, in~\autoref{chpt:outlook} we discuss possible future applications of~implemented techniques and~ways to~further improve them.
