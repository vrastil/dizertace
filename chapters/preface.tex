\chapter*{Introduction}
\addcontentsline{toc}{chapter}{Introduction}
Since the~twentieth century, astronomers have accumulated conclusive evidence that the~content of~the~Universe is mostly of~unknown origin and~that the~ordinary matter, baryonic matter, constitutes a~tiny fraction of~the~energy density of~the~Universe -- only 5\%. The~remaining bulk of~the~Universe is composed of~70\% of~dark energy causing the~accelerated expansion of~the~Universe, whereas 25\% is in~a~form of~a~dark matter causing the~formation of~the~structures in~the~Universe. Dark energy is one of~the~most important discoveries in~cosmology, with~major implications for~astronomy and~fundamental physics.

With~the~measurement of~an~accelerated rate of~expansion of~the~Universe, various alternatives to~standard Einstein's theory of~gravity have been suggested to~explain the~acceleration, attempting to~bypass problems connected to~the~cosmological constant. These theories usually add some new degrees of~freedom, either in~a~form of~new fields or by modifying existing fields.

There are many different ways how to~study the~dark energy (or modified gravity), e.g. through a~growth of~structures on~large scales and~consequent formation of~cluster and~galaxies on~smaller scales, bending of~light in~the~Universe, or baryonic oscillations in~the~early Universe.

Predictions of~structure formation and~other observables cannot be obtained analytically even for~the~standard theory of~gravity, let alone for~highly non-linear equations of~modified gravity. One must employ numerical methods such as \nbodysim s or perturbative methods. What is possible to~study numerically in~the~fully non-linear regime for~standard gravity can be very expensive for~modified gravity, and~it is difficult, if not impossible, to~explore such a~huge space of~models and~parameters using high-resolution \nbodysim s. Even in~the~mildly non-linear regime, perturbative methods can become extremely complex.

Due to~the~numerical difficulties connected with~highly non-linear equations of~modified gravities, one must often look for~new ways how to~address these problems. We explore whether simplified dynamical approximations, applicable for~a~certain set of~cosmological probes, can be used to~investigate models of~modified gravity with~acceptable accuracy in~the~latter instance.

For~our purposes, from many different models of~modified gravity, we chose to~explore and~test these simplified methods on~the~Hu-Sawicki \fR\ model. \fR\ gravity represents a~broad class of~theories. It can be considered as the~simplest example of~an~extended theory of~gravity. \fR\ gravity can be also studied from a~different point of~view than a~modification of~gravity. We can see this extra degree of~freedom as a~new scalar field, chameleon field, with~strong non-standard gravitation coupling to~other fields. The~chameleon field has a~mass-dependent on~surrounding density and~can, therefore, escape standard Solar system tests through this so-called chameleon mechanism. We will study this mechanism on~galaxy scales, cluster scales, and~mainly on~large cosmological scales.

All theoretical predictions which can be obtained analytically or numerically through simulations need to~verify by real-life experiments. Present-day experiments such as DES, BOSS, or Planck place constraints on~modified gravities and~so far no extension of~Einstein's gravity has been confirmed but many alternatives remain viable. But a~new era of~next-generation experiments is coming in~the~next decade, such as the~Vera C. Rubin Observatory, Euclid, or W-FIRST.

% The~Vera C. Rubin Observatory, previously referred to~as the~Large Synoptic Survey Telescope (LSST), is a~ground-based telescope being built in~northern Chile. Thanks to~a~large aperture, wide-field survey telescope, and~3200 Megapixel camera it will image faint astronomical objects across the~sky. The~LSST will rapidly scan the~sky, charting objects that change or move: from exploding supernovae to~potentially hazardous near-Earth asteroids. The~LSST will produce a~very deep survey and~its images will trace billions of~remote galaxies, providing multiple probes of~the~dark matter and~dark energy.

The~thesis is organized as follows: in~\autoref{chpt:cosmo_evol} we review basics of~the~evolution of~the~Universe; in~\autoref{chpt:de_mg} we describe various theories of~dark energy and~modified gravity while focusing on~the~chameleon theory. We study the~non-linear behavior of~the~chameleon field numerically in~systems exhibiting spherical symmetry in~\autoref{sec_cham}, which is something not studied previously. In~\autoref{chpt:cosmo_sim} we describe different techniques when dealing with~large cosmological simulations, how we adapted them in~our own code for~\nbodysim s, and~our contributions to~a~publicly available software \code{CCL} which compute basic cosmological observables. In~\autoref{chpt:app_schemes} we introduce different approximations, those we use in~our simulations, and~also approximations that are used by other codes or have been studied in~the~past. Unlike previously studied cases of~Einstein--de Sitter universes we adapted these approximations to~general \LCDM\ cosmologies. In~\autoref{chpt:app_sims} we describe original results of~our cosmological \nbodysim s using approximate schemes. In~this chapter, we also study the~chameleon gravity behavior on~cosmological scales. %In~\autoref{chpt:cosmo_surveys} we review present-day and~future experiments designed to~study our Universe. 
In~the~end, in~\autoref{chpt:outlook} we discuss possible future applications of~implemented techniques and~ways to~further improve them.

\section*{Units and~conventions}
Throughout this work, we use units such that $c=\hbar=k_B=1$, where $c$ is the~speed of~light, $\hbar$ is reduced Planck's constant, and~$k_B$ is Boltzmann's constant. We list frequently used symbols after the~Table of~Contents. We adopt the~metric signature $(-, +, +, +)$. The~Greek indices such as $\mu$ and~$\nu$ run from 0 to~3 whereas the~Latin indices such as $i$ and~$j$ run from 1 to~3. When referring to~present values of~quantities we use subscript $0$, e.g. $H_0$ to~denote the~present value of~the~Hubble parameter.