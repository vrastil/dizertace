\chapter{Simulations with Approximation Methods}
\todo{Heyrovsky D.:

\textit{The model choice is unusual, since the primary relevance of the chameleon field is away from highdensity regions, where it can mimic the effect of the cosmological constant. Nevertheless, it is a good idea to check if the chameleon field doesn’t interfere at these scales. In this context one should point out that for a galactic halo with a 10 kpc scale it does not make much sense to plot the rotation curve out to 300 kpc. Regarding the cosmological simulations, attempts to jointly solve a gravitational Nbody code with the Poisson-like chameleon equation do not look very promising. It might be better to use a gravitational hydrodynamic code, which solves similar types of PDEs.}
}

\section{Motivation}

\section{Results for standard gravity}

\todo{snapshots for different redshifts, PM simulations}


%%%%%%%%%%%%%%%%%%%%%%%
% Matter power spectrum
%%%%%%%%%%%%%%%%%%%%%%%
\subsection{Matter Power Spectrum}
\label{sec:pwr_spec}
The power spectrum is defined by equation~\eqref{eq:pk}. In \autoref{fig:pwr_spec_all} we show the power spectra $P(k)$ at redshift $z=0$ for the different approximation schemes. The gray areas represent variation across different realizations (different simulation runs). We see that on large (linear) scales all approximations reproduce the linear theory prediction (note that this is because the FFA and FFP results have been compensated for slower growth, as described below). On smaller scales, differences between the approximations become apparent. The ZA loses power compared to the linear power spectrum on these scales due to shell-crossing. In contrast, FFA and FPA do much better and can even partially simulate non-linear clustering.
% In \autoref{fig:pwr_spec_all}, we used the so-called \textit{effective time}, described in the following subsection, to study the differences in the \textit{shape} of power spectra between the individual approximations. In the following we will show that FFA and FPA actually have slower growth than linear theory.

\begin{figure}[hbt]
\centering
	\begin{subfigure}{0.9\textwidth}
        \includegraphicscustomlegend{simulations_approx/pwr_spec/pwr_spec}
	\end{subfigure}
	\begin{subfigure}{0.9\textwidth}
		\includegraphicscustom{simulations_approx/pwr_spec/pwr_spec}
	\end{subfigure}%
    \caption{Matter power spectrum $P(k)$ at redshift $z=0$ for different approximation schemes. Grey areas represent variations across different runs. Due to shell-crossing, ZA filters the power at higher $k$, falling below the linear power spectrum, whereas FFA and FPA retain some features of the full non-linear clustering.}
    \label{fig:pwr_spec_all}
\end{figure}

%%%%%%%%%%%%%%%%%%%%%%%%%%%%%%%%%%%%%%%%
%%%%%  -> power spectrum difference
%%%%%%%%%%%%%%%%%%%%%%%%%%%%%%%%%%%%%%%%
\subsubsection{Comparison with linear theory}
The relative differences between power spectra and the linear predictions at different redshifts are plotted in \autoref{fig:pwr_spec_diff}. By the linear prediction, in this case, is meant the initial power spectrum of the realization linearly evolved to the given epoch. The error bars once again represent variation across different realizations but individual differences are computed from the same realization of the power spectrum.
\begin{figure*}[!hbt]
\centering
	\begin{subfigure}{1.0\textwidth}
        \includegraphicscustomlegend{simulations_approx/pwr_spec/pwr_spec_diff_init_ZA}
	\end{subfigure}
	\begin{subfigure}{0.5\textwidth}
		\includegraphicscustom{simulations_approx/pwr_spec/pwr_spec_diff_emu}	
		\caption{\LCDM\ (nl)}	
	\end{subfigure}
	\begin{subfigure}{0.5\textwidth}
		\includegraphicscustom{simulations_approx/pwr_spec/pwr_spec_diff_init_ZA}
		\caption{Zel'dovich approximation}
	\end{subfigure}%
	\begin{subfigure}{0.5\textwidth}
		\includegraphicscustom{simulations_approx/pwr_spec/pwr_spec_diff_init_TZA}
		\caption{Truncated Zel'dovich approximation}
	\end{subfigure}
	\begin{subfigure}{0.5\textwidth}
		\includegraphicscustom{simulations_approx/pwr_spec/pwr_spec_diff_init_FF}
		\caption{Frozen-flow approximation}
	\end{subfigure}%
	\begin{subfigure}{0.5\textwidth}
		\includegraphicscustom{simulations_approx/pwr_spec/pwr_spec_diff_init_FP}
		\caption{Frozen-potential approximation}
	\end{subfigure}
	\caption{Relative differences between power spectra of approximations and the linear prediction at different redshifts. ZA predicts power spectra at large scales very well but fails on small scales at later times. FFA and FPA do not have this problem at small scales but the power spectrum grows more slowly across all scales. The (CosmicEmu) non-linear power spectrum is shown in the upper panel for comparison.}
	\label{fig:pwr_spec_diff}
\end{figure*}

The upper panel shows the non-linear power spectrum $P(k)$ generated by the CosmicEmu emulator \parencite{Heitmann:2015xma} for comparison purposes. The emulator predictions involve interpolations across results from a finite number of high-resolution simulations run with different cosmological parameters; the results are accurate at the level of a few percent.

As expected, the ZA predicts power spectra at large scales very well but fails on small scales at later times. In the case of FFA and FPA, the power spectrum growth is slower than linear theory predicts but, unlike ZA, this behavior is across all scales and there is significantly less suppression of power on smaller scales.

%%%%%%%%%%%%%%%%%%%%%%%
% Effective redshift
%%%%%%%%%%%%%%%%%%%%%%%
\subsection{Effective Redshift}
\label{sec:z_eff}
%%%%%%%%%%%%%%%%%%%%%%%%%%%%%%%%%%%%%%%%
%%%%%  -> suppression
%%%%%%%%%%%%%%%%%%%%%%%%%%%%%%%%%%%%%%%%
The power spectrum growth is slower than the linear prediction in the case of FFA or FPA. For FFA, this can be understood from the equation of motion \eqref{eq:FFA}. As the particles approach the minimum of the gravitational well, their velocities decrease as the gradient of the potential drops. In these approximations, as the velocity potential is constant in time, there is no change in slope as more and more particles arrive into the gravitational well, which is certainly not realistic. (If possible, the resulting artifacts should be corrected when using these approximations.)

For FPA, the reason for the suppression is very similar, although not as significant. Particles are not moving without any memory of their previous positions as in the case of FFA. They do not stop at the bottom of the well but rather oscillate around it. However, equation \eqref{eq:FPA} still drives them toward the velocities of FFA and they slow down inside gravitational wells.

Note that this effect of slower growth due to a decreasing gradient of the velocity potential is not bound to some particular scale -- it occurs both in shallow large wells as well as in deep and more concentrated wells. Consequently, it can be compensated via a correction factor. A similar situation occurs in particle-mesh codes when the number of time-steps is restricted because the assumption of constancy in velocity and force (the ``drift'' and ``kick'' terms in a standard integrator) fails to hold. In approximate particle-mesh codes, this error may be compensated by employing a ZA-motivated correction~(\cite{Ref:Feng}) in the time-stepping. The same technique could be applied here by combining the approximations in a suitable way or, as we do, by simply calibrating against linear theory, which yields the same result.

Additionally, the dynamical approximations in FFA and FPA can lead to artificial non--linear enhancement of the growth at later times on small scales within deep wells. It takes more time to get into a local gravitational well than in standard \nbody\ but once the particles are there, they may form stable cores as they simply move towards the local potential minimum. On these small scales, the approximations are not valid in any case, so this point is mostly academic.

As discussed above, the suppression of linear growth in FFA and FPA can be viewed as a modified growth function and we can introduce a simple rescaling via an effective redshift $z\eff$ so that the power spectrum on large linear scales matches the linear prediction
\eq{
	\langle P(k, z\eff) \rangle = \langle P\lin(k, z)\rangle\,,
}
where we are averaging over ``large scales''. In our simulations this mean we are fitting the linear power spectrum in the range from the minimum available wave-number $k_\text{\tiny min}=2\pi/L$ to half a decade $k=\sqrt{10}k_\text{\tiny min}$.

In the rest of the paper when we are comparing our results with a prediction of the linear or non--linear theory, we use this effective time instead of the simulation time unless stated otherwise.

In \autoref{fig:D_eff_Pk}, we compare the effective growth function with the linear growth function of \LCDM. For ZA and TZA there is almost no suppression as we are comparing large scales. However, at later times (or if we had used smaller scales) we would see an exponential suppression as described in \cite{Bharadwaj_1996}. For FFA and FPA we have an almost linear dependence of $D\eff$ on $a$, for FFA with a slope of approximately $8\%$ and for FPA with the slope being approximately $6\%$.
\begin{figure}[bt]
  \centering
    \begin{subfigure}{0.9\textwidth}
        \includegraphicscustomlegend{simulations_approx/z_eff/D_eff_Pk}
	\end{subfigure}
	\begin{subfigure}{0.9\textwidth}
        \includegraphicscustom{simulations_approx/z_eff/D_eff_Pk}
	\end{subfigure}
  \caption{Effective growth factor $D\eff$ for different approximation schemes based on ratio of power spectrum on large scales compared to the linear prediction.}
  \label{fig:D_eff_Pk}
\end{figure}

In \autoref{fig:timestep} we study the dependence of $D\eff$ on the number of time-steps. A larger number of time-steps improves $D\eff$ for FPA but, as expected, cannot eliminate the effect. In the case of FFA, $D\eff$ actually decreases with increase in number of time-steps. In this case, for smaller number of time-steps, particles do not evolve exactly along characteristic curves of the (initial) potential and the deceleration is suppressed.
\begin{figure}[bt]
  \centering
    \begin{subfigure}{0.9\textwidth}
        \includegraphicscustomlegend{simulations_approx/z_eff/timesteps}
	\end{subfigure}
	\begin{subfigure}{0.9\textwidth}
        \includegraphicscustom{simulations_approx/z_eff/timesteps}
	\end{subfigure}
  \caption{Effective growth factor $D\eff$ at $z=0$ for FFA and FPA as a function of number of time-steps.}
  \label{fig:timestep}
\end{figure}

After the linear theory compensation via the effective time, all approximations match the linear prediction on large scales (by definition), however, on smaller scales there are considerable differences. In the case of ZA (and TZA) at early times there is enhancement of power above linear theory, but at later times, the lack of the ability to follow small-scale structure (``diffusion'' at caustics) causes a suppression of power that continues to leak to larger scales. For FFA and FPA, however, the behavior is quite different. At early times particles on the smallest scales are almost in the minimum of the local gravitational potential (and consequently at the minimum of the velocity potential) and do not evolve much. This effect is weaker for FPA where particles have inertia and are not overdamped. At later times more and more particles end up in these (constant-in-time) gravitational wells and we observe (partial) non--linear gravitational clustering.


%%%%%%%%%%%%%%%%%%%%%%%
% Correlation function
%%%%%%%%%%%%%%%%%%%%%%%
\subsection{Correlation Function}
\label{sec:corr}
The two-point correlation function is defined as
\eq{
\label{eq:bao}
\xi(r)=\left\langle \delta(\mb x)\delta(\mb{x+r})\right\rangle=\int\frac{\dd^3\mb k}{(2\pi)^3}P(k)e^{i\mb{k\cdot r}}\,.
}
We computed the correlation function $\xi(r)$ for different approximation schemes at different redshifts directly from the measured matter power spectrum $P(k)$ using equation \eqref{eq:bao}. In \autoref{fig:corr_func}, we display an example of the correlation function at redshift $z=0.5$. All approximations agree reasonably with the full non-linear predictions of the emulator for the location of the BAO peak and its width.
\begin{figure}
\centering
	\begin{subfigure}{\textwidth}
        \includegraphicscustomlegend{simulations_approx/corr_func/corr_func_r2_z_z_eff}
	\end{subfigure}
	\begin{subfigure}{\textwidth}
		\centering
		\includegraphicscustom{simulations_approx/corr_func//corr_func_r2_z_z_eff}
	\end{subfigure}
	\caption{Two-point correlation function for different approximation schemes at $z=0.5$}.
	\label{fig:corr_func}
\end{figure}

Individual features of the BAO peak -- location $r_0$, amplitude $A$ and width $\sigma$ -- are obtained by fitting a Gaussian to $r^2\xi(r)$ around the BAO peak
\eq{
\xi_G(r)=A\cdot e^{-(r-r_0)^2/2\sigma^2}
}
In \autoref{fig:corr_peak}, we show the these features of the BAO peak relative to the non-linear predictions of the emulator as a function of time. All approximations can predict the location of the peak with 1\% accuracy. In predicting the shape of the peak, however, they do worse -- all approximations deviate in predictions for the peak width and amplitude. Note that which approximation does best (compared to the others) for a given quantity is a function of redshift.

%From comparison with the linear theory we see that FFA and FPA are predicting something between the linear and non-linear prediction but ZA and TZA predict values quite off the linear prediction as one would expect from these approximations.

\begin{figure}
\centering
	\begin{subfigure}{\textwidth}
        \includegraphicscustomlegend{simulations_approx/corr_func/corr_peak_loc_z_eff}
	\end{subfigure}
	\begin{subfigure}{\textwidth}
		\centering
		\includegraphicscustom{simulations_approx/corr_func//corr_peak_loc_z_eff}
		\caption{Location}
	\end{subfigure}
	\begin{subfigure}{\textwidth}
		\centering
		\includegraphicscustom{simulations_approx/corr_func//corr_peak_amp_z_eff}
		\caption{Amplitude}
	\end{subfigure}
	\begin{subfigure}{\textwidth}
		\centering
		\includegraphicscustom{simulations_approx/corr_func//corr_peak_width_z_eff}
		\caption{Width}
	\end{subfigure}
	\caption{Location, amplitude and width of the BAO peak (relative to the non-linear prediction) as a function of the redshift.}
	\label{fig:corr_peak}
\end{figure}

%%%%%%%%%%%%%%%%%%%%%%%
% Halo mass function
%%%%%%%%%%%%%%%%%%%%%%%
\subsection{Halo mass function}
In \autoref{fig:hmf_diff_z}
% \renewcommand{\zone}{3.9}
% \renewcommand{\ztwo}{1.8}
% \renewcommand{\zthree}{0.5}
% \renewcommand{\zfour}{0.0}
\begin{figure*}
	\centering
		\begin{subfigure}{1.0\textwidth}
			\includegraphicscustomlegend{simulations_approx/hmf/z0.0_b500_M1024_hmf_z_z_eff}
		\end{subfigure}
		\begin{subfigure}{0.5\textwidth}
			\includegraphicscustom{simulations_approx/hmf/z3.9_b500_M1024_hmf_z_z_eff}
			\caption{$z = $}
		\end{subfigure}%
		\begin{subfigure}{0.5\textwidth}
			\includegraphicscustom{simulations_approx/hmf/z1.8_b500_M1024_hmf_z_z_eff}
			\caption{$z = $}
		\end{subfigure}
		\begin{subfigure}{0.5\textwidth}
			\includegraphicscustom{simulations_approx/hmf/z0.5_b500_M1024_hmf_z_z_eff}
			\caption{$z = $}
		\end{subfigure}%
		\begin{subfigure}{0.5\textwidth}
			\includegraphicscustom{simulations_approx/hmf/z0.0_b500_M1024_hmf_z_z_eff}
			\caption{$z = $}
		\end{subfigure}
		\caption{Halo mass function for different approximation schemes at different redshifts.}
		\label{fig:hmf_diff_z}
	\end{figure*}



\subsection{Halo mass function}

\section{Results for modified gravity}

\subsection{Effect of simulation resolution}

\subsection{Matter power spectrum}

\subsection{Correlation function}

\subsection{Halo mass function}