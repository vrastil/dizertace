\chapter{Outlook}
\label{chpt:outlook}
In the thesis we explored different ways to study modified gravities. The work was focused on quasi-linear regimes as the full non--linear behavior of modified gravities is hard to study due to its numerical difficulties. We showed what behavior one can expect on scales ranging from stars to that of superclusters. The implemented methods were tested on the example of the chameleon gravity and showed interesting results. These prepared techniques can now be used to explore other modified gravities.

In \autoref{chpt:de_mg} we studied the properties of the chameleon gravity in spherical systems -- stars and in NFW halos both on scales of galaxies and clusters of galaxies. Although the results showed that the Hu-Sawicki \fR\ theory is screened in this system, other modifications of gravity do not have to be and it is worth checking. Our code and implemented techniques can be used with slight modifications for studies of broader class of theories.

Implementing other models of \fR\ gravity exhibiting chameleon mechanism should be fairly straightforward and therefore we would like to explore this type of modification in a similar way as we did for the Hu-Sawicki model. We would like to study the the generic \fR\ models with negative and positive powers of curvature of \textcite{2003PhRvD..68l3512N} which can in principle unified the inflation and cosmic acceleration. The model is given by
\eq{
    f(R)=R-\frac{a}{(R-\Lambda_1)^n}+b(R-\Lambda_2)^m\,.
}
Another class of models we would like to explore was proposed by \textcite{2007JETPL..86..157S}. This class of modifications is given by
\eq{
    f(R)=R+\lambda R_0\left[\left(1+\frac{R^2}{R_0^2}\right)^{-n}-1\right]\,.
}
All of these models can be studied in dense objects surrounded by a vacuum (or decreasing density) such as stars and halos as we did in this work. Another approach, which can be tested in laboratory conditions on Earth, is quite the opposite. We can study the behavior of the chameleon in a vacuum chamber surrounded by a dense environment. This could be implemented easily in our code and is therefore worth to study. However, we showed that the effective screening potential is very low on scales of galaxies and therefore we do not expect this will lead to measurable results.

The main field for studying modified gravities remain on cosmological scales where it mimics the cosmological constant. In the work we dealt with approximations which can greatly help to study quasi-linear regimes. Although the non--linear results for used approximation schemes do not look very promising compared to simple particle-mesh codes such as \code{COLA} they proved that linear approximation can be used successfully and we can still obtain results in mildly non--linear regimes. This can be used for studies of modified gravity where solving the full non--linear equations is computationally very demanding and probing the huge space of different modifications almost impossible. We showed that linear approximation of chameleon equations can still lead to observable non--linear results.

This approach we want to pursue further, i.e. to use particle-mesh codes such as \code{COLA}. \code{HACC} or \code{GADGET} without demanding short-range interactions and implement modified equations for theories such as chameleon. To speed up the execution time we would use the linear predictions for these theories which we saw can still lead to non--linear results.

The next step we can make in studying the non--linear regime with modified gravities is to use \nbody\ code with short-range forces, use the linear predictions for chameleon gravity to avoid the lengthy computation of the chameleon field and therefore obtain a good approximation of the background value of the field. To solve the equations on small scales quickly we would use the introduced technique for spherical systems. As the halo finders are a standard part of the simulations, obtaining some approximate density profile of forming halos would not be too expensive. Consequent solving of the one-dimensional chameleon non--linear equations does not pose any substantial demands on computing time.