\chapter{Approximation Schemes}
\label{chpt:app_schemes}
Parts of this chapter has been published in \textcite{2020MNRAS.493.2085V}.

\section{Motivation}
Various approximations in different scientific fields have always been studied in detail. Many non-linear equations cannot be solved analytically and one can hope to achieve at least some results using linearization or other perturbation theories of higher order. Cosmological simulations are no exception. From the beginning of first attempts to simulate clustering of matter on large scales in late 60`s the approximate methods were developed to help understand dynamics of the Universe. The greatest pioneering efforts to improve and validate these approximate methods have been undertook in the 90`s.

These analytic or semi-analytic methods can be used to understand otherwise very complicated problem of structure formation. Instead of running a simulation ``blindly'' and trying to analyze them phenomenologically one can compare these simulations with much better understood approximate methods.

Besides usefulness of approximations in getting an insight into gravitational evolution they are very helpful in getting large number of simulations quickly. In order to study BAO one needs high number of simulations with very large volume and high number of particles which requires demanding resources. The BAO scale is of particular interest to us as it lies between the linear regime which can be studied analytically and highly non-linear regime of halo formation which requires precise \nbodysim s. The semi-analytical methods are expected to work very well in this regime.

In this chapter we describe several of approximations which have been studied in the past -- Zel`dovich approximation and its \textit{truncated} extension, frozen-flow approximation and frozen-potential approximation. The Zel'dovich approximation has been studied extensively in the past and here we use it mainly as a reference for comparison in the context of the other approximations.

\section{Recapitulation of linear theory}
\todo{Rewrite equations using $a$, use previous equations}
We begin with the linearized equations for gravitational clustering in an expanding Universe. We will be working in comoving coordinates -- the comoving position $\mb x$ is defined in terms of the proper position coordinate $\mb r = a\mb x$, where $a=a(t)$ is the scale factor. The comoving velocity $\mb v$ is then defined as a derivative with respect to cosmic time $t$, $\mb v = \dot{\mb x}$, where the overdot denotes a time derivative. The overdensity $\delta$ is defined in terms of the local density, $\rho$, and the background density, $\bar\rho$,  as $\delta=\rho/\bar\rho-1$. Particles move in the Newtonian gravitational potential, $\phi_G$. The equations for linear perturbations then read
\eq{
	\label{eq:lin_per_a}
	\dot\delta + \nabla\cdot\mb v &= 0, \\
	\label{eq:lin_per_b}
    \dot{\mb v} + 2\frac{\dot a}{a} \mb v &= -\frac{1}{a^2}\nabla\phi_G, \\
\begin{split}
	\label{eq:lin_per_c}
	\Delta\phi_G &= 4\pi G\bar\rho a^2 \delta \\
    			&= \frac32 H_0^2\Omega_{m, 0}\frac\delta a \equiv \mu^{-1}\frac\delta a\,,
\end{split}
}
where $H_0$ is the (present) Hubble constant, $\Omega_{m, 0}$ is the matter density (baryon and dark matter) and we defined the constant $\mu\equiv\left(\frac32 H_0^2\Omega_{m, 0}\right)^{-1}$. In this linear regime, the time and space dependence of the overdensity evolution are separable and we can write $\delta(a, \mb x)=D(a)\delta_0(\mb x)$ where the growth factor $D$ represents the growing solution (we neglect the decaying mode) and is normalized to unity at the present time.

It is often useful to rewrite these equations with a different time variable, namely the scale factor $a$. This is convenient both from the numerical and theoretical points of view because the quantities are ``more constant''. With the new time-variable $a$ and comoving velocity $\mb u = \dd \mb x/\dd a$, equation \eqref{eq:lin_per_a} becomes
\eq{
\label{eq:continuity_a}
	\dddd D a \delta_0 + \nabla\cdot\mb u = 0\,,
}
where $\dd D/\dd a = 1$ in the Einstein--de Sitter universe (hereafter EdS) and so the divergence of the velocity field remains constant in both time and space. Equation \eqref{eq:lin_per_b} then becomes
\seq{
	\label{eq:motion_EdS}
	\dddd{\mb u}{a} &= -\frac{3}{2a}\left[\mb u + \mu\nabla\phi_G \right] \\
\intertext{in EdS and, more generally,}
	\label{eq:motion_LCDM}
	\dddd{\mb u}{a} &= -\frac{3}{2a}\left[\mb u\left(1+f_\Lambda\right) + \mu\nabla\phi_G\left(1-f_\Lambda\right)\right]
}
in the \LCDM\ universe. The modification factor
\eq{
	f_\Lambda = \frac{\Omega_{\Lambda,0}a^3}{\Omega_{m,0} + \Omega_{\Lambda,0}a^3} = \Omega_\Lambda(a)
}
rises from $0$ to $\Omega_{\Lambda,0}$ and becomes statistically significant $(>5\%)$ around redshift $z\approx2.5$. Equation \eqref{eq:lin_per_c} expressed with the growth factor reads
\eq{
\label{eq:poisson_a}
	\Delta\phi_G(\mb x, a) = \mu^{-1}\frac D a \delta_0(\mb x)\,.
}
%%%%%%%%%%%%%%%%%%%%%%%%%%%%%%%%%%%%%%%%
% Zel`dovich Approximation
%%%%%%%%%%%%%%%%%%%%%%%%%%%%%%%%%%%%%%%%
\section{Zel`dovich approximation}
The Zel'dovich approximation \parencite[hereafter ZA;][]{1970A&A.....5...84Z} bears its name after a pioneer in the study of large-scale structure, a Soviet physicist Yakov Zel`dovich. The ZA provides an intuitive way to understand the emergence of filamentary structures (cosmic web) and can realize model of non-linear structure formation even though it is based only on linear approximations \parencite{2014MNRAS.439.3630W}. The Zel’dovich approximation predicts the rich structure of voids, clusters, sheets and filaments observed in the Universe.

The ZA is based on an ansatz that particles move in straight lines in the Lagrangian frame
\eq{
\label{eq:ZA}
	\mb x(\mb q, a) = \mb q + D(a)\mb S(\mb q)\,,
}
where $\mb q$ are the initial positions (Eulerian coordinates) of a particle and $\mb S$ is some displacement field. Inserting equation \eqref{eq:ZA} into the continuity equation \eqref{eq:continuity_a} yields
\eq{
	-\nabla\cdot S = \delta_0\,.
}
Combining with the Poisson equation \eqref{eq:poisson_a}, we can see that \eqref{eq:ZA} represents a potential flow, $\mb S = -\nabla\phi_V$, where the velocity potential $\phi_V$ obeys the Poisson equation
\eq{
	\label{eq:poisson_vel}
	\Delta\phi_V = \delta_0
}
and has a simple relation to the gravitational potential
\eq{
	\label{eq:vel_new}
	\phi_V=\mu\frac a D \phi_G\,.
}
This results into ZA being
\eq{
	\mb x(\mb q, a) = \mb q - D(a)\mb\nabla \phi_V(\mb q)\,.
}
Note that the velocity potential is not exactly a potential of our velocity field $\mb u$ but
\eq{
	\label{eq:ZA_u}
	\mb u\ZA(\mb x) = -\dddd D a \nabla\phi_V(\mb q)\,.
}
The ZA differs from other approximations (among other things) in how this (constant) velocity potential, $\phi_V$, enters the equations of motion. In order to avoid having different definitions of the \textit{real} velocity potential for the velocity fields in each approximation, we take the equation \eqref{eq:poisson_vel} as defining the velocity potential $\phi_V$.

The deformation tensor is defined as
\eq{
	d_{ij}=\partpart{x_i}{q_j}=\delta_{ij}+D\partpart{\nabla_i\phi_V}{q_j}\,.
}
The eigenvectors of the deformation tensor determine the principle directions of the collapse and the corresponding eigenvalues determine the time when the compression will be infinite. The density is given through eigenvalues $\lambda_i$ as
\eq{
	\rho=\frac{\bar\rho}{(1-D\lambda_1)(1-D\lambda_2)(1-D\lambda_3)}\,.
}
The time when the density in ZA becomes infinite corresponds to particles crossing the paths of other particles. Once this shell-crossing has occurred, the approximation has formally broken down, since there are no forces present to slow down the particles and capture them within halos.

\todo{Comparison of ZA at $z=0,1,2,3$, \autoref{fig:slice_dens_ZA}}
\begin{figure*}
	\begin{adjustwidth}{-3cm}{-1cm}
	\centering
		\begin{subfigure}{0.5\linewidth}
			\includegraphics[width=1.0\linewidth]{{simulations_approx/dens/za_dens_512m_1p_1024M_200b_z0.00}.png}
			\caption{$z=0$}
		\end{subfigure}%
		\begin{subfigure}{0.5\linewidth}
			\includegraphics[width=1.0\linewidth]{{simulations_approx/dens/za_dens_512m_1p_1024M_200b_z0.00}.png}
			\caption{$z=1$}
		\end{subfigure}
		\begin{subfigure}{0.5\linewidth}
			\includegraphics[width=1.0\linewidth]{{simulations_approx/dens/za_dens_512m_1p_1024M_200b_z0.00}.png}
			\caption{$z=2$}
		\end{subfigure}%
		\begin{subfigure}{0.5\linewidth}
			\includegraphics[width=1.0\linewidth]{{simulations_approx/dens/za_dens_512m_1p_1024M_200b_z0.00}.png}
			\caption{$z=3$}
		\end{subfigure}
		\caption{Projected density field at different redshifts for the Zel`dovich approximation. Each slice has a box-length of $200~\Mpch$ and is $1~\Mpch$ thick.}
		\label{fig:slice_dens_ZA}
	\end{adjustwidth}
\end{figure*}
%%%%%%%%%%%%%%%%%%%%%%%%%%%%%%%%%%%%%%%%
% Truncated Zel`dovich Approximation
%%%%%%%%%%%%%%%%%%%%%%%%%%%%%%%%%%%%%%%%
\section{Truncated Zel`dovich approximation}
The shell-crossing and diffusion of particles on small scales in ZA happens the sooner the more power there is on small scales. \textcite{doi:10.1093/mnras/260.4.765} suggested an improvement of the ZA by removing power on these small nonlinear scales, i.e. to set the initial power spectrum to zero for wave-numbers $k$ greater than a nonlinear scale $k_{nl}$ defined as
\eq{
\label{eq:k_nl}
    \frac{a^2(t)}{2\pi^2}\int_0^{k_{nl}}P(k)\dd k=1\,,
}
where the power spectrum $P(k)$ was defined in \eqref{eq:pk} as
\eq{
  \label{eq:pk}
  P(k)(2\pi)^3\delta_{\rm D}(k-k')\equiv \left\langle \hat\delta(k)\hat\delta^*(k')\right\rangle\,.
}

\textcite{doi:10.1093/mnras/269.3.626} further improved this truncation by applying a Gaussian window instead of an abrupt cutoff
\eq{
W(k)=e^{-k^2/2k^2_{G}}\,,
}
where the smoothing scale $k_{G}$ is 1 to 1.5 times $k_{nl}$. This filtering leads to the so called truncated Zel`dovich approximation (TZA). This removes most of the strongly non-linear behavior and allows the Zel’dovich pancakes to be seen.

\todo{Comparison of TZA at $z=0,1,2,3$, \autoref{fig:slice_dens_TZA}}
\begin{figure*}
	\begin{adjustwidth}{-3cm}{-1cm}
	\centering
		\begin{subfigure}{0.5\linewidth}
			\includegraphics[width=1.0\linewidth]{{simulations_approx/dens/tza_dens_512m_1p_1024M_200b_z0.00}.png}
			\caption{$z=0$}
		\end{subfigure}%
		\begin{subfigure}{0.5\linewidth}
			\includegraphics[width=1.0\linewidth]{{simulations_approx/dens/tza_dens_512m_1p_1024M_200b_z0.00}.png}
			\caption{$z=1$}
		\end{subfigure}
		\begin{subfigure}{0.5\linewidth}
			\includegraphics[width=1.0\linewidth]{{simulations_approx/dens/tza_dens_512m_1p_1024M_200b_z0.00}.png}
			\caption{$z=2$}
		\end{subfigure}%
		\begin{subfigure}{0.5\linewidth}
			\includegraphics[width=1.0\linewidth]{{simulations_approx/dens/tza_dens_512m_1p_1024M_200b_z0.00}.png}
			\caption{$z=3$}
		\end{subfigure}
		\caption{Projected density field at different redshifts for the truncated Zel`dovich approximation. Each slice has a box-length of $200~\Mpch$ and is $1~\Mpch$ thick.}
		\label{fig:slice_dens_TZA}
	\end{adjustwidth}
\end{figure*}
%%%%%%%%%%%%%%%%%%%%%%%%%%%%%%%%%%%%%%%%
% Frozen-flow Approximation
%%%%%%%%%%%%%%%%%%%%%%%%%%%%%%%%%%%%%%%%
\section{Frozen flow approximation}
The frozen-flow, or frozen-field, approximation (FFA) was originally proposed by \textcite{Matarrese:1992be} as the exact solution of equation \eqref{eq:motion_EdS} in EdS
\eq{
  \label{eq:FFA_orig}
  \tilde{\mb u}\FFA(\mb x) = \mb u_0(\mb x) = -\mu\nabla\phi_G(\mb x) = -\nabla\phi_V(\mb x)\,.
}
The velocity field $\tilde{\mb u}$ is frozen at each point to its initial value, i.e.
\eq{
	\partpart{\tilde{\mb u}}{a} = 0\,.
}
These equations are very similar to ZA but now the particles update their velocities to the local value of the velocity field (not the initial value), without any memory of their previous motion. This can be viewed as a movement of particles under some force in a medium with a very large viscosity. In our case of cosmological simulations, gravity represents the attractive force while Hubble friction represents the informal equivalent of a ``damping'' or ``viscous'' force.

Originally, \textcite{Matarrese:1992be} stated three main reasons of why should the FFA work:
\begin{itemize}
\item FFA is, by construction, consistent with linear theory and follows correctly the evolution at early times. Keeping the linear approximation for the velocity potential is justified by the fact that this quantity is more sensitive to large wavelength modes than the density, and is therefore less affected by strongly non-linear evolution.
\item Stream-lines are frozen to their initial shape, so multistream regions cannot form and FFA avoids formation of caustics at finite time and can therefore work well after shell-crossing would occur in ZA. A particle moving according to FFA has zero velocity at minima (or maxima) of the gravitational potential. It will slow down its motion when approaching such a position -- particles in FFA would need infinite time to reach such places. Particles move along curved paths and once they come close to pancake configurations they curve their trajectories, moving almost parallel to them, trying to reach the positions of filaments.
\item This type of dynamics implies an artificial thickening of particles around pancakes, filaments and knots, which mimics the real gravitational clustering around these types of structures.
\end{itemize}

The definition \eqref{eq:FFA_orig} is, however, no longer valid in the general \LCDM\ cosmology where even in the linear regime both gravitational potential and velocity field undergo evolution. We generalize FFA for the \LCDM\ cosmology by adding an extra time dependence in equation \eqref{eq:FFA_orig}
\eq{
	\label{eq:FFA}
	\mb u\FFA(\mb x, a) = -\dddd D a(a)\nabla\phi_V(\mb x)\,.
}
This velocity field solves \eqref{eq:motion_LCDM} exactly, due to the definition of the growth factor and its relation to the velocity field in the linear regime (see equation \eqref{eq:continuity_a}).

Although equation \eqref{eq:FFA} now does not represent a \textit{frozen} flow, particles still move along the same characteristic curves as in EdS, just with different velocities. The particle trajectories are described by the integral equation
\eq{
	\label{eq:FFA_int}
	\mb x(a) = \mb q + \int_0^a\dd \tilde a\mb u\FFA(\mb x(\tilde a), \tilde a)\,.
}

\todo{Comparison of FFA at $z=0,1,2,3$, \autoref{fig:slice_dens_FFA}}
\begin{figure*}
	\begin{adjustwidth}{-3cm}{-1cm}
	\centering
		\begin{subfigure}{0.5\linewidth}
			\includegraphics[width=1.0\linewidth]{{simulations_approx/dens/ff_dens_512m_1p_1024M_200b_z0.00}.png}
			\caption{$z=0$}
		\end{subfigure}%
		\begin{subfigure}{0.5\linewidth}
			\includegraphics[width=1.0\linewidth]{{simulations_approx/dens/ff_dens_512m_1p_1024M_200b_z0.00}.png}
			\caption{$z=1$}
		\end{subfigure}
		\begin{subfigure}{0.5\linewidth}
			\includegraphics[width=1.0\linewidth]{{simulations_approx/dens/ff_dens_512m_1p_1024M_200b_z0.00}.png}
			\caption{$z=2$}
		\end{subfigure}%
		\begin{subfigure}{0.5\linewidth}
			\includegraphics[width=1.0\linewidth]{{simulations_approx/dens/ff_dens_512m_1p_1024M_200b_z0.00}.png}
			\caption{$z=3$}
		\end{subfigure}
		\caption{Projected density field at different redshifts for the frozen flow approximation. Each slice has a box-length of $200~\Mpch$ and is $1~\Mpch$ thick.}
		\label{fig:slice_dens_FFA}
	\end{adjustwidth}
\end{figure*}
%%%%%%%%%%%%%%%%%%%%%%%%%%%%%%%%%%%%%%%%
% Frozen-potential Approximation
%%%%%%%%%%%%%%%%%%%%%%%%%%%%%%%%%%%%%%%%
\section{Frozen potential approximation}
The frozen-potential approximation (FPA) was introduced by \textcite{1994MNRAS.266..227B}. They exploit the fact that the gravitational potential changes much more slowly than the density contrast, and hence may be viewed as essentially frozen. Moreover, same as the velocity potential in FFA is more sensitive to large wavelength modes, this is doubly true for gravitational potential. They solved equation \eqref{eq:motion_EdS} at each time-step with the initial (constant) gravitational potential. In \LCDM
\eq{
	\label{eq:FPA}
	\dddd{\mb u\FPA}{a} &= -\frac{3}{2a}\left[\mb u\FPA(\mb x,a)\left(1+\Omega_\Lambda(a)\right) + \frac D a \nabla\phi_V(\mb x)\left(1-\Omega_\Lambda(a)\right)\right]
}
and the particle trajectories are given as in the case of FFA\,
\eq{
	\mb x(a) = \mb q + \int_0^a\dd \tilde a\mb u\FPA(\mb x(\tilde a), \tilde a)~.
}
Particles now follow the linear \textit{gravitational} potential (which evolves slightly in \LCDM) instead of the initial \textit{velocity} potential as in the case of FFA. Equation \eqref{eq:FPA} drives the particle velocities (approximately) to the velocities of FFA but unlike in the FFA, particles now keep their inertia. Same as in the case of FFA, the particles tend to move along the pancakes towards regions of lower potential to end up in a few clumps. The acceleration used in FPA is largest in regions where the instantaneous velocity vector points along the gradient of the potential, as happens for particles after they cross the pancake. In FFA the inertia of particles is ignored, whereas in the Zeldovich approximation inertia is assumed to dominate over change in the force field. FPA takes into consideration both factors but assumes a constant potential.

\todo{Comparison of FPA at $z=0,1,2,3$, \autoref{fig:slice_dens_FPA}}
\begin{figure*}
	\begin{adjustwidth}{-3cm}{-1cm}
	\centering
		\begin{subfigure}{0.5\linewidth}
			\includegraphics[width=1.0\linewidth]{{simulations_approx/dens/fp_dens_512m_1p_1024M_200b_z0.00}.png}
			\caption{$z=0$}
		\end{subfigure}%
		\begin{subfigure}{0.5\linewidth}
			\includegraphics[width=1.0\linewidth]{{simulations_approx/dens/fp_dens_512m_1p_1024M_200b_z0.00}.png}
			\caption{$z=1$}
		\end{subfigure}
		\begin{subfigure}{0.5\linewidth}
			\includegraphics[width=1.0\linewidth]{{simulations_approx/dens/fp_dens_512m_1p_1024M_200b_z0.00}.png}
			\caption{$z=2$}
		\end{subfigure}%
		\begin{subfigure}{0.5\linewidth}
			\includegraphics[width=1.0\linewidth]{{simulations_approx/dens/fp_dens_512m_1p_1024M_200b_z0.00}.png}
			\caption{$z=3$}
		\end{subfigure}
		\caption{Projected density field at different redshifts for the frozen potential approximation. Each slice has a box-length of $200~\Mpch$ and is $1~\Mpch$ thick.}
		\label{fig:slice_dens_FPA}
	\end{adjustwidth}
\end{figure*}

\todo{comparison of all approximation at $z=0$, \autoref{fig:slice_dens_all}}
\begin{figure*}
	\begin{adjustwidth}{-3cm}{-1cm}
	\centering
		\begin{subfigure}{0.5\linewidth}
			\includegraphics[width=1.0\linewidth]{{simulations_approx/dens/za_dens_512m_1p_1024M_200b_z0.00}.png}
			\caption{Zel`dovich approximation}
		\end{subfigure}%
		\begin{subfigure}{0.5\linewidth}
			\includegraphics[width=1.0\linewidth]{{simulations_approx/dens/tza_dens_512m_1p_1024M_200b_z0.00}.png}
			\caption{Truncated Zel`dovich approximation}
		\end{subfigure}
		\begin{subfigure}{0.5\linewidth}
			\includegraphics[width=1.0\linewidth]{{simulations_approx/dens/ff_dens_512m_1p_1024M_200b_z0.00}.png}
			\caption{Frozen-flow approximation}
		\end{subfigure}%
		\begin{subfigure}{0.5\linewidth}
			\includegraphics[width=1.0\linewidth]{{simulations_approx/dens/fp_dens_512m_1p_1024M_200b_z0.00}.png}
			\caption{Frozen-potential approximation}
		\end{subfigure}
		\caption{Projected density field at redshift $z=0$ for the different approximations, all run with the same initial conditions. Each slice has a box-length of $200~\Mpch$ and is $1~\Mpch$ thick.}
		\label{fig:slice_dens_all}
	\end{adjustwidth}
\end{figure*}

%%%%%%%%%%%%%%%%%%%%%%%%%%%%%%%%%%%%%%%%
% Other approximations
%%%%%%%%%%%%%%%%%%%%%%%%%%%%%%%%%%%%%%%%
\section{Other approximations}
Approximations described previously are studied numerically in detail in the next chapter. Here we present few of the other approximations studied in the past or used today.

\subsection{Adhesion approximation}
The adhesion approximation was introduced in \textcite{1989MNRAS.236..385G}. To study the evolution of density inhomogeneities they used model of non-linear diffusion (Burger`s equation), that gives an approximate description of the growth of structures at the advanced non-linear stage of gravitational instability.

To overcome problems of ZA with shell-crossing they propose a solution of ``sticking articles.'' Particles move according to ZA until they ran into one another. Then they move together, with the velocity conserving momentum. This model can be described mathematically by inserting the viscous term into equations of motion, simulating attractive forces of gravity
\eq{
	\dddd{\mb u\AAP}{a}=\nu\partpart{^2\mb u}{\mb x^2}\,.
}
The Burger`s equation has an analytical solution which can be used to study formation of structures. For more information regarding the adhesion approximation see  also \textcite{1990MNRAS.247..260W,1994ApJ...428...28M}.

\subsection{Stable clustering}
It was proposed by \textcite{1974ApJ...189L..51P} that clustering in the very non-linear regime might be understood by assuming that regions of high density contrast undergo virialization and subsequently maintain a fixed proper density, hence stable clustering. The correlation function for a population of such systems would then simply evolve according to
\eq{
	\xi(r,a)\propto1/\bar\rho\propto a^3\,.
}
\textcite{1991ApJ...374L...1H} developed a method for interpolating between linear theory on large scales and the non-linear predictions of the stable clustering hypothesis on small scales. They showed that the non-linear volume-averaged two-point correlation function could be parameterized by a simple function of the linear correlation function
\eq{
	\bar\xi_{NL}=f(\bar\xi_{L})\,,
}
where the functional form of $f$ can be derived from the spherical top-hat model without any shell-crossing. For the linear regime $\bar\xi_{L}\ll1$, $f(y)=y$, and for non-linear $\bar\xi_{L}\gg1$, $f(y)=y^{3/2}$. For more information regarding the stable clustering see  also \textcite{1996MNRAS.280L..19P,2003MNRAS.341.1311S}.
\subsection{Lagrangian perturbation theories of higher orders}
Success of ZA, the first order Lagrangian perturbation theory (LPT), has motivated studies of higher order corrections \parencite[see e.g.][]{10.1093/mnras/264.2.375,2002PhR...367....1B,2010MNRAS.403.1859J,2014ApJ...788...63S}. In the Lagrangian description, the spatial coordinates are transformed through the displacement vector $\Psi$ as
\eq{
	\mb x = \mb q + \Psi(a, \mb x)\,.
}
In LPT, this displacement vector field is expanded in a perturbation series in the linear growth function $D$ in Fourier space. Density perturbations $\delta$ are described as a function of the displacement vector through conservation of mass. This Lagrangian picture is intrinsically non-linear in the density field, and a small perturbation in Lagrangian fluid element paths carries a considerable amount of non-linear information about the corresponding Eulerian density and velocity fields. Different variants of LPT -- third order LPT (\cite{10.1093/mnras/264.2.375}), Truncated LPT (\cite{10.1093/mnras/260.4.765}), Augumented LPT (\cite{10.1093/mnrasl/slt101}), MUSCLE (\cite{10.1093/mnrasl/slv141}) -- have been tested against particle-mesh code COLA (\cite{2013JCAP...06..036T}) in \cite{2017JCAP...07..050M}, see \autoref{fig:app_compare}.

\begin{figure}[ht]
    \centering
    \includegraphics[width=0.9\textwidth]{cosmo_evol/app_compare.png}
    \caption{Power spectrum at $z = 0,\ 0.5$ and $1$ (top, middle and bottom panels, respectively) in real space and ratio with the \nbody’s one for the matter field (left panels) and for the halo catalogues (right panels). The vertical dashed line locates the $k = 0.5\hMpc$ where the one-halo term becomes significant. The vertical shaded area locates the region of the BAO peak, while the horizontal one locates the 1\% accuracy region.  \textit{Note:} Reprinted from \textcite{2017JCAP...07..050M}.}
    \label{fig:app_compare}
\end{figure}

\section{Application to modified gravity}
\todo{move to the next chapter?}