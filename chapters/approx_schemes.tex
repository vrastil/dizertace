\chapter{Approximation Schemes}
\label{chpt:app_schemes}
In~this chapter we will introduce different approximation methods that we plan to~use in~our numerical simulations. We will go over equations governing their behavior and~we also briefly mention some other approximations that were used in~the~past.

Parts of~this chapter have been published in~\textcite{2020MNRAS.493.2085V}.

\section{Motivation}
Various approximations in~different scientific fields have always been studied in~detail. Many non-linear equations cannot be solved analytically and~one can hope to~achieve at~least some results using linearization or other perturbation theories of~higher-order. Cosmological simulations are no exception. From the~beginning of~first attempts to~simulate clustering of~matter on~large scales in~the~late '60s the~approximate methods were developed to~help understand the~dynamics of~the~Universe. The~greatest pioneering efforts to~improve and~validate these approximate methods have been undertaken in~the~'90s.

These analytic or semi-analytic methods can be used to~understand the~otherwise very complicated problem of~structure formation. Instead of~running a~simulation ``blindly'' and~trying to~analyze them phenomenologically one can compare these simulations with~much better understood approximate methods.

Besides the~usefulness of~approximations in~getting an~insight into gravitational evolution, they are very helpful in~getting a~large number of~simulations quickly. In~order to~study BAO, one needs a~high number of~simulations with~very large volume and~a~high number of~particles which requires demanding resources. The~BAO scale is of~particular interest to~us as it lies between the~linear regime which can be studied analytically and~the~highly non-linear regime of~halo formation which requires precise \nbodysim s. The~semi-analytical methods are expected to~work very well in~this regime.

In~this chapter, we describe several approximations that have been studied in~the~past -- Zel'dovich approximation and~its \textit{truncated} extension, frozen-flow approximation, and~frozen-potential approximation. The~Zel'dovich approximation has been studied extensively in~the~past and~here we use it mainly as a~reference for~comparison in~the~context of~the~other approximations.

\section{Recapitulation of~the~linear theory}
Here we remind some results of~the~linear theory previously presented in~\autoref{chpt:cosmo_evol}. We rewrite the~equations using variables more suited for~cosmological simulations. We will be working in~comoving coordinates -- the~comoving position $\mb x$ is defined in~terms of~the~proper position coordinate $\mb r = a\mb x$. The~comoving velocity $\mb v$ is then defined as a~derivative with~respect to~cosmic time $t$, $\mb v = \dot{\mb x}$, where the~overdot denotes a~time derivative. Particles move in~the~Newtonian gravitational potential, $\Phi$. The~equations for~linear perturbations then read
\eq{
	\label{eq:lin_per_a}
	\dot\delta + \nabla\cdot\mb v &= 0, \\
	\label{eq:lin_per_b}
    \dot{\mb v} + 2\frac{\dot a}{a} \mb v &= -\frac{1}{a^2}\nabla\Phi, \\
\begin{split}
	\label{eq:lin_per_c}
	\Delta\Phi &= 4\pi G\bar\rho a^2 \delta \\
    			&= \frac32 H_0^2\Omega_{m, 0}\frac\delta a \equiv \mu^{-1}\frac\delta a\,,
\end{split}
}
where we defined the~constant $\mu\equiv\left(\frac32 H_0^2\Omega_{m, 0}\right)^{-1}$. In~this linear regime, the~time and~space dependence of~the~overdensity evolution are separable and~we can write $\delta(a, \mb x)=D(a)\delta_0(\mb x)$ where the~growth factor $D$ represents the~growing solution (we neglect the~decaying mode) and~is normalized to~unity at~the~present time.

It is often useful to~rewrite these equations with~a~different time variable, namely the~scale factor $a$. This is convenient both from the~numerical and~theoretical points of~view because the~quantities are ``more constant''. With~the~new time-variable $a$ and~comoving velocity $\mb u = \dd \mb x/\dd a$, equation \eqref{eq:lin_per_a} becomes
\eq{
\label{eq:continuity_a}
	\dddd D a \delta_0 + \nabla\cdot\mb u = 0\,,
}
where $\dd D/\dd a = 1$ in~the~Einstein--de Sitter universe (hereafter EdS) and~so the~divergence of~the~velocity field remains constant in~both time and~space. Equation \eqref{eq:lin_per_b} then becomes
\seq{
	\label{eq:motion_EdS}
	\dddd{\mb u}{a} &= -\frac{3}{2a}\left[\mb u + \mu\nabla\Phi \right] \\
\intertext{in~EdS and, more generally,}
	\label{eq:motion_LCDM}
	\dddd{\mb u}{a} &= -\frac{3}{2a}\left[\mb u\left(1+\Omega_\Lambda\right) + \mu\nabla\Phi\left(1-\Omega_\Lambda\right)\right]
}
in~the~\LCDM\ universe. The~omega factor of~the~cosmological constant
\eq{
	\Omega_\Lambda(a) = \frac{\Omega_{\Lambda,0}a^3}{\Omega_{m,0} + \Omega_{\Lambda,0}a^3}
}
rises from $0$ to~$\Omega_{\Lambda,0}$ and~becomes significant $(>5\%)$ around redshift $z\approx2.5$. The~transformation \eqref{eins_trans} from the~Jordan frame to~the~Einstein frame introduces non-standard coupling of~the~chameleon field to~standard matter resulting in~the~fifth force \eqref{cham_force}. Using our new variables, this equation reads
\eq{
	\label{eq:cham_force}
	\dddd{\mb u_{\chi}}{a} = -\frac{3\mu}{2a}\frac{\beta}{\Mpl}\mb{\nabla}\chi \,.
}
Equation \eqref{eq:lin_per_c} expressed with~the~growth factor reads
\eq{
\label{eq:poisson_a}
	\Delta\Phi(\mb x, a) = \mu^{-1}\frac D a \delta_0(\mb x)\,.
}
%%%%%%%%%%%%%%%%%%%%%%%%%%%%%%%%%%%%%%%%
% Zel'dovich Approximation
%%%%%%%%%%%%%%%%%%%%%%%%%%%%%%%%%%%%%%%%
\section{Zel'dovich approximation}
The~Zel'dovich approximation \parencite[hereafter ZA;][]{1970A&A.....5...84Z} bears its name after a~pioneer in~the~study of~large-scale structure, a~Soviet physicist Yakov Zel'dovich. The~ZA provides an~intuitive way to~understand the~emergence of~filamentary structures (cosmic web) and~can realize the~model of~non-linear structure formation even though it is based only on~linear approximations \parencite{2014MNRAS.439.3630W}. The~Zel’dovich approximation predicts the~rich structure of~voids, clusters, sheets, and~filaments observed in~the~Universe.

The~ZA is based on~an~ansatz that particles move in~straight lines in~the~Lagrangian frame
\eq{
\label{eq:ZA}
	\mb x(\mb q, a) = \mb q + D(a)\mb S(\mb q)\,,
}
where $\mb q$ are the~initial positions (Eulerian coordinates) of~a~particle and~$\mb S$ is some displacement field. Inserting equation \eqref{eq:ZA} into the~continuity equation \eqref{eq:continuity_a} yields
\eq{
	-\nabla\cdot S = \delta_0\,.
}
Combining with~the~Poisson equation \eqref{eq:poisson_a}, we can see that \eqref{eq:ZA} represents a~potential flow, $\mb S = -\nabla\phi_V$, where the~velocity potential $\phi_V$ obeys the~Poisson equation
\eq{
	\label{eq:poisson_vel}
	\Delta\phi_V = \delta_0
}
and~has a~simple relation to~the~gravitational potential
\eq{
	\label{eq:vel_new}
	\phi_V=\mu\frac a D \Phi\,.
}
This results to~ZA being
\eq{
	\mb x(\mb q, a) = \mb q - D(a)\mb\nabla \phi_V(\mb q)\,.
}
Note that the~velocity potential is not exactly a~potential of~our velocity field $\mb u$ but
\eq{
	\label{eq:ZA_u}
	\mb u\ZA(\mb x) = -\dddd D a \nabla\phi_V(\mb q)\,.
}
The~ZA differs from other approximations (among other things) in~how this (constant) velocity potential, $\phi_V$, enters the~equations of~motion. To~avoid having different definitions of~the~\textit{real} velocity potential for~the~velocity fields in~each approximation, we take the~equation \eqref{eq:poisson_vel} as defining the~velocity potential $\phi_V$.

The~deformation tensor is defined as
\eq{
	d_{ij}=\partpart{x_i}{q_j}=\delta_{ij}+D\partpart{\nabla_i\phi_V}{q_j}\,.
}
The~eigenvectors of~the~deformation tensor determine the~principal directions of~the~collapse and~the~corresponding eigenvalues determine the~time when the~compression will be infinite. The~density is given through eigenvalues $\lambda_i$ as
\eq{
	\rho=\frac{\bar\rho}{(1-D\lambda_1)(1-D\lambda_2)(1-D\lambda_3)}\,.
}
The~time when the~density in~ZA becomes infinite corresponds to~particles crossing the~paths of~other particles. Once this shell-crossing has occurred, the~approximation has formally broken down, since there are no forces present to~slow down the~particles and~capture them within halos.

In~\autoref{fig:slice_dens_ZA} we show a~comparison of~the~simulations with~ZA at~four different redshifts through the~projected density field. At~redshift $z=\ztwo$ the~cosmic web still looks nice but after that, at~redshifts $z=\zthree, \zfour$, we can see that shell-crossing occurred and~the~overall picture gets blurry. The~large-scale structures remain visible but the~small-scale structures get diluted at~later times.
\begin{figure*}[!htbp]
	\begin{adjustwidth}{-1cm}{-1cm}
	\centering
		\begin{subfigure}{0.5\linewidth}
			\includegraphics[width=1.0\linewidth]{{simulations_approx/dens/za_dens_512m_1p_1024M_200b_z\zone}.png}
			\caption{$z=\zone$}
		\end{subfigure}%
		\begin{subfigure}{0.5\linewidth}
			\includegraphics[width=1.0\linewidth]{{simulations_approx/dens/za_dens_512m_1p_1024M_200b_z\ztwo}.png}
			\caption{$z=\ztwo$}
		\end{subfigure}
		\begin{subfigure}{0.5\linewidth}
			\includegraphics[width=1.0\linewidth]{{simulations_approx/dens/za_dens_512m_1p_1024M_200b_z\zthree}.png}
			\caption{$z=\zthree$}
		\end{subfigure}%
		\begin{subfigure}{0.5\linewidth}
			\includegraphics[width=1.0\linewidth]{{simulations_approx/dens/za_dens_512m_1p_1024M_200b_z\zfour}.png}
			\caption{$z=\zfour$}
		\end{subfigure}
	\end{adjustwidth}
		\caption{Projected density field at~different redshifts for~the~Zel'dovich approximation. Each slice has a~box-length of~$200~\Mpch$ and~is $1~\Mpch$ thick.}
		\label{fig:slice_dens_ZA}
\end{figure*}
%%%%%%%%%%%%%%%%%%%%%%%%%%%%%%%%%%%%%%%%
% Truncated Zel'dovich Approximation
%%%%%%%%%%%%%%%%%%%%%%%%%%%%%%%%%%%%%%%%
\section{Truncated Zel'dovich approximation}
The~shell-crossing and~diffusion of~particles on~small scales in~ZA happens the~sooner the~more power there is on~small scales. \textcite{1993MNRAS.260..765C} suggested an~improvement of~the~ZA by removing power on~these small non-linear scales, i.e. to~set the~initial power spectrum to~zero for~wave-numbers $k$ greater than a~non-linear scale $k_{nl}$ defined as
\eq{
\label{eq:k_nl}
    \frac{a^2(t)}{2\pi^2}\int_0^{k_{nl}}P(k)\dd k=1\,,
}
where the~power spectrum $P(k)$ was defined in~\eqref{eq:pk} as
\eq{
  \label{eq:pk_cp}
  P(k)(2\pi)^3\delta_{\rm D}(k-k')\equiv \left\langle \hat\delta(k)\hat\delta^*(k')\right\rangle\,.
}

\textcite{doi:10.1093/mnras/269.3.626} further improved this truncation by applying a~Gaussian window instead of~an~abrupt cutoff
\eq{
W(k)=e^{-k^2/2k^2_{G}}\,,
}
where the~smoothing scale $k_{G}$ is 1 to~1.5 times $k_{nl}$. This filtering leads to~the~so-called truncated Zel'dovich approximation (TZA). This removes most of~the~strongly non-linear behavior and~allows the~Zel’dovich pancakes to~be seen.

In~\autoref{fig:slice_dens_TZA} we show a~comparison of~the~simulations with~TZA at~four different redshifts through the~projected density field. We see clear differences in~comparison with~ZA. Large-scale structures evolve similarly to~ZA but we see clear artifacts on~small scales given by artificial cutoff at~these scales. These small-scale structures in~filaments are not so diluted as in~the~case of~ZA, however, a~lot of~particles remain in~voids where they are frozen due to~the~lack of~the~initial kick.

\begin{figure*}[!htbp]
	\begin{adjustwidth}{-1cm}{-1cm}
	\centering
		\begin{subfigure}{0.5\linewidth}
			\includegraphics[width=1.0\linewidth]{{simulations_approx/dens/tza_dens_512m_1p_1024M_200b_z\zone}.png}
			\caption{$z=\zone$}
		\end{subfigure}%
		\begin{subfigure}{0.5\linewidth}
			\includegraphics[width=1.0\linewidth]{{simulations_approx/dens/tza_dens_512m_1p_1024M_200b_z\ztwo}.png}
			\caption{$z=\ztwo$}
		\end{subfigure}
		\begin{subfigure}{0.5\linewidth}
			\includegraphics[width=1.0\linewidth]{{simulations_approx/dens/tza_dens_512m_1p_1024M_200b_z\zthree}.png}
			\caption{$z=\zthree$}
		\end{subfigure}%
		\begin{subfigure}{0.5\linewidth}
			\includegraphics[width=1.0\linewidth]{{simulations_approx/dens/tza_dens_512m_1p_1024M_200b_z\zfour}.png}
			\caption{$z=\zfour$}
		\end{subfigure}
	\end{adjustwidth}
		\caption{Projected density field at~different redshifts for~the~truncated Zel'dovich approximation. Each slice has a~box-length of~$200~\Mpch$ and~is $1~\Mpch$ thick.}
		\label{fig:slice_dens_TZA}
\end{figure*}
%%%%%%%%%%%%%%%%%%%%%%%%%%%%%%%%%%%%%%%%
% Frozen-flow Approximation
%%%%%%%%%%%%%%%%%%%%%%%%%%%%%%%%%%%%%%%%
\section{Frozen flow approximation}
The~frozen-flow, or frozen-field, approximation (FFA) was originally proposed by \textcite{Matarrese:1992be} as the~exact solution of~equation \eqref{eq:motion_EdS} in~EdS
\eq{
  \label{eq:FFA_orig}
  \tilde{\mb u}\FFA(\mb x) = \mb u_0(\mb x) = -\mu\nabla\Phi(\mb x) = -\nabla\phi_V(\mb x)\,.
}
The~velocity field $\tilde{\mb u}$ is frozen at~each point to~its initial value, i.e.
\eq{
	\partpart{\tilde{\mb u}}{a} = 0\,.
}
These equations are very similar to~ZA but now the~particles update their velocities to~the~local value of~the~velocity field (not the~initial value), without any memory of~their previous motion. This can be viewed as a~movement of~particles under some force in~a~medium with~very large viscosity. In~our case of~cosmological simulations, gravity represents the~attractive force while Hubble friction represents the~informal equivalent of~a~``damping'' or ``viscous'' force.

Originally, \textcite{Matarrese:1992be} stated three main reasons of~why should the~FFA work:
\begin{itemize}
\item FFA is, by construction, consistent with~linear theory and~follows correctly the~evolution at~early times. Keeping the~linear approximation for~the~velocity potential is justified by the~fact that this quantity is more sensitive to~large wavelength modes than the~density, and~is, therefore, less affected by strongly non-linear evolution.
\item Stream-lines are frozen to~their initial shape, so multistream regions cannot form and~FFA avoids the~formation of~caustics at~a~finite time and~can, therefore, work well after shell-crossing would occur in~ZA. A~particle moving according to~FFA has zero velocity at~minima (or maxima) of~the~gravitational potential. It will slow down its motion when approaching such a~position -- particles in~FFA would need infinite time to~reach such places. Particles move along curved paths and~once they come close to~pancake configurations they curve their trajectories, moving almost parallel to~them, trying to~reach the~positions of~filaments.
\item This type of~dynamics implies an~artificial thickening of~particles around pancakes, filaments, and~knots, which mimics the~real gravitational clustering around these types of~structures.
\end{itemize}

The~definition \eqref{eq:FFA_orig} is, however, no longer valid in~the~general \LCDM\ cosmology where even in~the~linear regime both gravitational potential and~velocity field undergo evolution. We generalize FFA for~the~\LCDM\ cosmology by adding an~extra time dependence in~equation \eqref{eq:FFA_orig}
\eq{
	\label{eq:FFA}
	\mb u\FFA(\mb x, a) = -\dddd D a(a)\nabla\phi_V(\mb x)\,.
}
This velocity field solves \eqref{eq:motion_LCDM} exactly, due to~the~definition of~the~growth factor and~its relation to~the~velocity field in~the~linear regime (see equation \eqref{eq:continuity_a}).

Although equation \eqref{eq:FFA} now does not represent a~\textit{frozen} flow, particles still move along the~same characteristic curves as in~EdS, just with~different velocities. The~particle trajectories are described by the~integral equation
\eq{
	\label{eq:FFA_int}
	\mb x(a) = \mb q + \int_0^a\dd \tilde a\mb u\FFA(\mb x(\tilde a), \tilde a)\,.
}

In~\autoref{fig:slice_dens_FFA} we show a~comparison of~the~simulations with~FFA at~four different redshifts through the~projected density field. Unlike in~the~case of~ZA or TZA there is no shell-crossing and~structures remain clear even at~later times. As the~particles approach minima of~the~gravitational potential, they slow down and~we can see that resulting structures are elongated along stream-lines.
\begin{figure*}[!htbp]
	\begin{adjustwidth}{-1cm}{-1cm}
	\centering
		\begin{subfigure}{0.5\linewidth}
			\includegraphics[width=1.0\linewidth]{{simulations_approx/dens/ff_dens_512m_1p_1024M_200b_z\zone}.png}
			\caption{$z=\zone$}
		\end{subfigure}%
		\begin{subfigure}{0.5\linewidth}
			\includegraphics[width=1.0\linewidth]{{simulations_approx/dens/ff_dens_512m_1p_1024M_200b_z\ztwo}.png}
			\caption{$z=\ztwo$}
		\end{subfigure}
		\begin{subfigure}{0.5\linewidth}
			\includegraphics[width=1.0\linewidth]{{simulations_approx/dens/ff_dens_512m_1p_1024M_200b_z\zthree}.png}
			\caption{$z=\zthree$}
		\end{subfigure}%
		\begin{subfigure}{0.5\linewidth}
			\includegraphics[width=1.0\linewidth]{{simulations_approx/dens/ff_dens_512m_1p_1024M_200b_z\zfour}.png}
			\caption{$z=\zfour$}
		\end{subfigure}
	\end{adjustwidth}
		\caption{Projected density field at~different redshifts for~the~frozen flow approximation. Each slice has a~box-length of~$200~\Mpch$ and~is $1~\Mpch$ thick.}
		\label{fig:slice_dens_FFA}
\end{figure*}
%%%%%%%%%%%%%%%%%%%%%%%%%%%%%%%%%%%%%%%%
% Frozen-potential Approximation
%%%%%%%%%%%%%%%%%%%%%%%%%%%%%%%%%%%%%%%%
\section{Frozen potential approximation}
The~frozen-potential approximation (FPA) was introduced by \textcite{1994MNRAS.266..227B}. They exploit the~fact that the~gravitational potential changes much more slowly than the~density contrast, and~hence may be viewed as essentially frozen. Moreover, like the~velocity potential in~FFA is more sensitive to~large wavelength modes, this is doubly true for~gravitational potential. They solved equation \eqref{eq:motion_EdS} at~each time-step with~the~initial (constant) gravitational potential. In~\LCDM
\eq{
	\label{eq:FPA}
	\dddd{\mb u\FPA}{a} &= -\frac{3}{2a}\left[\mb u\FPA(\mb x,a)\left(1+\Omega_\Lambda(a)\right) + \frac D a \nabla\phi_V(\mb x)\left(1-\Omega_\Lambda(a)\right)\right]
}
and~the~particle trajectories are given as in~the~case of~FFA\,
\eq{
	\mb x(a) = \mb q + \int_0^a\dd \tilde a\mb u\FPA(\mb x(\tilde a), \tilde a)~.
}
Particles now follow the~linear \textit{gravitational} potential (which evolves slightly in~\LCDM) instead of~the~initial \textit{velocity} potential as in~the~case of~FFA. Equation \eqref{eq:FPA} drives the~particle velocities (approximately) to~the~velocities of~FFA but unlike in~the~FFA, particles now keep their inertia. Same as in~the~case of~FFA, the~particles tend to~move along the~pancakes towards regions of~lower potential. The~acceleration used in~FPA is largest in~regions where the~instantaneous velocity vector points along the~gradient of~the~potential, as happen for~particles after they cross the~pancake. In~FFA the~inertia of~particles is ignored, whereas in~the~Zel'dovich approximation inertia is assumed to~dominate over the~change in~the~force field. FPA takes into consideration both factors but assumes a~constant potential.

In~\autoref{fig:slice_dens_FPA} we show a~comparison of~the~simulations with~FPA at~four different redshifts through the~projected density field. At~the~first sight, it looks similarly as in~the~case of~FFA, especially at~early times. For~$z=\zthree, \zfour$ we can see the~differences due to~the~fact that particles have an~inertia and~resulting structures are not so elongated as in~the~case of~FFA.
\begin{figure*}[!htbp]
	\begin{adjustwidth}{-1cm}{-1cm}
	\centering
		\begin{subfigure}{0.5\linewidth}
			\includegraphics[width=1.0\linewidth]{{simulations_approx/dens/fp_dens_512m_1p_1024M_200b_z\zone}.png}
			\caption{$z=\zone$}
		\end{subfigure}%
		\begin{subfigure}{0.5\linewidth}
			\includegraphics[width=1.0\linewidth]{{simulations_approx/dens/fp_dens_512m_1p_1024M_200b_z\ztwo}.png}
			\caption{$z=\ztwo$}
		\end{subfigure}
		\begin{subfigure}{0.5\linewidth}
			\includegraphics[width=1.0\linewidth]{{simulations_approx/dens/fp_dens_512m_1p_1024M_200b_z\zthree}.png}
			\caption{$z=\zthree$}
		\end{subfigure}%
		\begin{subfigure}{0.5\linewidth}
			\includegraphics[width=1.0\linewidth]{{simulations_approx/dens/fp_dens_512m_1p_1024M_200b_z\zfour}.png}
			\caption{$z=\zfour$}
		\end{subfigure}
	\end{adjustwidth}
		\caption{Projected density field at~different redshifts for~the~frozen potential approximation. Each slice has a~box-length of~$200~\Mpch$ and~is $1~\Mpch$ thick.}
		\label{fig:slice_dens_FPA}
\end{figure*}

%%%%%%%%%%%%%%%%%%%%%%%%%%%%%%%%%%%%%%%%
% Particle-mesh simulation
%%%%%%%%%%%%%%%%%%%%%%%%%%%%%%%%%%%%%%%%
\section{Particle-mesh simulation}
For~comparison with~other approximations, we also implemented the~particle-mesh code, i.e. a~simulation where the~gravitation potential evolves according to~the~current position of~particles but the~particles feel only this long-range force without any short-range forces (for~details see \autoref{sec:PM}). The~particles move according to~the~equation of~motion \eqref{eq:motion_LCDM} and~the~gravitational potential evolve according to~Poisson equation \eqref{eq:lin_per_c}.

In~\autoref{fig:slice_dens_all} we show a~comparison of~all approximation methods with~PM at~$z=0$ through the~projected density field. Here we can see side-by-side the~main differences between individual approximation methods: ZA with~diluted small-scale structures, TZA with~(almost) frozen small-scale structures, FFA and~FPA with~elongated structures. We see that PM simulation has more concentrated filaments but also more particles in~voids than ZA, FFA, and~FPA.
% \afterpage{%
\begin{figure*}[!htbp]
	\thisfloatpagestyle{empty}
	\begin{adjustwidth}{-2cm}{-2cm}
	\centering
		\begin{subfigure}{0.4\linewidth}
			\includegraphics[width=1.0\linewidth]{{simulations_approx/dens/za_dens_512m_1p_1024M_200b_z0.00}.png}
			\caption{Zel'dovich approximation}
		\end{subfigure}%
		\begin{subfigure}{0.4\linewidth}
			\includegraphics[width=1.0\linewidth]{{simulations_approx/dens/tza_dens_512m_1p_1024M_200b_z0.00}.png}
			\caption{Truncated Zel'dovich approximation}
		\end{subfigure}
		\begin{subfigure}{0.4\linewidth}
			\includegraphics[width=1.0\linewidth]{{simulations_approx/dens/ff_dens_512m_1p_1024M_200b_z0.00}.png}
			\caption{Frozen-flow approximation}
		\end{subfigure}%
		\begin{subfigure}{0.4\linewidth}
			\includegraphics[width=1.0\linewidth]{{simulations_approx/dens/fp_dens_512m_1p_1024M_200b_z0.00}.png}
			\caption{Frozen-potential approximation}
		\end{subfigure}
		\begin{subfigure}{0.4\linewidth}
			\includegraphics[width=1.0\linewidth]{{simulations_approx/dens/pm_dens_512m_1p_1024M_200b_z0.00}.png}
			\caption{PM simulation}
		\end{subfigure}
	\end{adjustwidth}
		\caption{Projected density field at~redshift $z=0$ for~the~different approximations, all run with~the~same initial conditions. Each slice has a~box-length of~$200~\Mpch$ and~is $1~\Mpch$ thick.}
		\label{fig:slice_dens_all}
\end{figure*}
\floatpagestyle{plain}
% } %% end afterpage


%%%%%%%%%%%%%%%%%%%%%%%%%%%%%%%%%%%%%%%%
% Approximation methods in~modified gravity
%%%%%%%%%%%%%%%%%%%%%%%%%%%%%%%%%%%%%%%%
\section{Approximation methods in~modified gravity}
Here we remind the~basic chameleon equations we want to~solve numerically. The~non-linear Poisson equation
\eq{
\label{eq:cham_u_cp}
	\Delta\left(\chi/\chi_a\right)= C_\chi(a)\left[1+\delta-\left(\frac{\chi_a}{\chi}\right)^{1-n}\right]\,,
}
where
\eq{
	C_\chi(a)\equiv\frac{3H_0^2\Omega_m}{2\Phiscrz}a^{-3\frac{2-n}{1-n}}=\left(a\mu\Phiscra\right)^{-1}\,,
}
the~linear solution
\eq{
\label{eq:chi_lin_x_cp}
	\chi(\mb x, a) = \chi_a(a)\left(1 + \frac{\Phi_G(\mb x, a)}{\Phiscra(a)} \right)\,.
}
the~linear solution in~$k-$space
\eq
{
\label{eq:chi_lin_k_cp}
	\hat{\chi}(k)=-\frac{\chi_a}{1-n}\frac{m^2}{m^2+k^2}\hat{\delta}(k) = -\frac{\beta\bar\rho_m}{\Mpl}\frac{\hat{\delta}(k)}{k^2+m^2}\,.
}
where the~mass of~the~chameleon field is
\eq{
    \label{eq:chi_m_cp}
	m^2(a)\equiv\frac{1-n}{a\mu\Phiscra}\,,
}
and~the~screened solution inside massive objects
\eq{
	\chi=\frac{\chi_a}{\left(1+\delta\right)^{1/(1-n)}}\,.
	\label{eq:chi_bulk_cp}
}

When applying the~approximation methods to~the~chameleon equations, we have three choices on~how to~arrive at~an~approximate solution:
\begin{enumerate}
\item \label{itm:lin_q} Purely linear prediction in~$(\mb q, a)$-space, solution \eqref{eq:chi_lin_x_cp}

\item \label{itm:lin_k} Purely linear prediction in~$(\mb k, a)$-space, solution \eqref{eq:chi_lin_k_cp}

\item \label{itm:nl_x} Non-linear prediction in~$(\mb x, a)$-space, solution \eqref{eq:cham_u_cp}
\end{enumerate}

Choice \ref{itm:lin_q} means that there is no computational overhead and~we can simply take the~chameleon force to~be $2\beta^2a^{-2}$ of~the~gravitational one. This method clearly overestimates the~chameleon force at~early times when the~chameleon's Compton wavelength is short. This is because the~solution \eqref{eq:chi_lin_x_cp} does not take into account the~non-zero mass of~the~field.

A~better choice is to~use \ref{itm:lin_k} where the~non-zero mass is incorporated. This solution adds relatively little computational overhead over normal gravity -- needing (at~least) one Fourier transform and~also extra storage. The~overdensity $\delta(\mb k, a)$ is either the~linearly evolved one, i.e. $\delta(\mb k, a) = D(a)\delta_0(\mb k)$, or we can compute $\delta$ at~each time-step from the~current positions of~particles. This adds extra computation when assigning the~mass of~particles on~the~grid and~one extra Fourier transform to~get $\delta(\mb k, a)$. 
However, we cannot use \eqref{eq:chi_lin_k_cp} blindly to~get a~solution in~real space as this linear approximation breaks down inside massive objects where we would get a~negative solution. This effect is similar to~usage of~linear evolution for~$\delta$ where we can end up with~regions where $\delta<-1$. We need to~check if the~resulting chameleon field is positive and~fix it where it is not. We use the~screening regime value \eqref{eq:chi_bulk_cp} to~get a~positive solution. We will refer to~this prediction as pseudo-linear since it can address some effects of~the~screening mechanism.

The~most expensive choice is \ref{itm:nl_x} where we must iteratively solve nonlinear equations. Unlike other methods, this one can address the~screening regime inside and~near massive objects but at~the~cost of~the~most computational overhead.

%%%%%%%%%%%%%%%%%%%%%%%%%%%%%%%%%%%%%%%%
% Other approximations
%%%%%%%%%%%%%%%%%%%%%%%%%%%%%%%%%%%%%%%%
\section{Other approximations}
Approximations described previously are studied numerically in~detail in~the~next chapter. Here we present a~few of~the~other approximations studied in~the~past or used today.

\subsection{Adhesion approximation}
The~adhesion approximation was introduced in~\textcite{1989MNRAS.236..385G}. To~study the~evolution of~density inhomogeneities they used the~model of~non-linear diffusion (Burger's equation), that gives an~approximate description of~the~growth of~structures at~the~advanced non-linear stage of~gravitational instability.

To~overcome problems of~ZA with~shell-crossing they propose a~solution of~``sticking articles.'' Particles move according to~ZA until they ran into one another. Then they move together, with~the~velocity conserving momentum. This model can be described mathematically by inserting the~viscous term into equations of~motion, simulating attractive forces of~gravity
\eq{
	\dddd{\mb u\AAP}{a}=\nu\partpart{^2\mb u}{\mb x^2}\,.
}
The~Burger's equation has an~analytical solution that can be used to~study the~formation of~structures. For~more information regarding the~adhesion approximation see  also \textcite{1990MNRAS.247..260W,1994ApJ...428...28M}.

\subsection{Stable clustering}
It was proposed by \textcite{1974ApJ...189L..51P} that we can study the~behavior of~a~non-linear clustering in~high-density regions by~assuming, that these regions have undergone virialization and~now maintain a~fixed proper density, hence stable clustering. The~correlation function of~such systems would evolve according to
\eq{
	\xi(r,a)\propto1/\bar\rho\propto a^3\,.
}
\textcite{1991ApJ...374L...1H} developed a~model which enabled interpolating between the~linear theory on~large scales and~the~non-linear predictions of~the~stable clustering on~small scales. They showed that the~non-linear two-point correlation function could be parameterized by a~simple function of~the~linear correlation function
\eq{
	\bar\xi_{NL}=f(\bar\xi_{L})\,,
}
where the~functional form of~$f$ can be derived from the~spherical top-hat model without any shell-crossing. For~the~linear regime $\bar\xi_{L}\ll1$, $f(y)=y$, and~for~non-linear $\bar\xi_{L}\gg1$, $f(y)=y^{3/2}$. For~more information regarding the~stable clustering see  also \textcite{1996MNRAS.280L..19P,2003MNRAS.341.1311S}.
\subsection{Lagrangian perturbation theories of~higher-orders}
The~success of~ZA, the~first-order Lagrangian perturbation theory (LPT), has motivated studies of~higher-order corrections \parencite[see e.g.][]{10.1093/mnras/264.2.375,2002PhR...367....1B,2010MNRAS.403.1859J,2014ApJ...788...63S}. In~the~Lagrangian description, the~spatial coordinates are transformed through the~displacement vector $\Psi$ as
\eq{
	\mb x = \mb q + \Psi(a, \mb x)\,.
}
In~LPT, this displacement vector field is expanded in~a~perturbation series in~the~linear growth function $D$ in~Fourier space. Density perturbations $\delta$ are described as a~function of~the~displacement vector through conservation of~mass.