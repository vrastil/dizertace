\section{Other theories}
In this section we briefly mention some of the other theories od modified gravity. This list is in no way complete and serves only as an example of different approaches to modifications of gravity. See references for further reading.
\subsection{Quintessence}
Quintessence, from the Latin ``fifth element'', is according to ancient and medieval philosophy the fifth element, or ether, supposed to be the constituent matter of the heavenly bodies after air, fire, earth, and water. The name quintessence, or the $Q$ component, was first used by \textcite{1998PhRvL..80.1582C} for the canonical scalar field $\phi$ evolving along a potential $V(\phi)$. Such a dynamical field can reproduce the late-time acceleration with the equation of state $w=w(t)\approx1$. Although quintessence can alleviate the coincidence problem of dark energy via the so-called tracker solution, it still suffers by the fine-tunning problem as the potential needs to be flat enough to lead to the slow-roll inflation today with an energy scale $\rho_{DE}\simeq10^{-120}\Mpl^4$ and a mass scale $m_\phi\lesssim10^{-33}$ eV. However, such fine-tuned potentials can be constructed within the framework of particle physics.

Quintessence is one of the simple models of dark energy as it is a canonical scalar field that interacts with all the other components only through standard gravity. The Lagrangian density for the quintessence field is
\eq{
\label{Qlagr}
\LL_\phi=-\frac12(\partial\phi)^2-V(\phi)
}
We can compute the stress-energy tensor as
\eq{
T^\phi_\uv\equiv\frac{-2}{\dg}\frac{\delta(\dg\LL_\phi)}{\delta g^\uv}=\partial_\mu\phi\partial_\nu\phi-g^\uv\left(\frac12(\partial\phi)^2+V(\phi)\right).
}
Now, the energy density and pressure are given by components of the stress-energy tensor. For FLRW background and $\phi$ only time-dependent we get
\eq{
\rho_\phi=-T^0_0=\frac{1}{2}\dot{\phi}^2+V(\phi)\ \ \ \ p_\phi=\frac{1}{3}T^i_i=\frac{1}{2}\dot{\phi}^2-V(\phi).
}
Equation of state for the quintessence is then
\eq{
\label{eosQ}
w\equiv\frac{p}{\rho}=\frac{\dot{\phi}^2-2V(\phi)}{\dot{\phi}^2+2V(\phi)}\,.
}
We require the condition $w<-1/3$ to realize the late-time cosmic acceleration, which translates into the condition  $\dot{\phi}^2<V(\phi)$, i.e. the potential needs to be shallow enough for the field to evolve slowly along the potential. For a slow-rolling field such as $\dot{\phi}\ll V(\phi)$ equation of state \eqref{eosQ} reduce to $w\approx-1$ as indicated by cosmological measurements.

The variation of \eqref{Qlagr} with respect to $\phi$ gives us the equation of motion for the scalar field $\phi$
\eq{
\ddot{\phi}+3H\dot{\phi}^2+V_{,\phi}=0\,.
}

Depending on which term and when determines the evolution of the field, the quintessence models have been dynamically classified into \textit{freezing} models and \textit{thawing} models \parencite{2005PhRvL..95n1301C}. In the freezing models the field was rolling along the potential in the past, but the movement gradually slows down as the field approaches the minimum of the potential $(\dot{\phi}\rightarrow0)$ and the system enters the phase of the cosmic acceleration $(w\rightarrow-1)$. In the thawing models, the field was initially frozen $(\dot{\phi}\approx0)$ in the early matter era because of the Hubble friction (the term $H\dot{\phi}$) until recently and then it begins to evolve once $H$ drops below $m_\phi$ and $w$ evolves from $-1$.

A potential of the freezing models is for example
\eq{
\label{Qtr}
V(\phi)=M^{4+n}\phi^{-n}\ (n>0),
}
which appears in the fermion condensate model as dynamical supersymmetry breaking \parencite{1999PhRvD..60f3502B}. This potential does not possess a minimum and hence the field rolls down the potential toward infinity. Another example of potential in the freezing models is
\eq{
V(\phi)=M^{4+n}\phi^{-n}\exp{(\alpha\phi^2/\Mpl^2)},
}
which can be constructed in the framework of supergravity \parencite{1999PhLB..468...40B}. This potential has a minimum at which the field is eventually trapped (corresponding to $\dot{\phi}=0$ and hence $w=-1$).

The broader class of potentials belonging to the thawing models are so-called hilltop quintessence models \parencite{2008PhRvD..78l3525D}, in which the scalar field is rolling near a local \textbf{maximum} in the potential but it begins to roll down around the present. A particular example that is well-described by this model is the pseudo-Nambu-Goldstone Boson (PNGB) model of \textcite{1995PhRvL..75.2077F}, for which the potential is given by
\eq{
V(\phi)=M^{4}\left[\cos{(\phi/f)}+1\right].
}
\subsection{K-essence}
Quintessence models are based on a scalar field with a canonical kinetic term and a slowly varying potential. However, in the context of particle physics there appear scalar fields with non-canonical kinetic terms. In \textcite{1999PhLB..458..209A} it is shown that a large class of scalar fields with non-canonical kinetic terms can, without the help of potential terms, drive an inflationary evolution starting from rather generic initial conditions. The Lagrangian density for the k-essence is
\eq{
\label{Klagr}
\LL_K=P(\phi, X),
}
where $X=-\frac12(\partial\phi)^2$ is the canonical kinetic energy and the function $P(\phi, X)$ must vanish for $X\rightarrow0$ (otherwise there would be some potential term). 

The energy-momentum tensor of the k-essence is given by
\eq{
T^K_\uv\equiv\frac{-2}{\dg}\frac{\delta(\dg\LL_K)}{\delta g^\uv}=P_{,X}\partial_\mu\phi\partial_\nu\phi+g^\uv P,
}
which is of the form of a perfect fluid, $T_\uv=(\rho+p)u_\mu u_\nu+g_\uv p$, with a four-velocity $u_\mu=\partial_\mu\phi/\sqrt{2X}$, pressure $p_K=P$ and energy density
\eq{
\rho_K=2XP_{,X}-P.
}
The equation of state of the k-essence is then
\eq{
w_K=\frac{p_K}{\rho_K}=\frac{P}{2XP_{,X}-P},
}
which is $w_K\approx-1$, as long as the condition $XP_{,X}\ll P$ is satisfied.

In the low-energy effective string theory appear higher-order derivative terms coming from $\alpha$ and loop corrections to the tree-level action \parencite{2003PhR...373....1G}. The k-essence action for these theories is for example
\eq{
P=K(\phi)X+L(\phi)X^2.
}
Phantom or ghost scalar fields with a negative kinetic energy $-X$ and $w\lesssim-1$ can also fit the current observations. These ghost fields generally suffer from a quantum instability problem unless higher-order terms in $X$ or $\phi$ are taken into account in the Lagrangian density \parencite{2010deto.book.....A}. The action of the so-called dilatonic ghost condensate model is \parencite{2004JCAP...07..004P}
\eq{
P=-X+e^{\kappa\lambda\phi}X^2/M^4.
}
\subsection{Gauss-Bonnet Dark Energy Models}
In \fR\ gravity one adds the general function of the Ricci scalar. But in principle, one can add general functions of the Ricci and Riemann tensors as well, e.g. $f(R,R_\uv R^\uv,R_{\alpha\beta\gamma\delta}R^{\alpha\beta\gamma\delta},...)$ \parencite{2005PhRvD..71f3513C}. These Lagrangians are generally plagued by the existence of ghosts.  However, there exists a combination of Ricci and Riemann tensors that keeps the equations at second-order in the metric and does not necessarily give rise to instabilities -- so-called Gauss-Bonnet term $\GB$ coupled to a scalar field
\eq{
\GB\equiv R^2-4R_\uv R^\uv+R_{\alpha\beta\gamma\delta}R^{\alpha\beta\gamma\delta}.
}
The Gauss-Bonnet term is the unique invariant for which the highest (second) derivative occurs linearly in the equations of motion and thus ensuring the uniqueness of solutions. The Gauss-Bonnet term naturally arises as a correction to the tree-level action of low-energy effective string theories \parencite{2000PhR...337..343L}. The starting action is given by
\eq{
S=\int\dd^4x\dg\left[\frac{\Mpl^2}{2}R-\frac{\gamma}{2}\left(\nabla\phi\right)^2-V(\phi)+f(\phi)\GB\right],
}
where $\gamma=\pm1$ (+1 for the canonical scalar). For more details see \textcite{2005PhRvD..71l3509N,2006JCAP...06..004N,2013PhRvD..87h4037C}.
\subsection{Braneworld Models}
In the braneworld model of Dvali, Gabadadze, and Porrati \parencite[DGP model][]{2000PhLB..485..208D} the 3-brane is embedded in a Minkowski bulk spacetime with infinitely large extra dimensions. The theory gives rise to the correct 4D potential at short distances whereas at large distances the potential is that of a 5D theory. The action of the theory is
\eq{
S=\frac{M^3_{(5)}}{2}\int\dd^5X\dgt\tR+\frac{\Mpl^2}{2}\int\dd^4x\dg R + \int\dd^4x\dg \LL_m,
}
where $g_{AB}(X)=g_{AB}(x,y)$ denotes a 5D metric for which the 5D Ricci scalar is $\tR$ and $M_{(5)}$ is the 5D Planck mass. Capital letters are used for 5D quantities $(A,B=0,1,2,3,5)$. The brane is located at $y=0$. The induces 4D metric on the brane is denoted by
\eq{
g_\uv(x)\equiv\tilde{g}_\uv(x,y=0).
}
The cross-over scale $r_0$ is defined by
\eq{
r_0\equiv\frac{\Mpl^2}{2M^3_{(5)}}.
}
At short distances $r\ll r_0$ gravity behaves as usual 4D theory, i.e the gravitational potential has correct $1/r$ behavior (except small logarithmic repulsion term). On the other hand at large distances $r\gg r_0$ the potential scales as $1/r^2$ according to the laws of 5D theory.

The presence of the extra dimension has severe consequences on the cosmology as well. It can be shown \parencite{2010deto.book.....A,2009PhLB..674..237M} that the matter-dominated Universe approaches the de Sitter solution $H=r_0\mins$. This cosmological solution drives our Universe into the self-inflationary regime without dark energy. From $H_0\approx r_0\mins$ we get $M_{(5)}\approx10-100$ MeV.
\subsection{Massive Gravity}
The idea to give a mass to the graviton (infrared modification of gravity) is not new and has been investigated since the first years of General Relativity. It is a less minimal theory than \fR\ theories or modified gravities with an extra scalar field because it introduces three new degrees of freedom rather than one. By giving a mass $m$ to the graviton we deform the classical potential to the Yukawa profile $\sim\frac1r e^{-mr}$ which departs from the classical one at distances $r>1/m$. By choosing the graviton mass to be of the order of the Hubble constant $m\sim H$ one can hope to explain the acceleration of the universe without dark energy.

The simplest theory for a non-self-interacting massive graviton is the Fierz-Pauli theory of \textcite{1939RSPSA.173..211F}. The action for a single massive spin 2 particle in flat space, carried by a symmetric tensor field $h_\uv$ is
\eq{
\begin{split}
S=&\int\dd^4x\Big[-\frac12\partial_\lambda h_\uv\partial^\lambda h^\uv+\partial_\mu h_{\nu\lambda}\partial^\nu h^{\mu\lambda}-\partial_\mu h^\uv\partial_\nu h \\
&+\frac12\partial_\lambda h \partial^\lambda h-\frac12 m^2\left(h_\uv h^\uv-h^2\right)\Big]+S_m.
\end{split}
}
Any other combination of $h_\uv h^\uv$ and $h^2$ would lead to instabilities. Varying the action with respect to $h_\uv$ yields the equation of motion
\eq{
R_\uv-\frac12Rg_\uv+\frac12m^2(h_\uv-h\eta_\uv)=\frac{1}{\Mpl^2}T_\uv,
}
where all quantities are linearized around $\eta_\uv$.

Because of the so-called vDVZ discontinuity (``van Dam-Veltman-Zakharov'') in the propagator of a graviton, the Fierz-Pauli theory leads to different physical predictions from those of GR regardless the mass of the graviton (even when $m\to0$). The Vainshtein mechanism \parencite{1972PhLB...39..393V} allows in principle to get rid of the vDVZ discontinuity by introducing non-linear Fierz-Pauli gravity.

The Vainshtein mechanism is another example of the screening mechanism and restores the continuity with GR on scales below the so-called Vainshtein radius $r_V$, defined as
\eq{
r_V\equiv(GM/m^4)^{1/5}.
}
Much below the Vainshtein radius, which grows as the graviton's mass approaches $0$, only linear terms play a crucial role and the GR is restored.

For more about the massive gravity and the Vainshtein mechanism see e.g. \textcite{2013CQGra..30r4001B,2012RvMP...84..671H}.

\subsection{Acceleration without Dark Energy}
So far we studied some kind of dark energy -- either in the form of an exotic matter or by modifying gravity itself. But this need for dark energy as an explanation of the acceleration comes from our observations which are based on the presumption of the homogeneous and isotropic Universe. What we observe are different expansion rates at different distances rather than an increase in the expansion rate at all distances. But this can be caused by strong inhomogeneities in the distribution of matter rather than by an accelerating Universe.
\subsubsection{Void models}
The basic idea behind void models is that we live in the middle of a huge spherical region which is expanding faster because it is emptier than the outside. That means that the Universe as a whole does not accelerate but that we observe an \textit{apparent} cosmic acceleration. The edge of this void should be located around $z\sim0.3-0.5$, the value at which in the standard interpretation we observe the beginning of the acceleration. These models are described by the Lema\^\i tre-Tolman-Bondi (LTB) spherically symmetric metric -- the generalization of a FLRW metric in which the expansion factor along the radial coordinate is different relative to the surface line element $\dd\Omega^2$ \parencite{2013JCAP...02..047D,2006PhRvD..73h3519A}.

The inhomogeneous LTB model matches to the supernovae data and the location of the first acoustic peak of the CMB temperature power spectrum but cannot satisfactorily reproduce the entire CMB angular power spectrum \parencite{2011JCAP...02..013C}. The observed isotropy of the CMB radiation implies that we must live close to the center of the void -- nearer than 15 Mpc \parencite{2006PhRvD..74j3520A}. Moreover, there is no valid mechanism at present to explain the formation of such huge inhomogeneities, let alone one with our Galaxy near the center.
\subsubsection{Backreaction}
Unlike the void models, which regard the acceleration as an apparent one, \textit{backreaction} models try to explain the cosmic acceleration by arranging inhomogeneities so that the deviation from the FLRW metric can produce a real acceleration \parencite{2011CQGra..28w5002S,2004JCAP...02..003R,2005PhRvD..72b3507M}. Because GR equations are non-linear, averaging the inhomogeneities and then solving the GR equations (which is the usual approach) is not the same as first solving the full (inhomogeneous) GR equations and then averaging them -- the expected value of a non-linear function is not the same as the non-linear function of the expected value.

Any large inhomogeneities must be concealed from our sight to fit observations. Strong peculiar velocities instead of strong density fluctuations can do this job, but there are strong constraints on peculiar velocities from e.g., the kinematic Sunyaev--Zel'dovich effect. Moreover, the accompanying anisotropy is another source of observable effects difficult to accommodate with current observations.

\section{Parametrization of models}
We saw in the previous chapter that at the background level the evolution is governed by Friedman equations \eqref{eq:Friedmann} -- \eqref{eq:Friedmann-continuity} and consequently omega parameters \eqref{eq:omega} and Hubble parameter $H$ which can be obtained, e.g., from distance measurements. As we discussed, the cosmological constant is not the only possible explanation of the accelerated expansion and the equation of state \(w\) of dark energy does not have to be exactly \(w=-1\). For the arbitrary equation of state $w_{DE}=p_{DE}/\rho_{DE}$ in the flat Universe with a negligible contribution of radiation we can obtain the following equation
\eq{
    \label{eq:w_de}
    w_{DE}(z)=\frac{(1+z)(E^2(z))'-3E^2(z)}
                   {3\left[E^2(z)-\Omega_{m,0}(1+z^3)\right]}\,,
}
where a prime denotes derivative with respect to \(z\). We see that \(w_{DE}\) cannot be determined solely from \(E(z)\) (obtainable through distance measurements) and we need also present density of matter \(\Omega_{m,0}\). However, if we parametrize \(w_{DE}\) in some way as it is usually done through measurements of \(E(z)\) at several redshifts we can constraint both \(w_{DE}\) and \(\Omega_{m,0}\).

Several parametrization  of \(w_{DE}\) have been proposed so far. We can write such parametrization  in the form
\eq{
    w_{DE}=\sum_{n=0}w_nx_n(z)\,,
}
where he expansions can be given by
\eq{
    & \rm{(i) Redshift:}      & x_n(z) &= z^n\,, \\
    & \rm{(ii) Scale factor:} & x_n(z) &= (1-a)^n=\left(\frac{z}{1+z}\right)\,, \\
    & \rm{(iii) Logarithmic:} & x_n(z) &= \left[\ln\left(1+z\right)\right]^n\,.
}
Parametrization (ii) is usually written for \(n\leq1\) as \(w=w_0+w_a(1-a)\).

For the generic \fR\ gravity, the equation of state is given by \parencite{2013qopu.conf...73B}
\eq{
    w_{DE} = \frac{-(1/2)(F\R R-F)+\ddot F\R+2H\dot F\R-(1-F\R)\left(2\dot H + 3H^2\right) }
                  {(1/2)(F\R - F) - 3H\dot F\R + 3 (1-F\R)H^2}\,,
}
which then can be compared with observations \eqref{eq:w_de}. For different examples of evolution of $w_{DE}$ see, e.g., \textcite{2020arXiv200707717A}.