\section{\textit{f(R)}-gravity}
One of the simplest modified gravity models is the so-called $f(R)$ gravity in which we consider general functions of the Ricci scalar $R$ in the action
\eq{
	\label{eq:S_fr}
	S=\frac{\Mpl^2}{2}\int\dd^4x\dg\left[F(R)\right]+S_m[\psi_m;g_\uv]\,,
}
where $F(R)=R+f(R)$ and $S_m$ is the matter action with matter fields $\psi_m$ which are minimally coupled to gravity, i.e. they interact with gravity only through the determinant of the metric $\dg$ and the canonical kinetic term $-\frac12g^\uv\partial_\mu\psi\partial_\nu\psi$. The matter fields $\psi_m$ obey standard conservation equations and therefore the metric $g_\uv$ corresponds to the Jordan frame.

Variation with respect to the metric $g^\uv$ gives us equation of motion
\eq{
	\label{eq:fR}
	F\R R_\uv-\frac{1}{2}F g_\uv+g_\uv\Box F\R-\nabla_\mu\nabla_\nu F\R=\frac{1}{\Mpl^2}T_\uv\,.
}
For $f(R)=-2\Lambda$ the standard Einstein gravity is reconstructed. Taking the trace of \eqref{eq:fR} we get
\eq{
	\label{eq:fR_tr}
	3\Box F\R+F\R R-2F=\frac{1}{\Mpl^2}T\,.
}
We see that there is a propagating scalar degree of freedom, so-called \textit{scalaron} $F\R$ with mass $m^2=F\R/(3F\RR)$, which corresponds to the scalar field conformally coupled to matter in the Einstein frame.

To get the inflation we need a solution that approaches the de Sitter solution characterized by vacuum space with with a constant positive curvature. Thus $\Box F\R=0$ and \eqref{eq:fR_tr} becomes
\eq{
	F\R R-2F=0.
}
For example the model $F(R)=\alpha R^2$ gives rise to an asymptotically exact de Sitter solution and can be responsible for the inflation in the early Universe. In the model $f(R) = R + \alpha R^2$, the inflation ends when the quadratic term becomes smaller than the linear term. As at the present epoch is the curvature very small this model is not suitable to realize the present cosmic acceleration. Models like $f(R)=-\alpha/R^n$ with $\alpha>0,\ n>0$ could in principle give rise to the present acceleration. However, these models do not satisfy local gravity constraints because of the instability associated with negative values of $f\RR$. Moreover, the standard matter epoch is not present because of a large coupling between the Ricci scalar and the non--relativistic matter.

There are four conditions for the viability of \fR\ models \parencite{Amendola_2007}
\begin{itemize}
	\item $F\R>0\ (\rm{for\ } R>R_0)$, where $R_0$ is the Ricci scalar at the present epoch,\\
	-- required to avoid anti-gravity \parencite{2010deto.book.....A}\\
	\item $F\RR>0\ (\rm{for\ } R>R_0)$,\\
	-- required for consistency with local gravity tests \parencite{2005gr.qc.....5136O}, for the presence of the matter--dominated epoch \parencite{2007PhRvL..98m1302A} and for the stability of cosmological perturbations \parencite{2007PhRvD..75d4004S}\\
	\item $F(R)\rightarrow R-2\Lambda\ (\rm{for\ } R\gg R_0)$,\\
	-- required for consistency with local gravity tests \parencite{2008PhRvD..77b3507T} and for the presence of the matter--dominated epoch \parencite{Amendola_2007}\\
	\item $0<\frac{RF\RR}{F\R}<1\ (\rm{for\ } F\R R-2F=0)$.\\
	-- required for the stability of the late-time de Sitter solution \parencite{1988PhLB..202..198M}
\end{itemize}
Some examples of \fR\ models that satisfy these conditions:
\eq{
	f(R)&=-\mu R_c(R/R_c)^p	&\mbox{for\ }&0<p<1;\ \mu,R_c>0\,,\\
	f(R)&=-\mu R_c\frac{(R/R_c)^{2n}}{(R/R_c)^{2n}+1} 	&\mbox{for\ }&n,\mu,R_c>0\,,\\
	f(R)&=-\mu R_c\left[1-(1+R^2/R^2_c)^{-n}\right] 	&\mbox{for\ }&n,\mu,R_c>0\,,\\
	f(R)&=-\mu R_c\tanh(R/R_c)	&\mbox{for\ }&\mu,R_c>0\,.
}
One of the main prediction of \fR\ gravity is different structure formation history than in \LCDM. For the large--scale structure formation on subhorizon scales \mbox{$k\gg H$} in quasi-static approximation one gets the modified equation for matter density perturbation \parencite{2011RSPTA.369.4947B}
\eq{
	\ddot{\delta}_m+2H\dot{\delta}_m-4\pi G\eff \rho_m\delta_m\approx0\,,
}
where the effective gravitational constant is defined by
\eq{
	G\eff \equiv\frac{G}{1+f\R}\frac{4k^2+3a^2m^2}{3k^2+3a^2m^2}\,.
}
On scales much larger than the scalaron Compton wavelength $m\mins$, gravity is unmodified aside from the overall reduction factor $f\R$. However, on smaller scales the gravitational coupling increases by the factor $4/3$. As the scalaron mass $m$ and the factor $f\R$ depend on curvature (local density), the chameleon mechanism discussed earlier can prevent the detection of this effect in the Solar System.

%%%%%%%%%%%%%%%%%%%%%%%%%%%%%%
% Jordan vs. Einstein Frame
%%%%%%%%%%%%%%%%%%%%%%%%%%%%%%
\subsection{Jordan vs. Einstein Frame}
The action \eqref{eq:S_fr} is described in the so-called Jordan frame, where the matter fields are minimally coupled to the metric and follow geodesics. We can also describe the action in the so-called Einstein frame, where ``standard'' gravity is restored. Using the conformal transformations
\eq{
\label{eins_trans}
\begin{split}
\tilde{g}_\uv &\equiv F\R g_\uv \,, \\
\phi &\equiv\Mpl\sqrt\frac32\ln F\R \,, \\
A(\phi) &\equiv F\R^{-1/2}\,, \\
V(\phi) &\equiv\frac{\Mpl^2}{2} \frac{F\R R - F}{F\R^2}\,,
\end{split}
}
we can rewrite the action \eqref{eq:S_fr}
\eq{
\label{eq:S_ein_fr}
S=\int\dd^4x\dgt\left[\frac{\Mpl^2}{2}\tilde{R} - \frac12(\partial\phi)^2-V(\phi)\right]+S_m[\psi_m;A^2(\phi)\tilde{g}_\uv],
}
where tildes denote quantities in the Einstein frame. This action looks like the Einstein-Hilbert action with minimally coupled scalar but now the matter fields are also coupled with the scalar field via the factor $A(\phi)$ . %Also from the second row of \eqref{eins_trans} can be seen why is the Brans-Dicke parameter restricted to be $\omega>-3/2$.

There is a difference whether one takes action \eqref{eq:S_fr} or \eqref{eq:S_ein_fr} to be the fundamental action defining the modified gravity. In the former one there is only one coupling constant $\beta$, defined by $A(\phi)=\exp(\beta\phi/\Mpl)$, for all matter fields. If one takes the action in the Einstein frame to be the fundamental one the matter action is replaced by $S_m[\psi_m;A^2(\phi)\tilde{g}_\uv]\rightarrow S_i[\psi_i;A_i^2(\phi)\tilde{g}_\uv]$ where one can define the coupling strengths $\beta_i$ to the different matter components to be different. This is very important for tests of modified gravity. For instance, if there is minimal coupling to the baryonic matter -- $\beta_b=0$, Solar System or astrophysical tests do not constraint coupling strength to the cold matter $\beta_c$ whereas the cosmological observation do.

Note that the coupling constant for \fR\ is $\beta=\sqrt{1/6}$ but for other more general theories this coupling can vary. Even for \fR\ theories one expects that loop corrections can change the coupling strength. Also other theories such as Jordan-Brans-Dicke theory, Kaluza-Klein theories and higher derivative theories of gravity, can be formulated in two different ways \parencite{Faraoni:1998qx}.

These two conformally related frames are physically equivalent, i.e. physical observations are frame independent, but the frame dependence of cosmological perturbations has led to confusion in the past. There have been many debates about the (in)equivalence of these frames \parencite{Postma:2014vaa} and whether which one is the physical \parencite{Faraoni:1999hp}. Many contradictory arguments (sometimes incorrect) of both sides result into confusion and ambiguous viewpoints.

It has been shown in \textcite{Magnano:1993bd} that these two frames are \textit{mathematically} equivalent, i.e. every solution in one frame implies an existence of a solution in the other frame and can be mapped into this frame. The confusion about their physical equivalence comes from interpretations of experiments results. For example, \textit{ordinary} cosmology described by Einstein’s theory leads to an expanding universe solution. Coming to the Jordan frame we can interpret the scale factor as a scalar field which is coupled to matter. In this case, the redshift of spectral lines is no longer interpreted as an effect due to the expansion of the Universe, but due to a growth of coupling constants such that the present transition energies are higher than those in the past. Hence the Jordan frame physicist does not see an expanding Universe, but growing coupling constants. Nevertheless, the measured redshift of spectral lines is the same for both frames.

Both frames have some issues with fundamental principles. In the Jordan frame the weak energy condition can be violated and hence states with the negative energy are possible. Moreover, there is no guarantee of stability of ground state. All \textit{classical} fields are believed to satisfy the energy condition but no so in quantum theories. On the other hand in the Einstein frame the weak energy condition is satisfied but due to the non-universal coupling of the matter fields the equivalence principle is violated. However this violation is only weak and can pass the Solar system tests.

%%%%%%%%%%%%%%%%%%%%%%%%%%%%%%
% Screening Mechanisms
%%%%%%%%%%%%%%%%%%%%%%%%%%%%%%
\subsection{Screening Mechanisms}
We know that general relativity with the cosmological constant and assumptions about cold dark matter can describe our universe very well. That means that any modified cosmology must be able to recover \LCDM\ cosmology to a high accuracy. This is not normally an issue. However, since modifications of GR typically involve extra scalar degree of freedom there are interactions with matter that are unavoidable -- no symmetry can prevent all couplings to the standard model. This coupling to matter means that there should be a fifth force. Because we do not see any fifth forces or modifications of gravity in the laboratory or in the Solar System we need to suppress these fifth forces -- we need some sort of a \textit{screening mechanism}.

A nature of the screening mechanisms can be different. Let us start from \eqref{eq:S_ein_fr} with generalized kinetic term $-\frac12 Z(\phi,\partial\phi,...)(\partial\phi)^2$. We can solve the equations of motion for the background in a minimum of a potential $V(\phi)$ and write $\phi=\phi_0+\delta\phi$, where $\phi_0$ is a background solution and $\delta\phi$ is a fluctuation. The Lagrangian density for the fluctuations to the second order (first--order vanishes) is
\eq{
\LL\propto-\frac12 Z(\phi_0)(\partial\delta\phi)^2+\frac12 m^2(\phi_0)\delta\phi^2+\frac{\beta(\phi_0)}{M_p}\delta\phi\delta T,
}
where $m^2(\phi)\equiv V_{,\phi\phi}(\phi)$. Now, any of these three terms can serve as a screening term:
\begin{itemize}
	\item  \textit{Large inertia} -- a large $Z$ makes it hard for the scalar to propagate and leads to the kinetic type of the screening, where first or second derivatives being important; e.g. Galileons \parencite{2009PhRvD..79f4036N}, massive gravity \parencite{2012RvMP...84..671H} or Vainshtein mechanism \parencite{2013CQGra..30r4001B};
	\item \textit{Large mass} --  a large $m$ means the scalar propagates only over short distances and leads to the chameleon type of the screening, where in regions of high density, such as on the Earth, the field acquires a large mass -- the Chameleon mechanism \parencite{Waterhouse:2006wv};
	\item \textit{Weak coupling} -- a small $\beta$ in regions of high density makes the interaction with matter fields weaker and leads to symmetron \parencite{2010PhRvL.104w1301H}) or varying dilaton \parencite{Damour:1994zq,2011PhRvD..83j4026B} theories.
\end{itemize}