\section[Chameleon Gravity]{Chameleon Gravity\footnote{Some parts of this section have already been published in \textcite{mastersthesis_vrastil}.}}
\label{sec_cham}
We now describe \fR\ gravities exhibiting the~chameleon mechanism in~more detail. As stated, this mechanism uses the~large mass of~the~chameleon field in~high-density regions and~chameleon gravity can satisfy tests of~the~equivalence principle in~the~Solar System. The~action of~a~chameleon scalar field $\chi$ in~the~Einstein frame is given by the~action \eqref{eq:S_ein_fr}. Varying the~action with~respect to~the~field $\chi$ one can obtain the~equation of~motion
\eq{
	\label{eom:cham}
	\Box\chi=V_{,\chi}-\sum_i\frac{\beta_i}{\Mpl}e^{4\beta_i\chi/\Mpl}g^\uv_{(i)}T^{(i)}_\uv,
}
where $T^{(i)}_\uv$ is the~stress-energy tensor for~the~$i$-th matter component. For~a~perfect isotropic fluid the~equation of~motion is
\eq{
	\Box\chi=V_{,\chi}+\sum_i(1-3w_i)\frac{\beta_i}{\Mpl}\rho_i e^{(1-3w_i)\beta_i\chi/\Mpl}.
}
This equation could be read as
\eq{
	\Box\chi=V_{\eff,\chi}\left(\chi\right),
}
where the~effective potential $V_{\eff}$ is defined by
\eq{
	V_{\eff}\left(\chi\right)\equiv V(\chi)+\sum_i\rho_i e^{(1-3w_i)\beta_i\chi/\Mpl}.
}
If the~couplings $\beta_i$ are the~same for~each matter component with~the~same $w$ (we can omit the~radiation in~the~sum) and~the~overall density is $\rho=\sum_i\rho_i$, then the~effective potential reads
\eq{
	V_{\eff}\left(\chi\right)\equiv V(\chi)+\rho e^{(1-3w)\beta\chi/\Mpl}.
}
For~the~quasi-static and~weak $(\beta\chi/\Mpl\ll1)$ field in~a~weak gravity background (the~Minkowski background) with~the~non-relativistic matter, the~equation further simplifies as
\eq{
	\label{eq:cham}
	\Delta \chi=\frac{\beta}{\Mpl}\rho+V_{,\chi},
}
which looks like the~normal Poisson equation but with~an~extra non-linear term.
\subsection{Chameleon Force}
The~interaction of~the~chameleon field with~matter is described by the~conformal coupling \eqref{eins_trans}. Free matter fields $\psi_m^{(i)}$ follow geodesics of~the~Jordan frame metric. In~the~Einstein frame, they follow modified trajectories affected by the~chameleon field \parencite{Waterhouse:2006wv}
\eq{
\frac{\dd^2x^\mu}{\dd\tau^2}+\Gamma^\mu_{\alpha\beta}\dddd{x^\alpha}{\tau}\dddd{x^\beta}{\tau}=-\frac{\beta_i}{\Mpl}\left(2\chi_{,\alpha}\dddd{x^\alpha}{\tau}\dddd{x^\mu}{\tau}+g^{\beta\mu}\chi_{,\beta}\right).
}
Note that the~chameleon force violates the~weak Equivalence Principle only if there exist two matter species with~differing values of~$\beta_i$. In~the~non-relativistic limit, a~test particle of~mass $m$ of~species $i$ in~a~static chameleon field $\chi$ is moving under a~force $\mb{F}_\chi$ given by
\eq{
\label{cham_force}
\frac{\mb{F}_\chi}{m}=-\frac{\beta_i}{\Mpl}\mb{\nabla}\chi
}
\subsection{Chameleon mechanism}
As discussed previously, we need some sort of~a~screening mechanism to~avoid Solar System tests of~GR. It means as seen from \eqref{cham_force} that the~chameleon potential needs to~approach some constant value in~dense regions or at~least have a~marginally suppressed amplitude.

Suppose we have a~background solution $\chi_0$ which minimizes the~effective potential with~$\rho=\rho_0$. For~small fluctuations $\chi=\chi_0+\delta\chi$ and~$\rho=\rho_0+\delta\rho$ we can linearize \eqref{eq:cham} to~obtain
\eq{
\label{eq:cham_lin}
\Delta \delta\chi=\frac{\beta}{\Mpl}\delta\rho+m^2_0\delta\chi,
}
where
\eq{
m^2_0\equiv V_{,\chi\chi}(\chi_0).
}
Except for~the~screening term, the~equation \eqref{eq:cham_lin} has the~same behavior as the~Poisson equation for~the~Newtonian potential $\Phi_G$. For~a~spherically symmetric density profile, this gives solution
\eq{
\chi=\chi_0+2\beta\Mpl\Phi_G\left(r\right)e^{-m_0 r}.
}
As the~objects in~the~background become more massive (larger and/or denser) the~Newtonian potential grows larger (in~magnitude) and~so the~deviation of~$\chi$ from background solution $\chi_0$. At~some point, this deviation is no longer small and~the~potential term in~\eqref{eq:cham} cannot be treated perturbatively. It starts canceling the~first source term and~eventually the~field $\chi$ posses a~new value which minimizes the~effective potential inside an~object.

This is the~essence of~the~chameleon mechanism. Let us derive the~mechanism more properly and~exactly. 
\subsection{Chameleon Profile}
\label{cham_prof}
To~obtain the~chameleon behavior described above we need to~choose a~chameleon potential $V(\chi)$ with~the~right properties. To~have a~screening mechanism in~\eqref{eq:cham} we need $V_{,\chi}<0$ to~cancel the~source term and~$V_{,\chi\chi}>0$ to~have a~real mass of~the~field and~stable behavior of~perturbations.

We wish to~find a~solution for~spherically symmetric matter distributions of~a~single species of~pressureless matter such that
\begin{equation*}
\rho(r)=
\begin{cases}
\rho_c & r<R_s \\
\rho_0 & r>R_s,
\end{cases}
\end{equation*}
where $\rho_c>\rho_0$. Further, we define $\chi_c$ and~$\chi_0$ with~their masses $m_c$ and~$m_0$ (the~masses of~small fluctuations about $\chi_c$ and~$\chi_0$) such as
\begin{align*}
V_{\eff,\chi}\left(\chi_c\right)_{|\rho=\rho_c}&\equiv0	&	m^2_c&\equiv V_{\eff,\chi\chi}\left(\chi_c\right) \\
V_{\eff,\chi}\left(\chi_0\right)_{|\rho=\rho_0}&\equiv0	&	m^2_0&\equiv V_{\eff,\chi\chi}\left(\chi_0\right).
\end{align*}
In~the~background with~low density, the~curvature of~the~potential is much shallower, corresponding to~a~light scalar that mediates a~long-range force. Inside the~object of~high density, the~scalar acquires a~large mass, and~the~force shuts off.

In~spherical coordinates assuming spherical symmetry, equation \eqref{eq:cham} becomes
\eq{
\label{eq_cham_r}
\frac{\dd^2\chi}{\dd r^2}+\frac{2}{r}\dddd{\chi}{r}=\frac{1}{r}\frac{\dd^2\left(r\chi\right)}{\dd r^2}=V_{,\chi}\left(\chi(r)\right)+\frac{\beta}{\Mpl}\rho(r).
}
We must impose two boundary conditions which are
\begin{align*}
\dddd{\chi}{r}(r=0)&=0 \\
\chi(r\rightarrow\infty)&=\chi_0.
\end{align*}
The~first one corresponds to~a~non-singularity of~the~solution at~the~origin while the~later one ensures that the~chameleon force vanishes at~the~infinity (as $\dd\chi/\dd r\rightarrow0$).

The~equation \eqref{eq_cham_r} drives the~field $\chi$ toward the~$\chi_0$ outside the~object and~toward $\chi_c$ inside the~object. To~solve \eqref{eq_cham_r}, we must do several approximations. Outside the~object, we assume that the~field sits near the~extreme $\chi_0$ and~we can linearize our equation
\eq{
\frac{1}{r}\frac{\dd^2\left(r\chi\right)}{\dd r^2}=m^2_0(\chi-\chi_0),
}
with~the~decaying solution
\eq{
\chi(r)=-\frac{\beta}{4\pi\Mpl}\frac{\tilde{M}}{r}e^{-m_0 r}+\chi_0.
}
Note that the~integration constant $\tilde{M}$ is not generally the~mass of~the~object $M_c$ as in~the~case of~the~Newtonian potential because it is determined by the~field inside the~object which has different behavior than the~Newtonian potential. As we will see later, for~small Newtonian potentials (in~magnitude) this effective mass $\tilde{M}\approx M_c$ but as the~potential grows larger part of~the~object's mass is screened away $\tilde{M}< M_c$.

Inside the~object, we use one of~the~two approximations based on~the~initial value of~$\chi_i\equiv\chi(0)$ -- either $\chi_i\approx\chi_c$ or $\chi_i\gg\chi_c$ .
\subsubsection{Thin-shell regime}
In~the~\textit{thin-shell} regime, the~field initially sits very close the~minimum $\chi_c$, i.e. we require
\eq{
(\chi_i-\chi_c)/\chi_c\ll1.
}
The~field is frozen near this value until the~friction term is sufficiently small to~allow the~field to~roll. This ``moment'' is denoted by $R_{roll}$. As soon as $\chi$ is displaced significantly from $\chi_c$ we may neglect the~potential term in~\eqref{eq_cham_r}. This gives us the~solution
\eq{
\chi(r)=
\begin{cases}
\chi_c & 0<r<R_{roll} \\
\frac{\beta}{6\Mpl}\rho_cr^2+\frac{A}{r}+D & R_{roll}<r<R_s.
\end{cases}
}
We have boundary conditions coming from the~requirement on~matching $\chi$ and $\dd\chi/\dd r$ at~$R_{roll}$, namely: $\chi=\chi_c$ and~$\dd\chi/\dd r=0$ at~$r=r_{roll}$. This fixes our constants and~the~solution is
\eq{
\label{eq_thin}
\chi(r)=
\begin{cases}
\chi_c & 0<r<R_{roll} \\
\frac{\beta\rho_c}{3\Mpl}\left(\frac{r^2}{2}+\frac{R^3_{roll}}{r}\right)-\frac{\beta\rho_cR^2_{roll}}{2\Mpl}+\chi_c & R_{roll}<r<R_s.
\end{cases}
}
The~approximation of~separating the~solution into the~two regions only makes sense if $(R_s-R_{roll})/R_s\ll1$. Otherwise, there is no clear separation between the~two regions, and~one needs a~solution valid over the~entire range $0<r<R_s$. In~\autoref{sec:num_cham} we solve equation \eqref{eq_cham_r} numerically and~we will check these approximations against numerical solutions.

With~approximation $(R_s-R_{roll})/R_s\ll1$, we can determine the~effective mass of~the~object $\tilde{M}$ from the~requirement $\chi(R_s^-)=\chi(R_s^+)$ and~$\dd\chi/\dd r(R_s^-)=\dd\chi/\dd r(R_s^+)$.
\eq{
\tilde{M}=\frac{3\Delta R_s}{R_s}M_c,
}
where
\eq{
\frac{\Delta R_s}{R_s}\equiv\frac{\chi_0-\chi_c}{6\beta\Mpl|\Phi_G(R_s)|}\approx\frac{R_s-R_{roll}}{R_s}\ll1.
}
This qualitative derivation of~the~thin-shell regime is using too much assumptions and~can be done more precisely without ignoring some of~the~terms but then it is harder to~see the~principle of~the~thin-shell effect. For~more details see e.g. \textcite{Tamaki:2008mf,2007PhRvD..75f3501M,Waterhouse:2006wv}.
\subsubsection{Thick-shell regime}
In~the~\textit{thick-shell} regime, the~field is initially sufficiently displaced from the~minimum -- $\chi_i\gg\chi_c$ that it begins to~roll almost immediately (no friction term). Hence the~interior solution is most easily obtained by taking the~$R_{roll}=0$ in~\eqref{eq_thin} and~replacing $\chi_c$ by $\chi_i$
\eq{
\label{eq_thick}
\chi(r)=\frac{\beta\rho_cr^2}{6\Mpl}+\chi_i\ \ \ 0<r<R_s.
}
By matching the~interior and~exterior solutions, we obtain
\eq{
\begin{split}
\chi_i &=\chi_0-3\beta\Mpl\Phi_G(R_s)\\
\tilde{M} &=M_c,
\end{split}
}
which is the~linear regime with~no screening. From the~definition of~$\Delta R_s/R_s$ we also obtain
\eq{
\frac{\Delta R_s}{R_s}\equiv\frac{\chi_0-\chi_c}{6\beta\Mpl|\Phi_G(R_s)|}>1.
}
\subsubsection{Thin-shell suppression factor}
The~chameleon force outside the~object (where experiments take place) comparing to~the~Newtonian force is
\eq{
\begin{split}
\label{eq_cham_suppression}
\frac{F_{thick}}{F_N}&=2\beta^2 \\
\frac{F_{thin}}{F_N}&=2\beta^2\frac{3\Mpl\left(\chi_0-\chi_c\right)}{\beta\rho_cR^2_c},
\end{split}
}
where we ignore the~term $m_0 r\ll1$. Therefore for~the~coupling $\beta$ of~order unity, the~chameleon force is as strong as gravity unless it is screened away by the~thin-shell effect.
%%%%%%%%%%%%%%%%%%%%%%%%%%%%%%
% HU-SAWICKI
%%%%%%%%%%%%%%%%%%%%%%%%%%%%%%
\subsection{Hu-Sawicki \texorpdfstring{\textit{\lowercase{f}(R)}}{fR} Model}
We wish to~study a~class of~$f(R)$ models that accelerate cosmic expansion at~late times, without the~cosmological constant, while satisfying both cosmological and~Solar System tests. We consider the~family of~Hu-Sawicki $f(R)$ models \parencite{Hu-Saw}. The~action of~these models is given by \eqref{eq:S_fr} and~$f(R)$ has a~broken power-law form
\eq{
	f(R)=-M^2\frac{c_1(R/M^2)^m}{c_2(R/M^2)^m+1}\,,
}
where the~mass scale $M^2\equiv\bar\rho_0/3\Mpl^2$, $m>0$, and~$c_1$ and~$c_2$ are dimensionless parameters such that at~high redshifts \LCDM\ cosmology is restored.

The~formulation of~modified gravity in~this frame leads to~second-order differential equations of~motion \eqref{eq:fR} for~$R$ and~fourth-order field equations for~$g_\uv$. With~a~conformal transformation \eqref{eins_trans} we may rewrite these equations in~the~Einstein frame with~second-order differentials only \parencite[see, e.g.,][]{CHIBA20031}. In~the~Einstein frame, the~Hu-Sawicki models correspond to~chameleon gravity with~the~potential
\eq{
	V(\chi) &= \Mpl^2\Lambda-\frac{\beta\bar\rho_0}{n\Mpl}\left(2\beta\Mpl\Phiscrz\right)^{1-n}\chi^n\,, \\
    V_{,\chi}(\chi) &= -\frac{\beta}{\Mpl}\bar\rho_0\left(\frac{2\beta\Mpl\Phiscrz}{\chi}\right)^{1-n}\,,
}
where $\beta=\sqrt{1/6}$ and~the~power-law exponent $n$ and~screening potential $\Phiscrz$ are now the~free parameters of~the~theory. The~screening potential has the~following relation to~the~present scalaron value in~$f(R)$-gravity:
\eq{
    \Phiscrz=\frac{3}{2}\ln{(1+f_{R0})}\approx\frac{3}{2}f_{R0}.
}
The~chameleon obeys the~equation of~motion \eqref{eom:cham} which reduces for~our study case (non-relativistic pressureless matter) to
\eq{
\label{eq:cham_husa}
	\Delta \chi = \frac{\beta}{\Mpl}\rho - \frac{\beta}{\Mpl}\bar\rho_0\left(\frac{2\beta\Mpl\Phiscrz}{\chi}\right)^{1-n}
}
We rescale the~equations to~units in~which we can clearly see the~role of~the~screening potential $\Phiscr$ and~its relation to~the~gravitational potential $\Phi_G$. We start by defining a~few special values of~the~chameleon potential -- the~current background value
\eq{
	\chi_0\equiv2\beta\Mpl\Phiscrz\,,
}
the~background value at~a~given time (for~a~matter-dominated universe)
\eq{
	\chi_a(a)\equiv \chi_0 a^{3/(1-n)}
}
and~the~value of~the~screening potential at~a~given time
\eq{
	\Phiscra\equiv\Phiscrz a^{\frac{5-2n}{1-n}}\,.
}
With~these definitions, we rewrite equation \eqref{eq:cham} as
\eq{
\label{eq:cham_u}
	\Delta\left(\chi/\chi_a\right)= C_\chi(a)\left[1+\delta-\left(\frac{\chi_a}{\chi}\right)^{1-n}\right]\,,
}
where the~pre-factor $C_\chi(a)$ is defined by
\eq{
	C_\chi(a)\equiv\frac{3H_0^2\Omega_m}{2\Phiscrz}a^{-3\frac{2-n}{1-n}}=\left(a\mu\Phiscra\right)^{-1}\,.
}
\subsubsection{Linear prediction}
Equation \eqref{eq:cham_u} is similar to~the~Poisson equation for~the~gravitational potential and~gives meaning to~the~screening potential $\Phiscra$. If we assume $\chi\approx\chi_a$, then
\eq{
	\label{eq:chi__scr_mean}
	\Delta\left(\chi/\chi_a\right) \approx \left(\mu\Phiscra\right)^{-1}\frac{\delta}{a} = \Delta\left(\Phi_G/\Phiscra\right)
}
and~we can write down a~linear solution as
\eq{
\label{eq:chi_lin_x}
	\chi(\mb x, a) = \chi_a(a)\left(1 + \frac{\Phi_G(\mb x, a)}{\Phiscra(a)} \right)\,.
}
Here we can clearly see the~role of~the~(time-dependent) screening potential $\Phiscra$ -- as long as $|\Phi_G| < \Phiscra$ we have a~valid solution but once the~gravitational potential is large enough (in~its negative values) the~linear solution breaks down, as the~chameleon field would become negative.

We may derive a~more accurate solution in~Fourier space. If the~chameleon field sits near its background value, i.e. $\chi=\chi_a\left(1 + \delta\tilde\chi \right)$, where $\delta\tilde\chi \ll 1$, we can rewrite \eqref{eq:cham_u} as
\eq{
	\Delta\delta\tilde\chi=\frac{m^2}{1-n}\delta + m^2\delta\tilde\chi\,,
}
where the~mass of~the~chameleon field is
\eq{
    \label{eq:chi_m}
	m^2(a)\equiv\frac{1-n}{a\mu\Phiscra}\,.
}
This equation has a~solution in~$k-$space of~the~form
\eq
{
\label{eq:chi_lin_k}
	\hat{\chi}(k)=-\frac{\chi_a}{1-n}\frac{m^2}{m^2+k^2}\hat{\delta}(k) = -\frac{\beta\bar\rho}{\Mpl}\frac{\hat{\delta}(k)}{k^2+m^2}\,.
}

The~other regime, which can be solved approximately, is the~screened regime inside massive objects. When the~solution \eqref{eq:chi_lin_x} breaks down, and~if $\delta(x)$ is approximately constant, the~solution of~equation \eqref{eq:cham_u} is
\eq{
	\chi=\frac{\chi_a}{\left(1+\delta\right)^{1/(1-n)}}\,.
	\label{eq:chi_bulk}
}
Because $\chi_a$ is constant in~space the~chameleon force \eqref{cham_force} vanishes in~this screened regime.

In~\autoref{fig:chi_evol} we show the~evolution of~background parameters of~the~chameleon field -- Compton wavelength $\lambda_c=m^{-1}$, background value of~the~chameleon field $\chi_a$ and~the~screening potential $\Phiscra$ -- for~different values of~the~power-law exponent $n$ and~the~screening potential $\Phiscrz$.

\begin{figure}[hbt]
\centering
	\begin{subfigure}{1.0\textwidth}
        \includegraphicscustomlegend{simulations_approx/chi/chi_evol}
	\end{subfigure}
	\begin{subfigure}{1.0\textwidth}
		\includegraphicscustom{simulations_approx/chi/chi_evol}
	\end{subfigure}
    \caption{Evolution of~background parameters of~the~chameleon field. From top to~bottom: Compton wavelength $\lambda_c$, chameleon field $\chi_a$ and~screening potential $\Phiscra$.}
    \label{fig:chi_evol}
\end{figure}

The~background value of~the~Compton wavelength informs us about the~global behavior of~the~chameleon field whereas the~screening potential describes its behavior locally. At~high redshifts, the~chameleon's Compton wavelength is too short to~have any effects -- on~large scales, due to~its low background value, and~on~small scales due to~a~low value of~the~screening potential. At~lower redshifts, the~chameleon field starts to~affect matter, initially only on~small scales but with~the~passage of~time also on~large scales. We thus expect the~strongest effects to~be on~small scales.
\subsection{Numerical solutions}
\label{sec:num_cham}
In~this section, we will show the~results of~numerical solutions of~the~chameleon profile. We will solve the~equations for~the~Hu-Sawicki \fR\ model, \eqref{eq:cham_husa}. In~this section we will focus only on~systems with~spherical symmetry \eqref{eq_cham_r}, i.e. we will solve the~following equation
\eq{
	\label{eq:cham_husa_r}
	\frac{\dd^2\chi}{\dd r^2}+\frac{2}{r}\dddd{\chi}{r} = \frac{\beta}{\Mpl}\rho - \frac{\beta}{\Mpl}\bar\rho_0\left(\frac{\chi_0}{\chi}\right)^{1-n}
}
Our algorithm for~finding solutions to~\eqref{eq:cham_husa_r} uses the~shooting method \parencite{10.5555/42249} and~is based on~the~original algorithm of~\textcite{mastersthesis_vrastil}. We further improved the~code applicability, readability, and~its parametrization. The~code is publicly available at~\code{\url{https://github.com/vrastil/chi_r_solver}}.

\subsubsection{Stars}
We will first consider a~case where some approximate solutions exist -- a~compact spherical object of~constant density $\rho_c$ surrounded by the~background of~density $\rho_0$ as discuss in~\autoref{cham_prof}. We expect that for~low-mass objects the~chameleon field will track the~Newtonian potential while for~massive objects the~chameleon field will be frozen inside the~sphere and~outside it will be following the~Newtonian behavior but with~decreased amplitude.

In~\autoref{fig:starlike} we show results for~the~chameleon profile. We used the~notation $\tilde\chi\equiv(\chi-\chi_0)/2\beta\Mpl$ for~better comparison with~the~Newtonian potential. We see that for~$\Phiscr>\Phi_G$ we have an~unscreened solution as expected. For~lower values of~$\Phiscr$ the~field is frozen inside the~object and~have lower amplitude outside the~object as expected from analytical solutions.
\begin{figure}
	\centering
	\includegraphics[width=1.0\linewidth]{{spherical_cham/starlike}.png}
	\caption{Chameleon profile for~several screening potentials. The~top solution is in~the~linear regime and~is identical to~the~gravitational potential. The~other two solutions are in~the~screened regime and~the~amplitude of~the~field is suppressed.}
	\label{fig:starlike}
\end{figure}

Let us focus on~the~regime which cannot be treated analytical, i.e. regime between thin-shell and~thick-shell solutions. This regime corresponds to~the~situation when the~linear approximation (thick-shell) breaks down inside the~object, i.e. the~gravitational potential cancels screening potential somewhere inside the~object. We will denote $\Req$ the~\textit{equivalence radius} -- radius at~which the~Newtonian potential equals the~screening potential $|\Phi_G(\Req)|=\Phi_s$. By letting the~equivalence radius posses also negative values such as $(1+|\Req|/R_s)|\Phi_G(0)|=\Phi_s$ we can clearly distinguish between the~linear $(\Req<0)$ and~the~screening $(\Req>0)$ regime.

In~\autoref{fig:starlike_forces} we show the~behavior of~the~chameleon fifth force in~this regime. We see that for~the~linear regime the~fifth force is as strong as standard gravity (up to~the~factor $2\beta^2$). For~$\Req\ll R_s$ the~chameleon field manages to~catch up with~the~linear solution inside the~object and~there is no screening outside. As the~$\Req$ grows the~field is not able to~catch up with~the~linear solution while inside the~object and~the~force outside is screened.
\begin{figure}
	\centering
	\includegraphics[width=1.0\linewidth]{{spherical_cham/starlike_forces}.png}
	\caption{Chameleon force relative to~the~standard gravitational force for~several screening potentials (given through the~equivalence radius). For~$\Req\ll R_S$ there is no screening outside the~object. As the~$\Req$ grows the~chameleon enters the~screened regime.}
	\label{fig:starlike_forces}
\end{figure}

\subsubsection{NFW Halo}
The~Navarro-Frenk-White (NFW) profile proposed by \textcite{1996ApJ...462..563N} describes the~distribution of~cold dark matter. The~NFW profile of~matter overdensity is given by
\eq{
	\label{NFW_rho}
	\delta\rho_{\rm NFW}(r)=\frac{\rho_c}{r/r_s\left(1+r/r_s\right)^2},
}
where $\rho_c$ is the~density scale and~$r_s$ is the~scale radius. We will also be using the~dimensionless radius $x\equiv r/r_s$. The~total mass of~the~halo is divergent (logarithmically) so we take a~cut-off at~the~radius $r_{200}$, which is defined as the~radius at~which the~density is 200 times the~critical density. Then the~mass of~the~halo is
\eq{
	M_{200}=\int_0^{r_{200}}4\pi r^2\rho(r)\dd r=4\pi\rho_cr_s^3\left(\ln(1+c)-\frac{c}{c+1}\right),
}
where $c\equiv r_{200}/r_s$ is the~concentration of~the~halo. For~a~given mass the~halo is fully characterized by the~concentration.

For~NFW halo the~density is not constant as in~the~case of~compact spherical objects (stars). The~chameleon mass and~the~screened solution are therefore also not constant and~one does not have analytical solutions as in~the~case of~stars. However, for~realistic halo, the~density varies on~scales much larger than the~Compton wavelength of~the~chameleon. In~such cases, we can treat the~field as frozen in~the~same sense as in~the~case of~stars. Therefore we expect the~chameleon field to~behave in~a~similar way as in~the~case of~stars, i.e. to~follow the~Newtonian potential in~the~linear case and~to~have screened behavior for~more massive halos.

In~\autoref{fig:nfwlike_forces} we show the~chameleon fifth force. We see that the~behavior is indeed similar to~stars although the~field catches up much later. This is because there is no sudden drop in~density (and~mass of~the~field) where the~chameleon can start to~behave as free but rather the~mass is slowly dropping. This indicates that it will be much harder to~detect the~chameleon fifth force on~scales of~galaxies than for~star-like objects. This is of~course true only assuming the~screening potential is the~same on~all scales.
\begin{figure}
	\centering
	\includegraphics[width=1.0\linewidth]{{spherical_cham/nfwlike_forces}.png}
	\caption{Chameleon force relative to~the~standard gravitational force for~several screening potentials (given through the~equivalence radius). For~$\Req\ll R_S$ there is no screening outside the~object. As the~$\Req$ grows the~chameleon enters the~screened regime.}
	\label{fig:nfwlike_forces}
\end{figure}

The~meaning of~the~screening potential and~its connection to~the~Newtonian potential in~\eqref{eq:chi__scr_mean} is given by the~value of~the~field at~the~background, i.e. value that minimizes right side of~\eqref{eq:cham_u}. However, objects like stars or galaxy halos do not sit directly in~the~overall average density of~the~universe but rather in~galaxy halo or halo of~the~cluster of~galaxies. Therefore the~value of~the~effective screening potential is given by the~density of~the~background object we can consider as being in~infinity relative to~the~scale of~the~studied object. In~\autoref{fig:nfwlike_pot_eff} we show the~value of~this effective screening potential for~a~cluster of~galaxies of~a~typical size -- $M=10^{14} M_\odot, c=4$. We see that the~screening potential is greatly reduced in~inner parts of~the~galaxy cluster halo. For~this reason, we do not think it much likely to~observe the~effects of~the~chameleon field on~scales smaller than Mpc.

\begin{figure*}
	\centering
		\begin{subfigure}{1.0\linewidth}
			\includegraphics[width=1.0\linewidth]{{spherical_cham/nfwlike_pot_eff}.png}
		\end{subfigure}
		\begin{subfigure}{1.0\linewidth}
			\includegraphics[width=1.0\linewidth]{{spherical_cham/nfwlike_pot_eff_n}.png}
		\end{subfigure}
		\caption{Effective screening potential relative to~the~screening potential for~a~cluster of~galaxies, $M=10^{14} M_\odot, c=4$. The~top Figure is shown for~several screening potentials (given through the~equivalence radius) while the~bottom for~different chameleon parameter $n$.}
		\label{fig:nfwlike_pot_eff}
\end{figure*}

As the~chameleon affects only non-relativistic matter it can be detected using a~combination of~dynamical measurements and~lensing measurements. Therefore, we considered what would be the~difference between the~mass distribution of~a~galaxy cluster measured via lensing (true mass) and~via dynamics of~enclosed galaxies. In~\autoref{fig:clustersYs} we plot this ratio for~five real clusters (simulated as having ideal NFW profile), see their parameters in~\autoref{tab:clusters}, and~for~four different values of~the~screening potential. We see that except for~the~case $\Phiscr=1$ one would need very precise (and~nowadays unrealistic) measurements of~the~mass distribution. We are therefore left with~only cosmological scales of~tens of~Mpc and~larger to~study the~chameleon. We will study this case in~\autoref{chpt:app_sims}.
\begin{table}[hbt]
	\centering
	\begin{tabular}{lcc|lcc}
		\hline \hline
		Cluster & $c$ & $M$ & Cluster & $c$ & $M$ \\
		\hline
		ClG 0054-27 & $1.2$ & $0.42\cdot10^{14}$ &
		Cl 0016+1609 & $2.1$ & $1.12\cdot10^{14}$ \\
		MS 2137.3-2353 & $13$ & $2.9\cdot10^{14}$ &
		ClG 2244-02 & $4.3$ & $4.5\cdot10^{14}$ \\
		MS 0451.6-0305 & $5.5$ & $18\cdot10^{14}$ & & & \\
		\hline \hline
	\end{tabular}
	\caption{Properties of~clusters simulated as perfect NFW halos: concentration $c$ and~mass $M [M_\odot]$. Parameters are taken from \textcite{2007MNRAS.379..190C}}
	\label{tab:clusters}
\end{table}

\begin{figure*}[!hbt]
\begin{adjustwidth}{-1cm}{-1cm}
	\centering
		\begin{subfigure}{0.5\linewidth}
			\includegraphics[width=1.0\linewidth]{{spherical_cham/clustersYs_-6}.png}
			\caption{$\Phiscr=10^{-6}$}
		\end{subfigure}%
		\begin{subfigure}{0.5\linewidth}
			\includegraphics[width=1.0\linewidth]{{spherical_cham/clustersYs_-4}.png}
			\caption{$\Phiscr=10^{-4}$}
		\end{subfigure}
		\begin{subfigure}{0.5\linewidth}
			\includegraphics[width=1.0\linewidth]{{spherical_cham/clustersYs_-2}.png}
			\caption{$\Phiscr=10^{-2}$}
		\end{subfigure}%
		\begin{subfigure}{0.5\linewidth}
			\includegraphics[width=1.0\linewidth]{{spherical_cham/clustersYs_0}.png}
			\caption{$\Phiscr=10^{0}$}
		\end{subfigure}
	\end{adjustwidth}
		\caption{Effective dynamical mass of~the~clusters relative to~the~actual (lensing) mass of~the~cluster. Cluster properties are shown in~\autoref{tab:clusters}.}
		\label{fig:clustersYs}
\end{figure*}