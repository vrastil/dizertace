\chapter*{Conclusion}
\addcontentsline{toc}{chapter}{Conclusion}
% Do autoref \url{http://home.ef.jcu.cz/~houda/publications/2009-dissertation-abstract.pdf}

\todo{

Describe each chapter -- what is original work and what is overtaken
}

The thesis was dealing with the topic of modified gravity and how we can discover its resulting deviations from general relativity. In \autoref{chpt:cosmo_evol} we summarized linear equations for the evolution of the Universe. We described how we can use the evolution process to studies of cosmological parameters of the Universe. We described basic cosmological observables and their usefulness when distinguishing between different cosmologies.

\autoref{chpt:de_mg} was focused on modifications of the general relativity. We describe successes and failures of the standard cosmological model and presented main issues with the cosmological constant. We described how we can modify theory of gravity and focused on one of the most studied extensions -- the \fR\ gravity. We studied the Hu-Sawicki \fR\ theory from a different angle of view than most authors -- as the chameleon gravity in the Einstein frame. We studied the chameleon behavior numerically and we checked the linear predictions with our numerical solutions. We studied the chameleon behavior in spherical systems (stars, galaxies, clusters) and concluded that the chameleon mechanism completely hides any fifth force on scales smaller than superclusters and chameleon needs to be studied on large cosmological scales through \nbodysim s. 

In \autoref{chpt:cosmo_sim} we described general techniques when dealing with cosmological \nbodysim s and how we implemented them in our own code for \nbodysim s -- with both standard and modified gravity. We described our own contribution to the \code{CCL} code and how the \code{CCL} can help scientists to speed up their work on cosmology.

In \autoref{chpt:app_schemes} we introduced different approximations that can be used to study the Universe quickly and intuitively. We generalized previously studied case of the Einstein--de Sitter Universe to the \LCDM\ cosmology. We described how we implemented these approximations in studies of modified gravity which has not been done before.

In \autoref{chpt:app_sims} we described results of our cosmological \nbodysim s using approximate schemes. We focused on aspects that have drawn less attention in the past -- a slower growth of structure formation (modified growth function), study of the correlation function, predictions of the shape of the peak of baryonic acoustic oscillations in real space, the ability to predict certain non-linear features of full \nbody s. We also tested these methods on one the chameleon gravity and showed what probes can be used in distances between different parameters of the chameleon gravity.

In \autoref{chpt:cosmo_surveys} we reviewed the present-day and future experiments designed to study our Universe. We showed what constraints on our Universe have already been observed and what accuracy of future experiments we can expect.

\subsubsection{Author`s work}
\todo{Describe List of publication}
