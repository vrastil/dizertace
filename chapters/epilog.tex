\chapter*{Conclusion}
\addcontentsline{toc}{chapter}{Conclusion}
% Do autoref \url{http://home.ef.jcu.cz/~houda/publications/2009-dissertation-abstract.pdf}
The thesis was dealing with the topic of modified gravity and how we can discover its resulting deviations from general relativity. In \autoref{chpt:cosmo_evol} we summarized linear equations for the evolution of the Universe. We described how we can use the evolution process to studies of cosmological parameters of the Universe. We described basic cosmological observables and their usefulness when distinguishing between different cosmologies.

\hyperref[chpt:de_mg]{Chapter 2} was focused on modifications of the general relativity. We described successes and failures of the standard cosmological model and presented main issues with the cosmological constant. We described how we can modify the theory of gravity and focused on one of the most studied extensions -- the \fR\ gravity. We studied the Hu-Sawicki \fR\ theory from a different angle of view than most authors -- as the chameleon gravity in the Einstein frame. We studied the chameleon behavior numerically and we checked the linear predictions with our numerical solutions. We studied the chameleon behavior in spherical systems (stars, galaxies, clusters) and concluded that the chameleon mechanism completely hides any fifth force on scales smaller than superclusters and chameleon needs to be studied on large cosmological scales through \nbodysim s. 

In \autoref{chpt:cosmo_sim} we described general techniques when dealing with cosmological \nbodysim s and how we implemented them in our own code for \nbodysim s -- with both standard and modified gravity. We described our own contribution to the \code{CCL} code and how the \code{CCL} can help scientists to speed up their work on cosmology.

In \autoref{chpt:app_schemes} we introduced different approximations that can be used to study the Universe quickly and intuitively. Most of the previous studies focused on the Einstein--de Sitter Universe wheres in this work we generalized the equations to the \LCDM\ cosmology. We described how we implemented these approximations in studies of modified gravity which has not been done before and is one of the original result of this work.

In \autoref{chpt:app_sims} we described original results of our cosmological \nbodysim s using approximate schemes. We focused on aspects that have drawn less attention in the past -- slower growth of structure formation (modified growth function), study of the correlation function, predictions of the shape of the peak of baryonic acoustic oscillations in real space and the ability to predict certain non-linear features of full \nbodysim s. We showed that these approximations can be used to study scales around the baryon acoustic oscillation scale, $k\sim 0.1~h\text{Mpc}^{-1}$ but not much further. We also tested these methods on the chameleon gravity and showed what probes are suited the best for distinguishing between different parameters of the chameleon gravity. Unlike matter power spectra and baryonic acoustic oscillation, the halo mass function does not seem to be a good way to study the chameleon gravity.

In \autoref{chpt:cosmo_surveys} we reviewed the present-day and future experiments designed to study our Universe. We showed what constraints on our Universe have already been observed and what accuracy of future experiments we can expect.

\subsubsection{Author`s publications}
\begin{refsection}[bibliography/my_work.bib]
In the \hyperref[chpt:list_publish]{List of publications} are listed all author`s publications (in alphabetical order by the authors` last names). The list includes the author`s previous work in the Cherenkov Telescope Array collaboration \parencite{2016arXiv161005151C,2017arXiv170903483A,2017ApJ...840...74A,2019scta.book.....C}. While working on the CTA, the author`s main interest were the atmospheric simulations of cosmic showers \parencite{2017EPJWC.14401014V,}. The purpose of these simulations is to improve the calibration of the atmospheric properties as well as a calibration of the detector response. One of the main contributions to the systematic uncertainties of the CTA measurements stems from the uncertainty on the atmospheric density profile, of molecules and aerosols, which these simulations help to reduce.

The author`s work inside the Dark Energy Science Collaboration (DESC) focused mainly on improving the Core Cosmology Library \parencite[\code{CCL},][]{2019ascl.soft01003C,2019ApJS..242....2C}. Initially, the author helped to create a documentation of the library, wrote many examples of usage of the \code{CCL}, and presented with other authors the \code{CCL} and its capabilities at several sessions to new users. The author also helped to improve several automatization processes regarding the releasing of the library and helped to improve its compatibility across different operating systems and environments.

The author`s main work, \textit{Fast approximate methods for modified gravity cosmological simulations} \parencite[published in Monthly Notices of the Royal Astronomical Society,][]{2020MNRAS.493.2085V}, summarizes results regarding the cosmological \nbodysim s described in this work (\autoref{chpt:cosmo_sim} -- \autoref{chpt:app_sims}). It is the result of several-year research regarding the cosmological simulations. During this time, the author wrote his own code for cosmological simulations, implemented methods for solving highly non-linear equations, and developed complex pipelines for processing and analyzing data of these simulations.
\end{refsection}
