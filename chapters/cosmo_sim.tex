\chapter{Cosmological Simulations}
Cosmic Linear Anisotropy Solving System \parencite[CLASS,][]{class}

For a comparison with standard gravity we also used linear and non-linear power spectra $P(k)$ generated by the CosmicEmu emulator \parencite{Heitmann:2015xma}

Core Cosmology Library \parencite{2018arXiv181205995C}

Gauss-Seidel relaxation \parencite{doi:10.1002/zamm.19720520813}

full approximation scheme \parencite[FAS, see, e.g.,][]{MG_overview}

initial conditions with opposite phases to accelerate convergence \parencite{PhysRevD.93.103519}.

Numerical Recipes \parencite{10.5555/42249}

\todo{Notation \(n,N,L,m\)

}

\section{\nbody\ problem}
Many problems in physics, including cosmology, involve particle systems with each particle interacting with all other particles present. In astronomy there is a gravitational interaction between starts, galaxies and cluster of galaxies, depending on scale which we are studying. The challenge of efficiently carrying out the related calculations is generally known as the \nbody\ problem.

The main problem is following -- we have $n$ particles all interacting gravitationally with each other. To compute a trajectory of even a single particle involves computing trajectories of all other particles as the gravitational force is dependent on time-varying positions of other particles. That means that at each time-step we have to compute force from all other $n$ particles and we need to compute these forces for each of $n$ particles. This leaves us with complexity of \(\OO(n^2)\). This brute force approach can be used only for small systems and is computationally unimaginable for large systems in cosmology which typically involve \(n\sim10^{9}\) particles.

This direct approach is generally referred to as the Particle-Particle (PP) method \parencite{Hockney:1988:CSU:62815}. Although computationally expensive the accuracy in the force calculation is of machine precision. To be able to simulate large systems of particles we need to drop the accuracy of continuous positions and use discrete positions for force calculations. In our code we use two main methods for force calculations -- the Particle-Mesh algorithm (PM) and grid-based methods, both of them we describe in more details below.
\subsection{Time integration}
The accurate time integration is a very important part of any \nbodysim. While there are many different methods to integrate particle trajectories \parencite[see e.g.][]{Hockney:1988:CSU:62815} we describe here one of the most used one in collisionless simulations -- the Leapfrog integrator.

The leapfrog integrator is an example of a symplectic integrator. Symplectic integrators exactly solve an approximate Hamiltonian. As a consequence, the numerical time evolution is a canonical map and preserves certain conserved quantities exactly, such as the total angular momentum, the phase-space volume, and the Jacobi constants. The idea is to approximate the Hamiltonian \(H\) governing motion of particles with an approximat one
\eq{
    \tilde H=H+H\err\,,
}
where \(H\err\) is the error Hamiltonian. Provided that \(\tilde H\) and \(H\) are time-invariant, the energy error is bounded at all times. The goal now is to find \(\tilde H\) that can be solved exactly by simple numerical means and minimises \(H\err\). Defining the combined phase-space coordinates \(w\equiv(x,p)\) the Hamilton’s equations are
\eq{
    \dot w=\HH w\,,
}
where \(\HH\equiv\{\cdot,H\}\) is an operator acting on \(w\) through Poisson bracket. Hamilton’s equations have then the formal solution
\eq{
    w(t+\Delta t)=e^{\Delta t\HH}w(t)\,.
}
\begin{sloppypar}
The operator \(e^{\Delta t\HH}\) can be split, in an approximate sense, into a succession of discrete but symplectic steps, each of which can be exactly integrated. The most common choice is to separate out the kinetic and potential energies, \({H=T(p)+V(x)}\), such that we can split
\end{sloppypar}
\eq{
    \label{eq:kdk}
    e^{\Delta t\HH}= e^{\Delta t(\TT+\VV)}\approx e^{\frac12\Delta t\VV}e^{\Delta t\TT}e^{\frac12\Delta t\VV}\,,
}
where operators \(\TT\equiv\{\cdot,T\}\) and \(\VV\equiv\{\cdot,V\}\) are known as drift and kick, as they only change either the positions (drift) or velocities (kick). Because these operators are non-commutative, the central relation in \eqref{eq:kdk} is only approximately true. This operator splitting is extremely useful, because the new equations have simple solution:
\eq{
    e^{\Delta t\TT}
    \begin{pmatrix}
        x \\
        p \\
    \end{pmatrix}=
    \begin{pmatrix}
        x+\Delta t\ p \\
        p \\
    \end{pmatrix}
    \mspace{20mu}\rm{and}\mspace{20mu}
    e^{\Delta t\VV}
    \begin{pmatrix}
        x \\
        p \\
    \end{pmatrix}=
    \begin{pmatrix}
        x \\
        p-\Delta t\ \nabla V \\
    \end{pmatrix}\,.
}
The splitting \eqref{eq:kdk}, also known as kick-drift-kick (KDK), is second order accurate, whereas simpler splitting only into one kick and one drift is first order accurate. The same accurcy and results has also similar splitting into drift-kick-drift which we use in our integrator.
\section{Particle-Mesh algorithm}
\todo{Continuous and discrete potential}
\subsection{Ewald method}

\subsection{Short-range forces}
\todo{P$^3$M

TreePM

The Fast Multipole Method}

\section{Grid-based methods}
\todo{need to solve non-linear equations

Gaus-Seidel relaxation method

Multigrid methods}

\section{Alternatives to \nbody\ simulations}
\todo{Boltzmann equation, orbit averaging, Monte-Carlo}

\section{Validation}