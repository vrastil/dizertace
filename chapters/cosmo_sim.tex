\chapter{Cosmological Simulations}
Cosmic Linear Anisotropy Solving System \parencite[CLASS,][]{class}

For a comparison with standard gravity we also used linear and non-linear power spectra $P(k)$ generated by the CosmicEmu emulator \parencite{Heitmann:2015xma}

Core Cosmology Library \parencite{2018arXiv181205995C}

Gauss-Seidel relaxation \parencite{doi:10.1002/zamm.19720520813}

full approximation scheme \parencite[FAS, see, e.g.,][]{MG_overview}

initial conditions with opposite phases to accelerate convergence \parencite{PhysRevD.93.103519}.

Numerical Recipes \parencite{10.5555/42249}

\todo{Notation \(n,N,L,m\)

}

\section{\nbody\ problem}
Many problems in physics, including cosmology, involve particle systems with each particle interacting with all other particles present. In astronomy there is a gravitational interaction between starts, galaxies and cluster of galaxies, depending on scale which we are studying. The challenge of efficiently carrying out the related calculations is generally known as the \nbody\ problem.

The main problem is following -- we have $n$ particles all interacting gravitationally with each other. To compute a trajectory of even a single particle involves computing trajectories of all other particles as the gravitational force is dependent on time-varying positions of other particles. That means that at each time-step we have to compute force from all other $n$ particles and we need to compute these forces for each of $n$ particles. This leaves us with complexity of \(\OO(n^2)\). This brute force approach can be used only for small systems and is computationally unimaginable for large systems in cosmology which typically involve \(n\sim10^{9}\) particles.

This direct approach is generally referred to as the Particle-Particle (PP) method \parencite{Hockney:1988:CSU:62815}. Although computationally expensive the accuracy in the force calculation is of machine precision. To be able to simulate large systems of particles we need to drop the accuracy of continuous positions and use discrete positions for force calculations. In our code we use two main methods for force calculations -- the Particle-Mesh algorithm (PM) and grid-based methods, both of them we describe in more details below.
\subsection{Time integration}
\todo{Leapfrog}
\section{Particle-Mesh algorithm}
\todo{Continuous and discrete potential}
\subsection{Ewald method}

\subsection{Short-range forces}
\todo{P$^3$M

TreePM

The Fast Multipole Method}

\section{Grid-based methods}
\todo{need to solve non-linear equations

Gaus-Seidel relaxation method

Multigrid methods}

\section{Alternatives to \nbody\ simulations}
\todo{Boltzmann equation, orbit averaging, Monte-Carlo}

\section{Validation}