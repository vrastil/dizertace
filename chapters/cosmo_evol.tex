\chapter{Cosmological Evolution}
\todo{Check citation -- cite vs parencite vs textcite
}

In this chapter we briefly describe the evolution of the Universe. For more comprehensive lectures see e.g. \cite{Ref:Weinberg}, \cite{2002col.luc..cosmology} or \cite{2010deto.book.....A}. We begin with basic equations for the evolution and apply them to cosmological scales. We describe different epoch in history of the Universe and also introduce number of cosmic distances often used to put observational constraints on dark energy. We show some fundamental issues with the simplest models -- horizon problem, missing (dark) matter and accelerated expansion of the Universe -- and arrive to concordance \LCDM\ model. We describe background evolution and formation of structures in \LCDM\ model and introduce some basic observables. At the end of the chapter we outline some issues with the cosmological constant and offer some basic alternative in modified gravity models.

\section{Background evolution}
\todo{cite cosmological principle}
Standard cosmological model is based on the \textit{cosmological principle}. The cosmological principle states that the Universe is homogeneous and isotropic on sufficiently large scales (beyond those traced by the large-scale structure of the distribution of galaxies). The cosmic microwave background (CMB) observations strongly support this statement as photons from different parts of the Universe are coming with almost identical temperatures. This isotropy, however, does not imply homogeneity. The \textit{Copernican Principle} states that the observer is not in a special place or time. With the observed isotropy the Copernican Principle implies the Cosmological Principle. The observed inhomogeneities and irregularities on local scales come from gravitational instability.
%%%%%%%%%%%%%%%%%%%%%%%
% Friedmann equations
%%%%%%%%%%%%%%%%%%%%%%%
\subsection{Friedmann equations}
The homogeneous and isotropic space-time is described by Friedmann-Lema\^{i}tre-Robertson-Walker (FLRW) metric given by
\eq{
    \label{eq:flrw}
    \dd s^2=g_\uv\dd x^\mu \dd x^\nu=-\dd t^2+a^2(t)\left[\frac{\dd r^2}{1-Kr^2} + r^2\left(\dd\theta^2+\sin^2\theta\dd\phi^2 \right) \right]\,,
}
where $g_{\mu\nu}$ is a metric tensor, $a(t)$ is a scale factor with cosmic time $t$ and $K=+1,-1,0$ is a curvature that corresponds to closed, open, and flat geometries. The scale factor can be arbitrary rescaled and we adopt the common normalization $a_0=1$. We can also transform metric \eqref{eq:flrw} to a more convenient form by setting $r=\sin\chi\ (K=+1),r=\chi\ (K=0)$, and $r=\sinh\chi\ (K=-1)$. The FLRW metric is then given by
\eq{
    \label{eq:flrw_chi}
    \dd s^2=-\dd t^2+a^2(t)\left[\dd \chi^2 + f_K^2(\chi)\left(\dd\theta^2+\sin^2\theta\dd\phi^2 \right)\right]\,,
}
where
\eq{
    f_K(\chi)=
    \begin{cases}
        \sin\chi & (K=+1)\,, \\
        \chi & (K=0)\,, \\
        \sinh\chi & (K=-1)\,.
    \end{cases}
}
By allowing taking the limit $K\to0$ we can rewrite $f_K(\chi)$ in a unified way
\eq{
    \label{eq:f_K}
    f_K(\chi)=\frac{1}{\sqrt{-K}}\sinh{(\sqrt{-K}\chi)}\,.
}
In addition to the cosmic time $t$, we also introduce the conformal time $\eta$ defined by
\eq{
    \eta\equiv\int{a^{-1}\dd t}\,.
}
The dynamical equations of motion in the expanding Universe can be derived from the Einstein equations resulting in the Friedmann equations
\eq{
    \label{eq:Friedmann}
    H^2 &= \frac{8\pi G}{3}\rho - \frac{K}{a^2}+\frac{\Lambda}{3}\,, \\
    \frac{\ddot a}{a} &= -\frac{4\pi G}{3}(\rho + 3p)+\frac{\Lambda}{3}\,,\\
    \label{eq:Friedmann-continuity}
    \dot\rho &= -3H(\rho+p)\,,
}
\todo{comment $\Lambda$}

where $H\equiv\dot a/a$ is the Hubble parameter, $\Lambda$ the cosmological constant, $\rho(t)$ is energy-density and $p$ pressure. We will be also using the conformal Hubble quantity
\eq{
    \mathcal{H}\equiv\frac{1}{a}\frac{\dd a}{\dd\eta}=Ha\,.
}
These three equations are not independent and we need to also specify the equation of state, $\rho=\rho(p,t)$. The first equation is usually also written in the form
\eq{
    \Omega_m+\Omega_\gamma+\Omega_K+\Omega_\Lambda=1\,,
}
where
\eq{
    \Omega_m\equiv\frac{8\pi G\rho_m}{3H^2}\,,\hspace{3pt}
    \Omega_\gamma\equiv\frac{8\pi G\rho_\gamma}{3H^2}\,,\hspace{3pt}
    \Omega_K\equiv-\frac{K}{a^2H^2}\,,\hspace{3pt}
    \Omega_\Lambda\equiv\frac{\Lambda}{3H^2}
}

Most matter species in the Universe can be described by a simple equation of state
\eq{
    p=w\rho\,,
}
where $w$ is constant. This lead to the following solutions in flat universe
\eq{
    \rho\propto a^{-3(1+w)}\,,\hspace{6pt}a\propto (t-t_i)^{2/(3(1+w)))}\,,
}
where $t_i$ is an integration constant. Relativistic matter has the equation of state $w=1/3$ and the cosmic evolution during the radiation-dominated epoch is given by $w\propto a^{-4}$ and $a\propto(t-t_i)^{1/2}$. Non-relativistic matter has negligible pressure $(w=0)$ and the evolution during the matter- dominated era is given by $w\propto a^{-3}$ and $a\propto(t-t_i)^{2/3}$.

The accelerated expansion of the Universe $(\ddot a>0)$ is possible only for $w<-1/3$ (see Friedman equation \eqref{eq:Friedmann}), i.e. negative pressure. Note that in Newtonian gravity there is no such equivalent. The pressure is related to a force associated with a local potential that depends on the position in space but in homogeneous and isotropic universe there is no such potential.

When $w=-1$, the energy-density $\rho$ is constant and corresponds to the cosmological constant. This energy-density connected to the cosmological constant is
\eq{
    \rho_\Lambda=\frac{\Lambda}{8\pi G}\,.
}
With constant energy-density the Friedmann equations lead to an exponential expansion $a\propto\exp{Ht}$.

%%%%%%%%%%%%%%%%%%%%%%%
% Hubble`s law
%%%%%%%%%%%%%%%%%%%%%%%
\subsection{Hubble`s law}
During the 1920s astronomers Slipher and Hubble found that the observed wavelength $\lambda_o$ of absorption lines of distant galaxies is larger than the wavelength $\lambda$ in the rest frame \cite{1929PNAS...15..168H}. This is due to the fact that the wavelength is stretched in proportion to the scale factor in an expanding Universe. The redshift $z$ is defined as
\eq{
    z\equiv\frac{\lambda_0}{\lambda} - 1=a^{-1}-1\,.
}
In an expanding Universe a physical distance $r$ from an observer (at the origin) to an object is given by $r=a(t)x$, where $x$ denotes the comoving distance. For objects which are motionless with respect to the Hubble flow the comoving distance remains constant. Taking the time derivative of $r$ we obtain
\eq{
    v_H\equiv\dot r=Hr+a\dot x
}
Because of the cosmic expansion, more distant objects are moving faster from us with velocity $v_H$. On the other hand, the peculiar velocity $v_p\equiv a\dot x$ describes the movement of an object with respect to the local Hubble flow. For small peculiar velocities and near objects $(z\ll1)$ we obtain
\eq{
    v\sim H_0r\,,
}
which is the law Hubble reported in 1929 by plotting the recessional velocity $v$ versus the distance $r$. His measurements were noisy but the current measurements give the Hubble constant the value \cite{planck_cosm}.
\eq{
    H_0=(67.4\pm0.5)\rm{\ km\ s}^{-1}\rm{Mpc}^{-1}
}

%%%%%%%%%%%%%%%%%%%%%%%
% Cosmic distances
%%%%%%%%%%%%%%%%%%%%%%%
\subsection{Cosmic distances}
In the expanding Universe there are many ways to specify the distance between two points due to the constantly changing distances and the fact that observers look back in time as they look out in distance.
\subsubsection{Comoving distance}
The light traveling along the $\chi$ direction satisfies the geodesic equation $\dd s^2=-\dd t^2+a^2(t)\dd \chi^2=0$. The light emitted at the time $t=t_1$ with $\chi=\chi_1$ (redshift $z$) reaches an observer at time$t=t_0$ with $\chi=0$ (redshift $z=0$). Integrating the geodesic equation
\eq{
    \label{eq:d_c}
    d_c\equiv\chi_1=\int_0^{\chi_1}\dd\chi=-\int_{t_0}^{t_1}\frac{\dd t}{a(t)}=\frac{1}{H_0}\int_0^z\frac{\dd\tilde z}{E(\tilde z)}\,,
}
where
\eq{
    E(z)\equiv H(z)/H_0=\sqrt{\Omega_{\gamma,0}(z+1)^4+\Omega_{m,0}(z+1)^3+\Omega_{K,0}(z+1)^2+\Omega_{\Lambda,0}}\,.
}
If we expand $E(z)$ around $z=0$ we can write the cosmic distance as
\eq{
    d_c=\frac{1}{H_0}z-\frac{E'(0)}{2H_0}z^2+\frac{2E'(0)^2-E''(0)}{6H_0}z^3+\mathcal{O}(z^4)\,,
}
where a prime represents a derivative with respect to $z$.
\subsubsection{Luminosity distance}
The luminosity distance $d_L$ is used in supernovae observations in order to link the supernova luminosity with the expansion rate of the Universe. It is defined by
\eq{
    d_L^2\equiv\frac{L_s}{4\pi F}\,,
}
where $L_s$ is the absolute bolometric (i.e., integrated over all frequencies) luminosity of a source and $F$ is the observed bolometric flux. The flux is defined by $F=L_0/S$, where $L_0$ is the observed luminosity and $S=4\pi f_K^2(\chi)$ is the area of a sphere at $z=0$.

The absolute luminosity is defined as energy emitted per unit time interval, $L=\Delta E/\Delta t$. The energy of a photon is inversely proportional to its wavelength, $E\propto\lambda^{-1}\propto 1+z$, which is stretching in an expanding universe. Also the time between arrival of two photons is proportional to the wavelength, $\Delta t\propto\lambda\propto(1+z)^{-1}$. The ratio $L_s/L_0$ is then
\eq{
    \frac{L_s}{L_0}=\frac{\Delta E_1}{\Delta E_0}\frac{\Delta t_0}{\Delta t_1}=(1+z)^2\,,
}
and the luminosity distance is
\eq{
    d_L=f_K(\chi)(1+z)\,.
}
Using definition \(f_K\) \eqref{eq:f_K} and comoving distance \eqref{eq:d_c} we can express $d_L$ as
\eq{
    \label{eq:luminosity}
    d_L=\frac{1+z}{H_0\sqrt{\Omega_{K,0}}}\sinh{\left(\sqrt{\Omega_{K,0}}\int_0^z{\frac{\dd\tilde z}{E(\tilde z)}}\right)}\,.
}
For small $z$ we can once again expand the expression and get
\eq{
    d_L=\frac{1}{H_0}z-\frac{E'(0)-2}{2H_0}z^2+\frac{2E'(0)^2-3E'(0)-E''(0)+\Omega_{K,0}}{6H_0}z^3+\mathcal{O}(z^4)\,.
}
\subsubsection{Angular diameter distance}
The angular diameter distance $d_A$ is defined as
\eq{
    d_A\equiv\frac{\Delta x}{\Delta\theta}\,,
}
where $\Delta\theta$ is the angle that subtends an object of actual size$\Delta x$ orthogonal to the line of sight. Whenever we look at objects of known size such as CMB anisotropies or BAO scale we use this distance.

The observer measures the size $\Delta x$ along the surface of a sphere with radius $\chi$ and from metric \eqref{eq:flrw_chi} follows
\eq{
    \Delta x=a(t)f_K(\chi)\Delta\theta\,.
}
The angular diameter distance is then
\eq{
    \label{eq:angular}
    d_A=a(t)f_K(\chi)=\frac{1}{1+z}\frac{1}{H_0\sqrt{\Omega_{K,0}}}\sinh{\left(\sqrt{\Omega_{K,0}}\int_0^z{\frac{\dd\tilde z}{E(\tilde z)}}\right)}\,.
}
Comparing angular diameter distance \eqref{eq:angular} and luminosity distance \eqref{eq:luminosity} we can see that they have a following relation
\eq{
    d_A=\frac{d_L}{(1+z)^2}\,.
}
\subsubsection{Degeneracy of the distance measurements}
We can se that up to the first order all the distances are the same and reduce to the Euclidean distance and that the Hubble`s law holds. With the increasing redshift the Hubble`s law dos not hold exactly and also different distances behave differently. This can be used to measure different properties of the Universe.

As the distances depend on the cosmological parameters through the integral $\int_0^z{\dd\tilde z/E(\tilde z)}$ and through $\Omega_K$ we can measure only those parameters contained in $E(z)$. Moreover, if we had distance measurements only around one particular redshift $z$ any combination of parameters that would produce similar $E(z)$ would be equally acceptable. This degeneracy can be broken by combination of measurements across different redshifts or by using other cosmological probes.

%%%%%%%%%%%%%%%%%%%%%%%
% Formation and evolution of LSS
%%%%%%%%%%%%%%%%%%%%%%%
\section{Formation and evolution of LSS}
So far we have considered only equations for smooth, homogeneous and isotropic background. In order to describe the real Universe with its rich structures we must employ perturbation theory for cosmological equations.
\subsection{Newtonian gauge}
We will consider first order (linear) perturbations of the metric:
\eq{
    g_{\mu\nu}=g_{\mu\nu}^{(0)}+\delta g_{\mu\nu}\,,
}
where $g_{\mu\nu}^{(0)}$ is the FLRW metric \eqref{eq:flrw} and all the entries in the perturbed metric $\delta g_{\mu\nu}$ have to be small with respect to the background metric. The general perturbed metric can be written as
\eq{
    \delta g_{\mu\nu}=a^2(\eta)
    \begin{pmatrix}
        -2\Psi & w_i \\
        w_i & 2\Phi\delta_{ij}+h_{ij} \\
    \end{pmatrix}\,,
}
where $\Psi(t,x)$ and $\Phi(t,x)$ are spatial scalars, $w_i(t,x)$ is 3-vector, and $h_{ij}(t,x)$ is a traceless 3-tensor. We have a freedom to choose coordinates in which we will describe the equations of motion -- we can choose our \textit{gauge}. We wish to keep our background metric $g_{\mu\nu}^{(0)}$ the same (i.e. FLRW) and only change $\delta g_{\mu\nu}$. Note that unlike ordinary coordinate transformations a gauge transformation (although expressed as a change of coordinates) does not link different observers in the same spacetime but it links two different spacetimes seen by the same observer. For a detailed discussion of gauge choices, see e.g. \cite{PhysRevD.40.1804,10.1143/PTPS.78.1,PhysRevD.22.1882}.

We will choose to attach the observers to the points in the unperturbed frame, the so called \textit{Newtonian} or \textit{longitudinal} or \textit{shear-free} gauge. In this case the observer will detect a velocity field of particles falling into the clumps of matter and will measure a gravitational potential. We can impose up to four conditions on the metric, which corresponds to the four gauge coordinate transformations. The final perturbed metric is:
\eq{
    \label{eq:flrw_pert}
    \dd s^2=a^2(\eta)\left[-(1+2\Psi)\dd\eta^2+(1+2\Phi)\delta_{ij}\dd x^i\dd x^j\right]\,.
}
\subsection{Linear perturbations}
We now decompose the Einstein tensor and the energy-momentum tensor into background and perturbed parts. The background cosmological evolution is obtained by solving the zero-th order Einstein equations and is given by the Friedmann equations. The first order equations are given by the perturbation of the Einstein tensor, i.e. by perturbed metric \eqref{eq:flrw_pert}, and by the perturbed energy-momentum tensor $T_\uv$. The energy-momentum tensor for a single perfect fluid is given by
\eq{
    T_\uv=\left(\rho+p\right)u_\mu u_\nu + pg_\uv\,,
}
where the four-velocity $u^\mu$ is up to the first order
\eq{
    u^\mu\equiv\dddd{x^\mu}{\tau}=\left[\frac{1}{a}(1-\Psi),\frac{v^i}{a} \right]\,,
}
where $\tau$ is the proper time and $v^i=\dd x^i/\dd\eta$ is the matter peculiar velocity with respect to the general expansion. We also assume that the perturbed fluid remains perfect fluid. We will use the following notation
\eq{
    \delta &\equiv\frac{\delta\rho}{\bar{\rho}}\equiv\frac{\rho-\bar{\rho}}{\bar{\rho}}\,,\\
    \theta &\equiv\nabla\cdot v\,,
}
where $\delta$ is the density contrast, the bar represents a mean value (spatial average) and $\theta$ is the velocity divergence. The first order equations are (for details, see e.g. \cite{2002col.luc..cosmology,10.1143/PTPS.78.1})
\eq{
    \label{eq:lin_1}
    3\mathcal{H}(\mathcal{H}\Psi-\Phi') + \nabla^2\Phi &= -4\pi G\bar\rho a^2 \delta\,,\\
    \label{eq:lin_2}
    \nabla^2(\Phi' - \mathcal{H}\Psi) &= 4\pi G\bar\rho a^2(1+w)\theta\,,\\
    \label{eq:lin_3}
    \Psi &=-\Phi\,,\\
    \label{eq:lin_4}
    \Phi''+2\mathcal{H}\Phi'-\mathcal{H}\Psi-(\mathcal{H}^2+2\mathcal{H}')\Psi &=-4\pi G\bar\rho a^2c^2_s\delta\,,
}
where the prime now denotes the derivative with respect to the conformal time $\eta$. Sound velocity $c_s$ is defined as
\eq{
    c_s^2\equiv\frac{\delta p}{\delta\rho}=\dddd p\rho = \frac{\dot p}{\dot\rho}\,,
}
where the last equality are valid only in the FLRW metric at background level. Conservation of the energy-momentum tensor leads to the perturbed continuity equation
\eq{
    \delta'+3\mathcal{H}(c_s^2-w)\delta=-(1+w)(\theta+3\Psi')\,,
    \label{eq:lin_5}
}
which reduces for non-relativistic matter to
\eq{
    \delta'=-\theta-3\Phi'\,.
}
Another equation coming from the conservation of the energy-momentum tensor is
\eq{
    \theta'+\left[\mathcal{H}(1-3w)+\frac{w'}{1+w}\right]\theta=-\nabla^2\left(\frac{c_s^2}{1+w}\delta + \Psi \right)\,,
    \label{eq:lin_6}
}
which reduces for non-relativistic matter to
\eq{
    \theta'+\mathcal{H}\theta=-\nabla^2\Psi-\nabla^2(c_s^2\delta)\,,
}
which is a relativistic analog of the Euler equation.


We will now transform all the equations to the Fourier space. The Fourier transformation is (up to non-relevant pre-factors)
\eq{
    f(\bm{x})=\int e^{i\bm{k\cdot x}}\hat{f}(\bm k)\dd^3k\,,
}
where $\bm k$ is a wavenumber with modulus $k$ and hat represents quantities in Fourier space. If there is no danger of misunderstanding we will drop the hat in the following. Since we are dealing with linear equations the different modes are not coupled and we can solve the equation for each $k$ independently.
 From equations \eqref{eq:lin_1}---\eqref{eq:lin_4}, \eqref{eq:lin_5} and \eqref{eq:lin_6} we will obtain the following equations in Fourier space:
\eq{
    \label{eq:lin_k_1}
    k^2\Phi + 3\mathcal{H}(\Phi'-\mathcal{H}\Psi) &= 4\pi G\bar\rho a^2 \delta\,,\\
    \label{eq:lin_k_2}
    k^2(\Phi' - \mathcal{H}\Psi) &= -4\pi G\bar\rho a^2(1+w)\theta\,,\\
    \label{eq:lin_k_3}
    \Psi &=-\Phi\,,\\
    \label{eq:lin_k_4}
    \Phi''+2\mathcal{H}\Phi'-\mathcal{H}\Psi-(\mathcal{H}^2+2\mathcal{H}')\Psi &=-4\pi G\bar\rho a^2c^2_s\delta\,,\\
    \label{eq:lin_k_5}
    \delta'+3\mathcal{H}(c_s^2-w)\delta &= -(1+w)(\theta+3\Psi')\,,\\
    \label{eq:lin_k_6}
    \theta'+\left[\mathcal{H}(1-3w)+\frac{w'}{1+w}\right]\theta &= k^2\left(\frac{c_s^2}{1+w}\delta + \Psi \right)\,,
}
where now $\theta=i\bm{k\cdot  v}$. Note that these six equations are not independent and $w$ and $c_s^2$ are arbitrary functions of time.
\subsubsection{Super-horizon scales}
In the large-scale limit $k\ll\mathcal{H}$, i.e. when the physical wavelength $\lambda_p=2\pi a/k$ of perturbations is much larger than the Hubble radius $H^{-1}$, and with the barotropic fluid, i.e. when the pressure depends only on the energy density, and with the constant $w$, we have $c_s^2=w$ and we can get an equation for $\Phi$
\eq{
    \Phi''+3\mathcal{H}(1+c_s^2)\Phi'=0\,.
}
The growing / dominant solution for $c_s^2>-1$ is $\Phi=\rm{const}$ from which follows
\eq{
    3\mathcal{H}^2\Phi=4\pi G\bar\rho a^2\delta\,.
}
Using the Friedmann equation we get $\delta=\Phi$, i.e. $\delta$ remains constant at large scales whenever $c_s^2=w$.
\subsubsection{Sub-horizon scales}
In the small-scale limit $k\gg\mathcal{H}$ we can get for a pressureless fluid $(w=0)$ with a small sound speed $c_s^2\ll1$ the Poisson equation
\eq{
    k^2\Phi=4\pi G\bar\rho a^2\delta\,,
}
and the energy conservation equation
\eq{
    \delta' &= \theta \\
    \theta' &= -\mathcal{H}\theta + c_s^2k^2\delta-k^2\Phi\,.
}
These equations lead to
\eq{
    \delta'' +\mathcal{H}\delta' +\left(c_s^2k^2 + \frac32\mathcal{H}^2 \right)\delta=0\,.
}
Perturbations on small scales with $(c_s^2k^2-\frac32\mathcal{H})>0$ do not grow but undergo damped oscillations. These are perturbations with physcial wavelength smaller than the Jeans length
\eq{
    \lambda_J=c_s\sqrt\frac{\pi}{G\bar\rho}\,.
}
For the photons and baryons before the decoupling epoch we have $c_S=c/sqrt3$ and the Jeans length is comparable to the Hubble radius $H^{-1}$ and the growth of perturbations is prevented on all scales smaller than that. For a small sound speed $c_sl\ll\mathcal{H}$ the perturbations grow freely (gravitational instability). The resulting equation for a single pressureless fluid becomes
\eq{
    \label{eq:lin_evol_m}
    \delta''+\mathcal{H}\delta'-\frac32\mathcal{H}^2\delta=0\,.
}
The growing solution during the matter dominated era is $\delta\propto a\propto t^{2/3}$ and the gravitational potential remains constant during this epoch.

%%%%%%%%%%%%%%%%%%%%%%%
% Growth function
%%%%%%%%%%%%%%%%%%%%%%%
\subsubsection{Growth function and rate}
In linear perturbation theory the gravitational evolution of fluctuations is separable into time-dependent and spatially-dependent (or wavenumber-dependent) parts, i.e.
\eq{
    \delta(a)=\delta_0D(a)\,,
}
where $\delta_0$ is present overdensity and the growth function $D$ is normalized as $D(a=1)\equiv1$. The logarithmic change of growth function is the growth rate $f$
\eq{
    \label{eq:grw_rate}
    f\equiv\dddd{\ln D}{\ln a}\,.
}
 A very good approximation of a solution to density evolution \eqref{eq:lin_evol_m} is
\eq{
    f\approx|\Omega_m(a)|^\gamma\,,
}
where $\gamma\approx0.55-0.6$ depends only weakly on cosmological parameters \parencite{1980_Peebles}.

%%%%%%%%%%%%%%%%%%%%%%%
% Photon propagation
%%%%%%%%%%%%%%%%%%%%%%%
\subsection{Photon propagation}
The above equations describe the evolution of matter density perturbations. Now we will focus on the propagation of light in this perturbed Universe. The photon momentum is defined via $k^\mu=\dd x^\mu/\dd\lambda_s$, where $\lambda_s$ is an affine parameter. The equations for the propagation are then
\eq{
    k_\mu k^\mu &= 0\,,\\
    \label{eq:p_geo}
    \dddd{k^\mu}{\lambda_s}+\Gamma^\mu_{\alpha\beta}k^\alpha k^\beta &= 0\,.
}
We will split the momentum vector into a background and a perturbed value $k^\mu=\bar k^\mu+\delta k^\mu$. The photon moves in the unperturbed metric along the direction $r$ so that $\dd\eta=\dd r$. At the background level we can integrate the photon path \eqref{eq:p_geo} to get $\bar k^0\propto a^{-2}$ and therefore the photon frequency $\nu\equiv\dd t/\dd\Lambda_s=a\bar k^0\propto a^{-1}$ as expected. The equations for the perturbed part $\delta k$ are (for details, see e.g. \cite{2010deto.book.....A})
\eq{
    \label{eq:SW_0}
    \dddd{(\delta k^0/k^0)}{\eta} &= -\left(\partpart{\Phi}{\eta}+\partpart{\Psi}{\eta}+2\partpart{\Psi}{r}\right)\,,\\
    \label{eq:WL_0}
    \frac{\dd^2x^i}{\dd\lambda_s^2}+2\mathcal{H}\dddd{\eta}{\lambda_s}\dddd{x^i}{\lambda_s} &= \left(\partpart{\Phi}{x^i}-\partpart{\Psi}{x^i}\right)\,,
}
where the two directions $x^1$ and $x^2$ are orthogonal to the propagation direction $r$.
\subsubsection{The Sachs--Wolfe effect}
The time part of the geodesic equations \eqref{eq:SW_0} describes changes in the frequency (redshift) of photons passing through changing gravitational potentials. Integrating the equation along the light-ray null path from the time of emission $e$ to the time of observation $o$ gives us
\eq{
    \left.\frac{\delta k^0}{k^0}\right\rvert_e^o=-2\left.\Psi\right\rvert_e^o-\int_{e}^{o}{\left(\partpart\Phi\eta-\partpart\Psi\eta\right)\dd\eta}\,,
}
where vertical bars represents change of quantities between $e$ and $o$. The first effect is referred to as the \textit{Sachs--Wolfe effect} and depends on the difference between the potential at emission and at observation. The second effect is called the \textit{integrated Sachs--Wolfe effect} and depends on the line-of-sight integral of $\Phi-\Psi$.
\subsubsection{Weak lensing}
The spatial part of the geodesic equations \eqref{eq:WL_0} can be written for $i=1,2$ as
\eq{
    \frac{\dd^2x^i}{\dd r^2}=\partpart{\psi}{x^i}\,,
}
where the lensing potential $\psi\equiv\Phi-\Psi$. For small displacements we can put $x^i=r\theta^i$. Integrated the above equation gives us
\eq{
    \theta^i=\theta^i_0+\int_0^r\dd r'\left(1-\frac{r'}{r}\right)\partpart{\psi(r'\theta_0^1,r'\theta_0^2,r')}{x^i}\,.
}
Two light rays separated by a small angle  $\Delta\theta^i$ on the source plane at $r=r_s$ are connected to the the observation plane at $r=0$ by the symmetric transformation matrix
\eq{
    A_{ij}\equiv\partpart{\theta^i_s}{\theta^j_o}=\delta_{ij}+D_{ij}\,,
}
where the distortion tensor
\eq{
    D_{ij}=\int_0^{r_s}\dd r'\left(1-\frac{r'}{r_s}\right)r'\frac{\partial^2\psi}{\partial x^i\partial x^j}=
    \begin{pmatrix}
        -\kappa-\gamma_1 & -\gamma_2 \\
        -\gamma_2 & -\kappa + \gamma_1
    \end{pmatrix}
    \,.
}
The convergence $\kappa$ describes the magnification of the source image and the two components of the shear field $\gamma_1,\gamma_2$ describe the distortion of the source image.
\todo{
\subsection{Non-linear cosmological perturbations}
Briefly describe 2d order, higher corrections, spherical collapse. Mention not main part -- we are interested in linear approximations.
}

%%%%%%%%%%%%%%%%%%%%%%%
% Cosmological observables
%%%%%%%%%%%%%%%%%%%%%%%
%%%%%%%%%%%%%%%%%%%%%%%
% Cosmological observables
%%%%%%%%%%%%%%%%%%%%%%%
\section{Cosmological observables}
Here we present a brief description of cosmological probes which can be used for constraining dark energy. For more details see e.g. \textcite{weinberg_observational_2013} or \textcite{DE_probes2}.
%In shear-shear tomography, the shear of background galaxies, binned in redshift, is cross correlated. 
%\ssec{Introduction}
%http://terapix.phys.cmu.edu/misc/desc\_feb\_2015\_de\_school\_ShirleyHo.pdf \\
%https://sites.google.com/site/cmucosmos13/lecture-notes \\
%http://arxiv.org/pdf/1201.2434v2.pdf \\
%%%%%%%%%%%%%%%%%%%%%%%
% Supernovae
%%%%%%%%%%%%%%%%%%%%%%%
\subsection{Supernovae}
Supernovae (SNe) are the most straightforward tool for studying cosmic acceleration, as they directly discovered the acceleration in the first place \parencite{riess}. Type Ia supernovae (SNe Ia) are exploding stars defined by the lack of hydrogen and the presence of silicon in their early-time spectra \parencite{SN}, and are a product of a thermonuclear explosion of a C/O white dwarf. Observations show that SNe Ia have a luminosity peak that is tightly correlated with the shape of their light curves -- supernovae that rise and fall more slowly have higher peak luminosity (first quantified by \cite{SN_lum}). From observations of (multiband) light curve shapes and colors the luminosity at a brightness peak can be predicted.

To measure cosmic expansion with Type Ia SNe, the observed flux and predicted luminosity are compared. From that the supernova's luminosity distance can be measured. An accurate redshift is obtained by measuring the host galaxy (calibrator). Since the distances to the local calibrators are usually determined from the Hubble expansion, this method gives the luminosity distance $D_L$ in units of $h\mins$ Mpc. Measured relation is used to constraint dark energy parameters.

Recent analysis of \textcite{Abbott_2019} uses data from the Dark Energy Survey Supernova Program (DES-SN) -- 207 spectroscopically confirmed SNe Ia from the first three years of DES-SN combined with a low-redshift sample of 122 SNe from the literature. For a flat $\omega$CDM\ model they found a matter density $\Omega_m=0.321\pm0.018$ and an equation of state $\omega=-0.978\pm0.059$, see \autoref{fig:des_sne_results}.

\begin{figure}[hbt]
    \centering
    \includegraphics[width=0.7\textwidth]{cosmo_evol/DES_SNe_results.png}
    \caption{Hubble diagram for the DES-SN3YR sample. Top: distance modulus $(\mu =5\log [d_L/10\rm{pc}])$ from ``BEAMS with Bias Corrections'' fit (\cite{Kessler_2017}, black bars, which are used for cosmology fits) and for each SN (red, orange circles). The dashed gray line shows best fit model, while the green and blue dotted lines show models with no dark energy and matter densities $\Omega_m = 0.3$ and $1.0$ respectively. Bottom: residuals to the best fit model; $1\sigma$ error bars show 68\% confidence. \textit{Note:} Reprinted from \textcite{Abbott_2019}.}
    \label{fig:des_sne_results}
\end{figure}
%%%%%%%%%%%%%%%%%%%%%%%
% BAO
%%%%%%%%%%%%%%%%%%%%%%%
\subsection{Baryonic acoustic oscillations}
Baryonic acoustic oscillations (BAO) provide an entirely independent way of measuring cosmic distance. Sound waves propagating before recombination imprint a characteristic scale on matter clustering. The acoustic length scale can be computed as
\eq{
r_s=\int_0^{t_\ast}{\frac{c_s(t)}{a(t)}\dd t}=\int_{z_\ast}^{\infty}{\frac{c_s(z)}{H(z)}\dd z},
}
where asterisk denotes time (redshift) at recombination and $c_s$ is the sound speed. The behavior of $H(z)$ depends on the ratio of the matter density to radiation density and the sound speed depends on the ratio of radiation pressure to the energy density of the baryon-photon fluid, determined by the baryon-to-photon ratio. Both the matter-to-radiation ratio and the baryon-to-photon ratio can be measured from the CMB anisotropy power spectrum. This gives $r_s\sim150$ Mpc. The scale of the acoustic feature is stable to better than 1\% accuracy, making it an excellent standard ruler.

This effect can be detected in the angular clustering of galaxies in bins of photometric redshift, yielding the angular diameter distance. Furthermore, measuring the BAO scale in a velocity separation allows a direct determination of $H(z)$. The BAO method measures $D(z)$ in absolute units -- Mpc not $h\mins$ Mpc like SNe measurements, and thus BAO measurements to the same redshift carry different information. At low redshift $(z\lesssim0.5)$, the BAO method strongly complements SN measurements, while at higher redshift $(z\gtrsim0.5)$ the BAO method is an powerful probe of dark energy and cosmic geometry.

BAO can be clearly seen in the correlation function (see e.g. \cite{1993ApJ...412...64L} for definition and estimators). The two-point correlation function $\xi(r)$ is the excess probability $\dd P$ of finding two pairs of galaxies in two volumes $\dd V_1$ and $\dd V_2$ at a given comoving distance $r$
\eq{
    \label{eq:corr}
    \dd P = \bar{n}^2[1+\xi(r)]\dd V_1 \dd V_2\,,
}
where $\bar{n}$ is the expected density of the distribution. The distances are usually measured using the redshift in the redshift space where the distortions (see \autoref{sec:rsd}) cause the redshift-space correlation $\xi(s)$ function to vary according to the angle between the separation vector and the line of sight. In \autoref{fig:xi_s} we see the spherically averaged correlation function from \textcite{2005ApJ...633..560E} with the clear bump at $100h^{-1}$ Mpc scale.
\begin{figure}[hbt]
    \centering
    \includegraphics[width=0.7\textwidth]{cosmo_evol/bao_rsd.png}
    \caption{The large-scale redshift-space correlation function of the SDSS LRG sample. The error bars are from the diagonal elements of the mock-catalog covariance matrix; however, the points are correlated. Note that the vertical axis mixes logarithmic and linear scalings. The inset shows an expanded view with a linear vertical axis. The models are $\Omega_mh^2=0.12$ (top, green), $0.13$ (red), and $0.14$ (bottom with peak, blue), all with $\Omega_bh^2=0.024$ and $n=0.89$ and with a mild non-linear prescription folded in. The magenta line shows a pure CDM model ($\Omega_mh^2=0.105$), which lacks the acoustic peak. It is interesting to note that although the data appears higher than the models, the covariance between the points is soft as regards overall shifts in $\xi(s)$. Subtracting $0.002$ from $\xi(s)$ at all scales makes the plot look cosmetically perfect, but changes the best-fit $\chi^2$ by only $1.3$. The bump at$100h^{-1}$ Mpc scale, on the other hand, is statistically significant. \textit{Note:} Reprinted from \textcite{2005ApJ...633..560E}.}
    \label{fig:xi_s}
\end{figure}

In \textcite{BAO_results} present a measurement of the BAO from the three-dimensional correlation of Lyman-$\alpha$ forest absorption and quasars from the SDSS Data Release 14 (the first two years of observations by the eBOSS) at redshift $z=2.35$. The position of the BAO peak is used to determine the Hubble distance $d_H$ and the comoving angular diameter distance $d_A$ relative to the sound horizon at the drag epoch $r_s: d_H(z=2.35)/r_s = 9.20\pm0.36$ and $d_M(z=2.35)/r_s = 36.3\pm1.8$. These results are $1.5\sigma$ from the flat \LCDM\ model of \textcite{2016A&A...594A..13P}.
%%%%%%%%%%%%%%%%%%%%%%%
% WL
%%%%%%%%%%%%%%%%%%%%%%%
\subsection{Weak lensing}
Gravitational lensing is the deflection of light from distant sources due to the bending of space-time by baryonic and dark matter (lenses) along the line of sight. It is a very useful cosmological probe because it is sensitive to all matter regardless of its nature. In the limit of very small deflection angles it is called weak lensing (WL). WL causes tiny distortions ($\sim0.5\%$), or ``shear'', in galaxy sizes and shapes -- see \autoref{fig_WL_Abell}. Intrinsic size or shape of a given galaxy are unknown, but normally, galaxy orientations are assumed to be random ($\sim30\%$ dispersion), so they should not exhibit statistically significant and coherent alignments. In the presence of lensing, small but coherent shears in background galaxy images are induced. This means that WL is statistically detectable by averaging shapes over many lensed galaxies. In principle either the shearing of galaxies (shape distortion) or their magnification (size distortion) can be measured. However, in practice the shape distortions is used much more widely, since the scatter in shapes of galaxies is less than the scatter in their sizes.

% Weak gravitational lensing bends the trajectory of the light according to \eqref{eq_kin}. The angular deflection integrated over the past light cone between $z$ and $z+\delta z$ is
% \eq{
%     \delta\vec{\alpha}=-\vec{\nabla}_\bot(\Phi+\Psi)\dddd\eta z \delta z,
% }
% where $\alpha$ is a (two-dimensional) vector perpendicular to the trajectory of the light. The gravitational lens equation \cite{WL_over}
% \beq
% \vec\beta=\vec\theta-\frac{r_{LS}}{r_S}\delta\vec\alpha,
% \eeq
% where $\theta$ is the observed direction of a ray and $\beta$ the direction to the source in the absence of the deflection. The distances are are the comoving angular distances from the lens plane to the source plane, and from the observer to the source plane respectively.

% We cannot observe the absolute deflection since we do not know the the true direction of the source. However, the gradient of the deflection results in an image magnification and shear that are observable. These are described by the inverse magnification matrix
% \beq
% \textbf{M}\mins(\vec\theta)=\partpart{\vec\beta}{\vec\theta}=\textbf{1}+\int_0^{z_S}\frac{r_Sr_{LS}}{r_S}\vec{\nabla}_\bot\vec{\nabla}_\bot(\Phi+\Psi)\dddd\eta z \dd z.
% \eeq
% This magnification matrix is parametrized by a convergence $\kappa$ and shear parameters $\gamma_1$ and $\gamma_2$
% \beq
% \begin{split}
% \textbf{M}(\vec\theta)= \left(
% \begin{array}{cc}
% 1+\kappa+\gamma_1 & \gamma_2 \\
% \gamma_2 & 1+\kappa-\gamma_1 \\
% \end{array}
% \right).
% \end{split}
% \eeq
% The weak lensing limit assumes $\kappa^2\ll1$ and $\gamma^2_i\ll1$. The convergence still cannot be measured directly as the size and flux of the source is generally unknown. However, the shear components can be measured statistically averaging galaxy image distortions over large sample of galaxies since the orientation of galaxies are assumed to be random. Weak lensing thus offers a probe of the sum of the two potentials $(\Phi+\Psi)$.
% \ssec{Current Limits}
% Using the Shapiro time delay one can obtain strong limits on the \textit{gravitational slip} (combination $\Psi-\Phi$) in the solar system. The round-trip time of signals from the Cassini spacecraft near Saturn as the signals pass close to the sun is longer than one would expect in flat spacetime. This gives tight limit on the slip \cite{test_pot_1}
% \beq
% \frac{|\Psi-\Phi|}{\Phi}=(2.1\pm2.3)\cdot\EXP{-5}
% \eeq
% within the solar system. Combining stellar dynamics with the strong gravitational lensing on galaxy scales limit the slip \cite{test_pot_2}
% \beq
% \frac{|\Psi-\Phi|}{\Phi}=0.01\pm0.05
% \eeq
% on kpc scales. Future surveys such as the LSST and Euclid will improve these limits and provide new one on larger scales.

Weak lensing provides a direct measure of the distribution of matter, independent of any assumptions about galaxy biasing\footnote{Galaxy bias is the difference in the distribution of galaxies and that of the underlying dark matter. The biasing factor $b$ is defined such that the relative fluctuations in the spatial number density of galaxies are $b$ times the relative density fluctuations.}. Since this distribution can be predicted theoretically, and its amplitude can be directly used to constrain cosmology, weak lensing has great potential as a cosmological probe. The correlation of the density field of nearby galaxies with the lensing shear measured on more distant galaxies is called \textit{galaxy-galaxy lensing}. Most lens systems involve sources (and lenses) at moderate or high redshift, and thus can lensing probe the geometry of the Universe -- the measurement of the shear correlation function as a function of the redshifts of observed galaxies is called \textit{tomography}. The scaling of the galaxy-galaxy lensing signal as a function of the source redshift, known as \textit{cosmography}, depends purely on geometric factors and hence can be used to construct a distance-redshift relation.
%%%%%%%%%%%%%%%%%%%
% TODO
%%%%%%%%%%%%%%%%%%%
% \begin{figure}
% 	\centering
% 		\includegraphics[scale=1.00]{img/WL_abel.jpg}
% 	\caption{(Left) Illustrations of the effect of a lensing mass on a circularly symmetric image. Weak lensing elliptically distorts the image, flexion provides an arc-ness and strong lensing creates large arcs and multiple images. (Right) Galaxy cluster Abell 2218, strongly lensed arcs can be seen in around the cluster. Every background galaxy is weakly lensed. From NASA, ESA, and Johan Richard (Caltech, USA).}
% 	\label{fig_WL_Abell}
% \end{figure}
%%%%%%%%%%%%%%%%%%%%%%%
% Supernovae
%%%%%%%%%%%%%%%%%%%%%%%
\subsection{Large-scale structure}
Studying the large-scale structures (LSS) of the Universe is of a great importance for the cosmology. Since the clustering of matter on scales from galaxies to superclusters came from quantum fluctuations in the very early Universe with important modification by radiation and baryons, the LSS encode critical information about the contents of the Universe, the origin of the fluctuations, and the cosmic expansion background in which the structures evolved.

Measurements of large-scale power spectrum for the spatial distribution of matter as a function of redshift constrain the cosmic expansion history, the cosmological distance scale, the growth rate of structures, the mass of the neutrinos and the abundance of dark matter. This includes BAO measurement of the distance-redshift relation (as a standard ruler). The BAO with the growth of the LSS in the Universe form two robust probes of dark energy, and a potential discriminator between dark energy and modified gravity models. Beyond the dark energy, the large scale power spectrum is a probe of both neutrino mass and primordial non-gaussianity.
\subsubsection{Matter power spectrum}
The matter power spectrum $P(k)$ is defined as a quadratic function of the Fourier transformation of the density contrast $\delta$
\eq{
  \label{eq:pk}
  P(k)(2\pi)^3\delta_{\rm D}(k-k')\equiv \left\langle \hat\delta(k)\hat\delta^*(k')\right\rangle\,,
}
where $\delta_{\rm D}$ is the Dirac delta function and we are averaging over possible realizations. The power spectrum is by far the most common descriptor of clustering in the linear and mildly non-linear regime and plays a central role in cosmology. The power spectrum is the Fourier transform of the correlation function \eqref{eq:corr}
\eq{
    P(k)=\int\xi(r)e^{-ik\cdot r}\dd^3r\,.
}

In \autoref{fig:pk_planck} is shown (linear) matter power spectrum inferred from different cosmological probes \parencite{2018arXiv180706205P}.

\begin{figure}[hbt]
    \centering
    \includegraphics[width=0.7\textwidth]{cosmo_evol/pk_planck.png}
    \caption{The (linear theory) matter power spectrum (at z = 0) inferred from different cosmological probes. The broad agreement of the model (black line) with such a disparate compilation of data, spanning 14 Gyr in time and three decades in scale is an impressive testament to the explanatory power of \LCDM. Earlier versions of similar plots can be found in, for example, \textcite{1994ARA&A..32..319W}, \textcite{1995Sci...268..829S}, \textcite{2002PhRvD..66j3508T}, and \textcite{2004ApJ...606..702T}. A comparison with those papers shows that the evolution of the field in the last two decades has been dramatic, with \LCDM\ continuing to provide a good fit on these scales. \textit{Note:} Reprinted from \textcite{2018arXiv180706205P}.}
    \label{fig:pk_planck}
\end{figure}

%%%%%%%%%%%%%%%%%%%%%%%
% Supernovae
%%%%%%%%%%%%%%%%%%%%%%%
\subsection{Galaxy clusters}
The observed number density and clustering of galaxy clusters as a function of mass and redshift provides a powerful toolset to constraining cosmology.  Galaxy clusters provided the first line of evidence for the existence of dark matter \textcite{zwicky} and cluster mass-to-light ratio measurements suggested that the matter density in the universe was sub-critical \textcite{Gott}. Galaxy clusters measurements are sensitive to both the expansion history and the growth of structure in the Universe enabling to distinguish between dark energy and modified gravity models for cosmic acceleration. Additional probes are measurements of the baryonic mass fraction in clusters, and of the tomographic lensing signatures through clusters.

The basic idea of cluster abundance studies is to compare the predicted space density of massive halos to the observed space density of clusters. The basic observables are the richness, the number of galaxies in a specified luminosity and color range. Halo abundance is sensitive to the amplitude of the matter power-spectrum $\sigma_8$\footnote{$\sigma_8$ is a linear fluctuation in the mass distribution on scales of $8h\mins$ Mpc} and the matter density $\Omega_m$, more precisely a combination of a form $\sigma_8\Omega_m^q$, with $q\approx0.4$ \textcite{white}. The degeneracy between $\sigma_8$ and $\Omega_m$ can be broken by measuring abundances at a variety of masses.
%%%%%%%%%%%%%%%%%%%%%%%
% Supernovae
%%%%%%%%%%%%%%%%%%%%%%%
\subsection{Strong lensing}
Strong gravitational lensing (SL) refers to the multiple imaging of a background object due to a massive foreground object (typically clusters of galaxies) -- see \autoref{fig_WL_Abell}. The resulting angular displacement, morphological distortion, and time delay can be used to measure dark energy parameters. Strong gravitational lensing time delays measure a combination of distances that combining with other dark energy probes can further constraint cosmological parameters. The time delays is also expected to test gravity on scales where the screening mechanisms is becoming active (more about screening mechanisms of modified gravities in \autoref{screen}).

Another independent way to measure dark energy parameters with SL is the analysis of systems with multiple sets of multiple images \textcite{SL_in_CLGs}. The positions of these multiple images depend strongly on the detailed properties of the lens mass distribution and on the angular diameter distance ratios between the lens, source and observer, they encapsulate information about the underlying cosmology. This dependence on the geometry can be used to derive constraints on the cosmological parameters.
%%%%%%%%%%%%%%%%%%%%%%%
% Supernovae
%%%%%%%%%%%%%%%%%%%%%%%
\subsection{Redshift-space distortions}
\label{sec:rsd}
When we observe distant galaxies, two features determine their redshifts -- the Hubble expansion and their peculiar velocities. The peculiar velocities of galaxies thus cause them to appear displaced along the line of sight in redshift space. These displacements lead to redshift distortions in the pattern of clustering of galaxies in redshift space and make large scale galaxy clustering anisotropic. Redshift-space distortions (RSD) have the tremendous advantage of bearing information about the dynamics of galaxies. The strength of the anisotropy is governed by distortion parameter $\beta = f(z)/b(z)$, where $f(z)$ is the logarithmic growth rate of fluctuations. By modeling the full redshift-space galaxy power spectrum one can obtain combination of the product of the matter clustering amplitude and the growth rate.

Anisotropy of galaxy clustering offers an alternative to weak lensing and cluster abundances as a tool for measuring the growth of structures. RSD directly measure the rate at which structure is growing at the redshift of observation unlike WL and galaxy cluster measurements  which measure the rate of growth at multiple redshifts. RSD measurements can improve constraints on dark energy models and they can be used to constrain departures from GR by testing consistency of the growth and expansion histories. The key challenge in modeling RSD is accounting for nonlinear effects, including nonlinear or scale-dependent bias between galaxies and matter, at the level of accuracy demanded by the LSST`s precision.

\section{Standard \LCDM model -- successes and issues}
\todo{Heyrovsky D.:

\textit{On the other hand, what is missing is an initial overview of the standard Lambda CDM model, its successes and outstanding issues (especially since the thesis focuses on developments “beyond” Lambda CDM).}
}

\todo{
\subsection{Dark matter}
Flatness and horizon problems, missing matter in clusters and galaxies
}

\subsection{Cosmic acceleration}
\todo{History, pre-1998, post 1998, $\Lambda$ and alternatives}

\subsection{Cosmological constant}

\todo{Heyrovsky D.:

\textit{
Specific comments on the scientific content: the cosmological-constant term is a natural part of the general form of Einstein’s equations, not a random ad-hoc term; most “problems” with Lambda are connected with its interpretation as vacuum energy density, they do not occur in its interpretation simply as a fundamental constant; the work could have included the observationally most relevant result: the cosmological evolution of the equation of state w(a) predicted by the different modified theories; weak lensing can be analyzed for rich galaxy clusters rather than for single galaxies.}
}

\todo{\parencite[e.g.][]{2015arXiv150205296P, 2012CRPhy..13..566M}}

\todo{Additional arguments against $\Lambda$ -- quantum corrections, etc.}
Standard cosmological \LCDM\ model is in a good agreement with all measurements of CMB \parencite{planck_cosm}, type Ia supernovae \parencite{Abbott_2019}, or BAO \parencite{BAO_results}. However, this concordance model has some significant fundamental problems and even though it may serve as an effective description of our Universe we should be looking for a deeper explanation of the accelerated expansion of the Universe.
\subsubsection{Fine-Tuning Problem}
The first problem could be phrased as the question: \textit{Why is the observed value of $\Lambda$ so small in Planck units?} In order to realize the cosmic acceleration today, we require that the cosmological constant $\Lambda$ is of the order of the square of the present Hubble parameter $H_0$. If we interpret this as an energy density, it is equivalent to
\eq{
\rho_\Lambda=\Lambda \Mpl^2\approx10^{-120} \Mpl^4\approx10^{-47}\mbox{ GeV}^4.
}
Suppose that this energy density comes from the vacuum energy. The zero-point energy of some field of mass $m$ with momentum $k$ is
\eq{
E_0=\left\langle0\left|H\right|0\right\rangle=V\int{\frac{\dd^3k}{(2\pi)^3}\frac12\sqrt{k^2+m^2}},
}
where $V=(2\pi)^3\delta^3(0)$ is the volume of space. If we trust our theory up to some cut-off scale $\Lambda_{UV}\gg m$, we obtain the vacuum energy density
\eq{
\rho_{vac}=\int_0^{\Lambda_{UV}}{\frac{4\pi k^2 \dd k}{(2\pi)^3}}\frac12\sqrt{k^2+m^2}\approx\frac{\Lambda_{UV}^4}{16\pi^2}.
}
If we take the cut-off scale to be order of the Planck scale $\Mpl\approx10^{18}$ GeV, we get a value of about 120 orders of magnitude larger than the observed value. This situation is not better for different scales in particle physics. For the SUSY\footnote{In some supersymmetric theories, the number of fermionic and bosonic degrees of freedom are equal. The energy of the vacuum fluctuations per degree of freedom is the same in magnitude but opposite in sign for fermions and bosons of the same mass. Therefore the fermion and boson contributions cancel each other and the total vacuum energy density (and consequently $\Lambda$) vanishes. As the supersymmetry has to be broken today (we do not observe supersymetry in nature) supersymmetric partners can have different masses of order $\Lambda_{SUSY}\approx10^{3}$ GeV.} scale $\Lambda_{SUSY}\approx10^{3}$ GeV, even for the QCD scale $\Lambda_{QCD}\approx0.1$ GeV, is $\rho_{vac}$ still much larger than $\rho_\Lambda$. Even if the value of $\Lambda$ does not originate from the vacuum energy there is no such a small scale in known physics -- not in GR nor in quantum physics. This huge discrepancy between theoretical predictions and observed value of $\Lambda$ is a major issue in physics and cosmology.
\subsubsection{Coincidence Problem}
The second problem  could be phrased as the question: \textit{Why is the energy density $\rho_\Lambda$ so close to the present matter density?} While the energy density of the cosmological constant (vacuum energy) remains constant in time and was completely negligible in most of the past, the energy density of matter evolves like $\rho_m\propto(1+z)^3$ and will be entirely negligible in the future. According to the Copernican principle \parencite{ellis_maartens_maccallum_2012} we do not live in a special place \textbf{nor time}. Thus it is very unlikely that these two components will have densities of the same order of magnitude in the present. If $\rho^{(0)}_\Lambda/\rho^{(0)}_m$ was just 10 or 100 times smaller, we would not see any accelerated expansion. If it were a few orders of magnitude larger than one, the transition to the accelerated universe would occur at a large redshift.
\\

The so-called anthropic principle can seemingly give us the explanation for both of the two cosmological constant problems, why it is small and why the acceleration starts now. Because if the vacuum energy has been big and dominant from the earlier epoch, there would be no chance to form structures in the Universe, like galaxies, stars, planets and us, intelligent lives. But this anthropic explanation of the value $\rho_\Lambda$ makes sense only if there is a multiverse with a lot different realizations of $\rho_\Lambda$. But we live only in our realization of the Universe and therefore we cannot verify whether the anthropic principle solves the cosmological constant problem or not.

Beside these two problems concerning the cosmological constant, the observations of coherent acoustic oscillations in the CMB \parencite{planck_cosm} has turned the notion of accelerated expansion in the very early universe (inflation) into an integral part of the cosmic standard model. This early accelerated expansion was not due to the cosmological constant, because in that case the inflation would not stop and today`s universe would not be possible. Therefore, we have to postulate some new scalar field (inflaton) \parencite{2015IJMPD..2430010C} that we know so little about. If we do not properly understand the past dynamics of the universe how can we accept the cosmological constant as an explanation of the present acceleration without doubt?